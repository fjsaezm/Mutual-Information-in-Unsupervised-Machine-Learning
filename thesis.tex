
% Plantilla para un Trabajo Fin de Grado de la Universidad de Granada,
% adaptada para el Doble Grado en Ingeniería Informática y Matemáticas.
%
%  Autor original de la plantilla original: Mario Román.
%  Enlace: https://github.com/mroman42/templates
%  Cambios en la plantilla por: Javier Sáez
%  Licencia: GNU GPLv2.
%
% Esta plantilla es una adaptación al castellano de la plantilla
% classicthesis de André Miede, que puede obtenerse en:
%  https://ctan.org/tex-archive/macros/latex/contrib/classicthesis?lang=en
% La plantilla original se licencia en GNU GPLv2.
%
% Esta plantilla usa símbolos de la Universidad de Granada sujetos a la normativa
% de identidad visual corporativa, que puede encontrarse en:
% http://secretariageneral.ugr.es/pages/ivc/normativa
%
% La compilación se realiza con las siguientes instrucciones:
%   pdflatex --shell-escape main.tex
%   bibtex main
%   pdflatex --shell-escape main.tex
%   pdflatex --shell-escape main.tex

% Opciones del tipo de documento
\documentclass[oneside,openright,titlepage,numbers=noenddot,openany,headinclude,footinclude=true, cleardoublepage=empty,abstractoff,BCOR=5mm,paper=a4,fontsize=11pt, dvipsnames]{scrreprt}

% LaTeX packages loaded on start
\usepackage[utf8]{inputenc}
\usepackage[T1]{fontenc}
\usepackage{fixltx2e}
\usepackage{graphicx} % Inclusión de imágenes.
\usepackage{grffile}  % Distintos formatos para imágenes.
\usepackage{longtable} % Tablas multipágina.
\usepackage{wrapfig} % Coloca texto alrededor de una figura.
\usepackage{rotating}
\usepackage[normalem]{ulem}
\usepackage{amsmath}
\usepackage{textcomp}
\usepackage{amssymb}
\usepackage{capt-of}
\usepackage[colorlinks=true]{hyperref}
\usepackage{tikz} % Diagramas conmutativos.
\usepackage{minted} % Código fuente.
\usepackage[T1]{fontenc}
\usepackage{natbib}
\usepackage{caption}
\usepackage{ mathrsfs }



\usepackage[toc,page]{appendix}

%% ---------------------- Self added packages
\usepackage{cancel}
\usepackage{cleveref}
\usepackage{bm}
\usepackage{witharrows}

%% ---------------------------------

% Plantilla classicthesis
\usepackage[beramono,eulerchapternumbers,linedheaders,parts,a4paper,dottedtoc,
manychapters,pdfspacing]{classicthesis}

% Geometría y espaciado de párrafos.
\setcounter{secnumdepth}{0}
\usepackage{enumitem}
\setitemize{noitemsep,topsep=0pt,parsep=0pt,partopsep=0pt}
\setlist[enumerate]{topsep=0pt,itemsep=-1ex,partopsep=1ex,parsep=1ex}
\usepackage[top=1in, bottom=1.5in, left=0.9in, right=1.2in]{geometry}
\setlength\itemsep{0em}
\setlength{\parindent}{0pt}
\usepackage{parskip}

% Algoritmos
\usepackage[ruled,vlined]{algorithm2e}
\newcommand\mycommfont[1]{\footnotesize\ttfamily\textcolor{blue}{#1}}
\SetCommentSty{mycommfont}

% Todo notes
\usepackage{todonotes}
\let\marginpar\oldmarginpar

% Profundidad de la tabla de contenidos.
\setcounter{secnumdepth}{3}

% Usa el paquete minted para mostrar trozos de código.
% Pueden seleccionarse el lenguaje apropiado y el estilo del código.
\usepackage{minted}
\usemintedstyle{colorful}
\setminted{fontsize=\small}
\renewcommand{\theFancyVerbLine}{\sffamily\textcolor[rgb]{0.5,0.5,1.0}{\oldstylenums{\arabic{FancyVerbLine}}}}

% Archivos de configuración.
%------------------------
% Math libraries
%------------------------
\usepackage{amsthm}
\usepackage{amsmath}
\usepackage{tikz}
\usepackage{tikz-cd}
\usetikzlibrary{shapes,fit}
\usepackage{bussproofs}
\EnableBpAbbreviations{}
\usepackage{mathtools}
\usepackage{scalerel}
\usepackage{stmaryrd}

%------------------------
% Estilos para los teoremas
%------------------------
\theoremstyle{plain}
\newtheorem{nth}{Theorem}[section]
% For theorems not in sections
\newtheorem{nthC}{Theorem}[chapter]

\newtheorem{nprop}{Proposition}
\newtheorem{lemma}{Lemma}
\newtheorem{corollary}{Corollary}
\theoremstyle{definition}
\newtheorem{ndef}{Definition}[section]
% For definitions not in sections
\newtheorem{ndefC}{Definition}[chapter]

\newtheorem{nproof}{Proof}

\theoremstyle{remark}
\newtheorem{remark}{Remark}
\newtheorem{nexample}{Example}


\theoremstyle{notation}
\newtheorem{notation}{Notation}


\begingroup\makeatletter\@for\theoremstyle:=definition,ndefC,remark,plain\do{\expandafter\g@addto@macro\csname th@\theoremstyle\endcsname{\addtolength\thm@preskip\parskip}}\endgroup

%------------------------
% Macros
% ------------------------

%Frequently used mathematical commands
\newcommand*\diff{\mathop{}\!\mathrm{d}}
\newcommand{\R}{\mathbb{R}}
% \newcommand{\E}{\mathbb{E}}
\newcommand{\bmu}{\bm{\mu}}
\newcommand{\bx}{\bm{x}}
\newcommand{\bX}{\bm{X}}
\newcommand{\bz}{\bm{z}}
\newcommand{\bZ}{\bm{Z}}
\newcommand{\bv}{\bm{v}}
\newcommand{\bh}{\bm{h}}
\newcommand{\bSigma}{\bm{\Sigma}}
\newcommand{\bpi}{\bm{\pi}}
\newcommand{\bLambda}{\bm{\Lambda}}
\newcommand{\btheta}{\bm{\theta}}

\newcommand{\V}{\mathcal{V}}
\newcommand{\D}{\mathcal{D}}
\newcommand{\X}{\mathcal{X}}
\newcommand{\I}{\mathcal{I}}
\newcommand{\Y}{\mathcal{Y}}

\newcommand\ddfrac[2]{\frac{\displaystyle #1}{\displaystyle #2}}

\newcommand\E[2]{\mathbb{E}_{#1}\Big[#2\Big]}
\newcommand\KL[2]{KL\Big(#1 \bigm| #2\Big)}
\newcommand{\bigCI}{\mathrel{\text{\scalebox{1.07}{$\perp\mkern-10mu\perp$}}}}
\newcommand{\bigCD}{\centernot{\bigCI}}

\DeclareMathOperator*{\argmax}{arg\,max}
\DeclareMathOperator*{\argmin}{arg\,min}

% My own commands:

\newcommand{\Prob}{\mathcal{P}}
\newcommand{\Alg}{\mathscr{A}}
\newcommand{\N}{\mathbb N}
\newcommand{\A}{\mathcal A}
%\newcommand{\abs}[1]{\lvert#1\rvert}  
\newcommand{\rv}{\mathbf{X}}
\newcommand{\rvc}{\mathbf{X} = (X_1,\dots,X_n)}
\newcommand{\pd}{p_{\text{data}}}
\newcommand{\xtk}{x_{t+k}}
\newcommand{\ps}{x^+}
\newcommand{\ns}{x^-}

\newcommand{\norm}[1]{\left\lVert#1\right\rVert}
\newcommand{\abs}[1]{\left\lvert#1\right\rvert}


% Set Graphics Path
\usepackage{float}
\graphicspath{{media/}}

% Math operators

\DeclareMathOperator*{\Var}{Var}

\DeclareMathOperator*{\Cov}{Cov}

% For resnet tables

\newcommand{\blockb}[3]{\left[\begin{array}{c}\text{1$\times$1, #2}\\[-.1em] \text{3$\times$3, #2}\\[-.1em] \text{1$\times$1, #1}\end{array}\right]\times#3}




  % En macros.tex se almacenan las opciones y comandos para escribir matemáticas.
\input{classicthesis-config} % En classicthesis-config.tex se almacenan las opciones propias de la plantilla.

% Color institucional UGR
% \definecolor{ugrColor}{HTML}{ed1c3e} % Versión clara.
\definecolor{ugrColor}{HTML}{c6474b}  % Usado en el título.
\definecolor{ugrColor2}{HTML}{c6474b} % Usado en las secciones.

% Datos de portada
\usepackage{titling} % Facilita los datos de la portada
\author{Francisco Javier Sáez Maldonado}
\date{\today}
\title{Mutual Information in Unsupervised Machine Learning}

% Portada
\include{titlepage}
\usepackage{wallpaper}

\begin{document}

\ThisULCornerWallPaper{1}{media/ugrA4.pdf}
\maketitle


\chapter*{Abstract}

Abstract

\tableofcontents

\ctparttext{
  \color{black}
  \begin{center}
    In this part we will introduce the underlying concepts of probability theory and probability distributions that will be needed.
  \end{center}
}
\part{Basic Concepts}

\chapter{Probability}


Underneath each experiment involving any grade of uncertainty there is a \emph{random variable}. This is no more than a \emph{measurable} function between two \emph{measurable spaces}.
A probability space is composed by three elements: $(\Omega, \Alg, \Prob)$. We will define those concepts one by one.

\section{Basic notions}

\begin{ndef}Let $\Omega$ be a non empty sample space. $\Alg$ is a $\sigma-$algebra over $\Omega$ if it is a family of subsets of $\Omega$ that verify that the emptyset is in $\Alg$, and it is closed under complementation and countable unions. That is:
\begin{itemize}
  \item $\emptyset \in \Alg$
  \item If $A \in \Alg$, then $\Omega \textbackslash A \in \Alg$
  \item If $\{A_i\}_{i \in \mathbb N} \in A$ is a numerable family of $\Alg$ subsets, then $\cup_{i \in \mathbb N} A_i \in \Alg$
\end{itemize}
\end{ndef}


The pair $(\Omega,\Alg)$ is called a \emph{measurable space} To get to our probability space, we need to define a \emph{measure} on the \emph{measurable space}.

\begin{ndef}
Given $(\Omega,\Alg)$, a measurable space, a \emph{measure} $\Prob$ is a countable additive, non-negative set function on this space. That is: $\Prob: \Alg \to \mathbb R_0^+$ satisfying:
\begin{itemize}
  \item $\Prob(A) \geq \Prob(\emptyset) = 0$ for all $A \in \Alg$
  \item $P(\cup_n A_n) = \sum_n P(A_n)$ for any countable collection of disjoint sets $A_n \in \Alg$.
\end{itemize}
\end{ndef}

If $\Prob(\Omega) = 1$, $\Prob$ is a \emph{probability measure} or simply a \emph{probability}. With the concepts that have just been explained, we get to the following definition:

\begin{ndef}
A \emph{measure space} is the tuple $(\Omega, \Alg,\Prob)$ where $\Prob$ is a \emph{measure} on $(\Omega, \Alg)$. If $\Prob$ is a \emph{probability measure} $(\Omega,\Alg,\Prob)$ will be called a \emph{probability space}.
\end{ndef}

Throughout this work, we will be always in the case where $\Prob$ is a probability measure, so we will always be talking about probability spaces. Some notation for these measures must be introduced. Let $A$ and $B$ be two events.
The notation $P(A,B)$ reffers to the probability of the intersection of the events $A$ and $B$, that is: $P(A,B) := P(A\cap B)$.
 It is clear that since $A \cap B = B \cap A$, then $P(A,B) = P(B,A)$. We remark the next definition since it will be important.

\begin{ndef}
Let $A,B$ be two events in $\Omega$. The \emph{conditional probability} of $B$ given $A$ is defined as:
$$
P(B|A) = \frac{P(A,B)}{P(A)}
$$
\end{ndef}



% Introduce here Bayes Theorem
% ------------------------------------------------------------------------------

There is an alternative way to state the definition that we have just made.

\begin{nth}[Bayes' theorem]
Let $A,B$ be two events in $\Omega$, given that $P(B) \neq 0$. Then
$$
P(B|A) = \frac{P(A|B) P(A)}{P(B)}
$$
\end{nth}
\begin{proof}
Straight from the definition of the conditional probability we obtain that:
$$
P(A,B) = P(A|B)P(B)
$$
We also see from the definition that
$$
P(B,A) = P(B|A)P(A)
$$
Hence, since $P(A,B) = P(B,A)$,
$$
P(A|B)P(B) = P(B|A)P(A) \implies P(A|B) = \frac{P(B|A)P(A)}{P(B)}
$$
\end{proof}


However, events might not give any information about another event occurring. When this happens, we call those events to be \emph{independent}. Mathematically, if $A$,$B$ are independent events:
$$
P(A,B) = P(A)P(B)
$$
and as a consequence of this, the conditional probabilty of those events is $P(A|B) = P(A)$.\\


\emph{Random variables} (R.V.) can now be introduced. Their first property is that they are measurable functions. Those kind of functions are defined as it follows:

\begin{ndef}
Let $(\Omega_1, \Alg),(\Omega_2, \mathcal B)$ be measurable spaces. A function $f: \Omega_1 \to \Omega_2$ is said to be \emph{measurable} if, $f^{-1}(B) \in \Alg$ for every $B \in \mathcal B$.
\end{ndef}

As a quick note, we can affirm that if $f,g$ are real-valued measurable functions, and $k \in \mathbb R$, it is true that $kf$, $f+g$ , $fg$ and $f/g$ (if $g$ is not the identically zero function) are also \emph{measurable functions}.

We are now ready to define one of the concepts that will lead us to the main objective of this thesis.

\begin{ndef}[Random variable]
Let $(\Omega,\Alg,\Prob)$ be a probability space, and $(E,\mathcal B)$ be a measurable space. 
A \emph{random variable} is a measurable function $X: \Omega \to E$, from the probability space to the measurable space. This means: for every subset $B \in (E,\mathcal B)$, its preimage
$$
X^{-1}(B) = \{\omega : X(\omega) \in B\} \in \Alg .
$$
\end{ndef}

Using that sums, products and quotients of measurable functions are measurable functions, we obtain that \emph{sums, products and quotients of random variables are random variables}.

Let now $X$ be a R.V. The \emph{probability} of $X$ taking a concrete value on a measurable set contained in $E$, say, $S \in E$, is written as:
$$
P_X(S) = P(X \in S) = P(\{a \in \Omega : X(a) \in S\})
$$

A very simple example of random variable is the following:

\begin{nexample}
  Consider tossing a coin. The possible outcomes of this experiment are \emph{Heads or Tails}. Those are our random events. We can give our random events a possible value. For instance, let \emph{Heads} be $1$ and \emph{Tails} be 0. Then, our random variable looks like this:
  \begin{equation*}
      X  = \left\{ \begin{aligned}
  1 & \text{if we obtain heads} \\
  0 & \text{if we obtain tails}
\end{aligned}\right.
  \end{equation*}

\end{nexample}

In the last example, our random variable is \emph{discrete}, since the set $\{X(\omega): \omega \in \Omega\}$ is finite.
 A \emph{Random Variable} can also be \emph{continuous}, if it can take any value within an interval.\\


\section{Expectation of a random variable}

\begin{ndef}
The \emph{cumulative distribution function } $F_X$ of a real-valued random variable $X$ is its probability of taking value below or equal to $x$. That is:
$$
F_X(x) = P(X \leq x) = P(\{\omega : X(\omega) \leq x\}) = P_X((-\infty,x]) \quad \forall x \in \mathbb R
$$
\end{ndef}

Depending on the image of a random variable $X$, we can difference between certain types of random variables. If the image $\mathcal X$ of $X$ is countable, we call it a \emph{discrete} random variable. Its \emph{probability mass function} gives the
probability of the r.v. being equal to a certain value:
$$
p(x) = P(X = x).
$$
If the image $\mathcal X$ of $X$ is uncountable and real, then $X$ is a \emph{continuous} random variable. In this case there might exist a non-negative Lebesgue-integrable function $f$ such that:
$$
F_X(x) = \int_{\infty}^x f(t) dt,
$$
called the \emph{probability density function} of $X$.\\


Usually, when it comes to applying these concepts to a real problem, we will be looking at multiple variables. We would like to have a collection of random variables each one representing one of this variables.
In order to set the notation for these kinds of situations, we will introduce \emph{random vectors}.

\begin{ndef}
  A random vector is a row vector $\rvc$ whose components are rea-valued random variables on the same probabilty space $(\Omega,\Alg,P)$.
\end{ndef}

The probability distribution of a random variable can be extended in to the \emph{joint probability distribution} of a random vector.

\begin{ndef}
Let $\rvc$ be a random vector. The \emph{cumulative distribution funcion} $F_{\rv} : \R^n \to [0,1]$ of $\rv$ is defined as:
$$
F_{\rv}(x) = P(X_1 \leq x_1 , \dots, X_n \leq x_n)
$$
\end{ndef}

The distribution of each of the component random variables $X_i$ of $\rv$ are called \emph{marginal distributions}.

We would also like to know what are the most probably values that we can obtain out of a random variable.  This is called the \emph{expectation} of a random variable.

\begin{ndef}[Expectation of a \emph{R.V.}]
Let $X$ be a non negative random variable on a probability space $(\Omega,\Alg,\Prob)$. The expectation $E[X]$ of $X$ is defined as:
$$
E[X] = \int_\Omega X(\omega) \ dP(\omega)
$$
\end{ndef}
If $X$ is generic \emph{R.V}, the expectation is defined as:
$$
E[X] = E[X^+] - E[X^-]
$$
where $X^+,X^-$ are defined as it follows:
$$
X^+(\omega) = \max(X(\omega),0) \quad \quad  \quad \quad X^-(\omega) = \min(X(\omega),0)
$$

The \emph{expectation} $E[X]$ of a \emph{random variable} is a linear operation. That is, if $\mathcal Y$ is another random variable, and $\alpha,\beta \in \R$, then
$$
E[\alpha X + \beta \mathcal Y] = \alpha E[X] + \beta E[\mathcal Y]
$$
this is a trivial consequence of the linearity of the \emph{Lebesgue integral}.

As a note, if $X$ is a \emph{discrete} random variable and $\X$ is its image, its expectation can be computed as:
$$
E[X] = \sum_{x \in \X} x  P_X(x)
$$
where $x$ is each possible outcome of the experiment, and $P_X(x)$ the probability under the distribution of $X$ of the outcome $x$. The expression given in the definition before generalizes this particular case.

Using the definition of the \emph{expectation} of a random variable, we can approach to the \emph{moments} of a random variable.

\begin{ndef}
If $k \in \N$, then $E[X^k]$ is called the $k-th$ moment of $X$.
\end{ndef}
If we take $k = 1$, we have the definition of the \emph{expectation}. It is sometimes written as $m_X = E[X]$, and called the \emph{mean}. We use the \emph{mean} in the definition of the variance:

\begin{ndef}
Let $X$ be a random variable. If $E[X^2] < \infty$, then the \emph{variance} of $X$ is defined to be
$$
\Var(X) = E[(X - m_X)^2] = E[X^2] - m_X^2 
$$
\end{ndef}

Thanks to the linearity of the \emph{expectation} of a random variable, it is easy to see that
$$
Var(aX + b) = E[(aX + b) - E[aX + b])^2] = a^2E[(X - m_X)^2] = a^2 \Var(X)
$$


\clearpage
\chapter{Distributions}
We have introduced the concepts of \emph{random variable},  \emph{random vector} and its \emph{probability distribution}.  Now, given two distributions, in the following chapters we will like to see how different they are from each other.
In order to compare them, we enunciate the definition of the Kullback-Leibler divergence.

\begin{ndef}[Kullback-Leibler Divergence]
Let P and Q be probability distributions over the same probability space $\Omega$. Then, the Kullback-Leibler divergence is defined as:
$$
D_{KL}(P \ || \ Q) = E_P\left[\log{\frac{P(x)}{Q(x)}}\right]
$$
\end{ndef}
It is defined if and only if $P$ is \emph{absolutely continuous with respect to} $Q$, that is , if $P(A) = 0$ for any $A$ subset of $\Omega$ where $Q(A) = 0$. There are some properties of this definition that must be stated. The first one is the following proposition:

\begin{nprop}
If P,Q are two probability distributions over the same probability space, then $D_{KL}(P|Q) \geq 0$.
\end{nprop}
\begin{proof}
Firstly, note that if $a \in \R^+$, then $\log \ a \leq a-1$. Then:
\begin{align*}
-D_{KL}(P \ || \ Q) & = - E_P\left[\log{\frac{P(x)}{Q(x)}}\right] \\
             & = E_P\left[\log{\frac{Q(x)}{P(x)}}\right] \\
             & \leq E_P\left[\left(\frac{Q(x)}{P(x)} - 1\right)\right]\\
             & = \int P(x) \frac{Q(x)}{P(x)} dx -1 \\
             & = 0
\end{align*}
So we have obtained that $-D_{KL}(P\ ||\ Q) \leq 0$, which implies that $D_{KL}(P\ || \ Q) \geq 0$.
\end{proof}
As a corollary of this proposition, we can affirm that $D_{KL}(P\ ||\ Q)$ equals zero if and only if $P = Q$ almost everywhere. We will also remark the discrete case, as it will be used later. Let $P,Q$ be discrete probability distributions
defined on the same probability space $\Omega$. Then, 
$$
D_{KL}(P\ ||\ Q) = \sum_{x \in \Omega} P(x) \log \left( \frac{P(x)}{Q(x)}\right)
$$
\clearpage
\chapter{Statistical Inference}
Statistical inference is the process of deducing properties of an underlying distribution by analyzing the data that it is available. With this purpose, techniques like deriving estimates and testing hypotheses are used. 

Inferential statistics are usually contrasted with descriptive statistics, which are only concerned with properties of the observed data. The difference between these two is that in inferential statistics, we assume that the data comes from a larger
population that we would like to know.

In \emph{machine learning}, subject that concerns us the most, the term inference is sometimes used to mean \emph{make a prediction by evaluating an already trained model}, and in this context, inferring properties of the model is refered as \emph{training or learning}.

\section{Parametric Modeling}

In the following chapters, we will be trying to estimate density functions in a dataset. To do this we will be using \emph{parametric models}. We say that a \emph{parametric model}, $P_\theta(x)$, 
is a family of density functions that can be described using a finite numbers of parameters $\theta$. We can get to the concept of \emph{log-likelihood} now.

\begin{ndef}
The \emph{likelihood} $\mathcal L(\theta | x)$ of a parameter set $\theta$ is a function that measures how plausible is $\theta$, given an observed point $x$ in the dataset $\D$. It is defined as the value of the 
density function parametrized by $\theta$ at $x$. That is:
$$
\mathcal L(\theta|x) = P_\theta(x).
$$
\end{ndef}

In a finite dataset $\D$ consisting of independent observations, we can write:
\[
\mathcal L(\theta | X) = \prod_{x \in D} P_\theta(x).
\]

This can be computationally hard to work with, so the log-likelihood is often used instead.

\begin{ndef}
Let $\D$ be a dataset of independent observations and $\theta$ a set of parameters. Then, we define the \emph{log-likelihood} $\ell$ as the sum of the logarithms of the evaluations of $p_\theta$ in each $x$ in the dataset. That is:
\[
\ell (\theta | X) = \sum_{x \in \D} \log P_\theta(x).
\]
\end{ndef}

Our goal would be to find the optimal value $\hat{\theta}$ that maximizes the likelihood of observing the dataset $\D$. We get to the following definition:

\begin{ndef}
    We say that $\hat{\theta} = \hat\theta (\D)$ is a \emph{maximum likelihood estimator}(MLE) for $\theta$ if  
    $$
    \hat\theta \in \argmax_{\theta} \mathcal L(\theta | \D)
    $$
    for every observation $\D$. 
\end{ndef}

\section{Minimal sufficient statistics}

In parametric modeling, the goal was to determine the density function under a distribution. Another interesting task can be determining specific parameters or quantities related to a distribution, given a sample $X = (x_1,\cdots,x_n)$.

\begin{ndef}
    Let $(\Omega,\Alg)$ be a measurable space where $\Alg$ contains all singletons. A statistic is a measurable function of the data, that is: $T: X \to \Omega$ where $T$ is measurable.
\end{ndef}
\begin{remark}
    A statistic is also a random variable.
\end{remark}

However, not all statistics will provide useful information for the statistical inference problem, since almost anything can be a statistic. We would like to find statistics that provide relevant information.

\begin{ndef}
    Let $X \sim P_\theta$. Then, the statistic $T(X) = T : (\Omega, \Alg) \to (\mathbb T, \mathcal B)$, is sufficient for a family of parameters $\{P_\theta \ : \ \theta \in \Theta \}$ if the conditional distribution of $X$, given $T = t$, is indepentent of $\theta$.\\
\end{ndef}

\begin{nexample}
The simplest example of a sufficient statistic is the mean $\mu$ of a gaussian distribution with known variance. Oppositely, the \emph{median} of an arbitrary distribution
is not sufficient for the mean since, even if the median of the sample is known, more information about the mean of the population can be obtained from the mean of the sample itself.
\end{nexample}

Although it will not be shown in this document, sufficient statistics are not unique. In fact, if $T$ is sufficient, $\psi(T)$ is sufficient for any bijective mapping $\psi$. It would be interesting to find a sufficient statistic $T$ that is \emph{the smallest} of them.

\begin{ndef}
    A sufficient statistic $T$ is minimal if, for every sufficient statistic $U$, there exists a mapping $f$ such that $T(x) = f(U(x))$ for any $x \in \Omega$.
\end{ndef}
\clearpage

\ctparttext{
  \color{black}
  \begin{center}
  Information theory is the base for all the following work. In this part, \emph{Mutual Information} will be explained and then, bounds for this function will be given.
  \end{center}
}

\part{Information Theory}
\chapter{Mutual Information}
Obtaining good representations of data is one of the most important tasks in Machine learning. 
Recently, it has been discovered that maximizing \emph{Mutual Information} between two elements in our data can give us good representations for our data. We will go through the basic concepts first.


\section{Entropy}

The \emph{mutual information} concept is based on the \emph{Shannon entropy}, which we will introduce first, along with some basic properties of it. The \emph{Shannon entropy} its a way of measuring the uncertainty in a random variable. Given an event $\mathcal A \in \Omega$, $P$ a probability measure and $P[\A]$ the probability of $\mathcal A$, we can affirm that 
$$
\log\frac{1}{P[\mathcal A]}
$$
describes \emph{how surprising is that $\A$ occurs}. For instance, if $P[\A] = 1$, then the last expression is zero, which means that it is not a surprise that $\A$ occurred. With this motivation, we get to the following definition.


\begin{ndef}
Let $X$ be a discrete random variable with image $\X$. The \emph{Shannon entropy}, or simply \emph{entropy} , $H(X)$ of $X$ is defined as:
$$
H(X) = E_X\left[\log\frac{1}{P_X(X)}\right] =  \sum_{x \in \X} P_X(x) \log\frac{1}{P_X(x)}
$$
\end{ndef}
The \emph{entropy} can trivially be expressed as:
$$
H(X) = - \sum_{x \in \X}P_X (x)\log P_X(x)
$$
There are some properties of the \emph{entropy} that must be remarked. 
\begin{nprop}\label{entr:prop:1}
    Let $X$ be a random variable with image $\X$. then
    $$
0 \leq H(X) \leq \log(|\X|)
    $$
\end{nprop}
\begin{proof}
    Since $\log \ y$ is concave on $\R^+$, by Jensen's inequality, see ~\cref{prop:jensen},:
    $$
    H(X) = - \sum_{x \in X}P_X (x)\log P_X(x) \leq \log\left(\sum_{x \in \X} 1\right) = \log(|\X|)
    $$
    For the lower bound, it is easy to see that, since $P_X(x) \in [0,1] \ \  \forall x \in \X $, $\log P_X(x) \leq 0 \ \ \forall x \in \X$. Then, $-P_X(x) \log P_X(x) \geq 0$ for all $x \in X$ , so $H(X) \geq 0$.
\end{proof}
We can also see that the equality on the left holds if and only if exists $ x $ in  $X$ such that its probability is exactly one, that is $P_X(x) = 1$. The right equality holds if and only if , for all $x \in \X$, its probability is $P_X(x) = \frac{1}{\abs{X}}$.

\subsection*{Conditional entropy}
We have already said that \emph{entropy measures} how surprising is that an event occurs. Usually, we will be looking at two random variables and it would be interesting to see how surprising is that one of them, say $X$, occurred, if we already know that $Y$ occurred. This leads us to the definition of \emph{conditional entropy}. Lets see a simpler case first:

Let $A$ be an event, and $X$ a random variable. The conditional probability $P_{X|A}$ defines the entropy of $X$ conditioned to $ A$:
$$
H(X| A) = \sum_{x \in \X} P_{X|A}(x) \log\frac{1}{P_{X|A}(x)}
$$
If $Y$ is another random variable and $\mathcal Y$ is its image, intuitively we can sum the conditional entropy of an event with all the events in $\mathcal Y$, and this way we obtain the conditional entropy of $X$ given $Y$.
\begin{ndef}[Conditional Entropy]
Let $X,Y$ be random variables with images $\X,\mathcal Y$. The \emph{conditional entropy} $H(X | Y)$ is defined as:

\begin{equation*}
        \begin{split}
    H(X|Y) &  :=   \sum_{y \in \mathcal Y} P_{\mathcal Y}(y) H(X| Y = y)  \\ 
    & = \sum_{y \in \mathcal Y} P_{\mathcal  Y}(y) \sum_{x \in \X} P_{X | Y}(x|y)\log\frac{1}{P_{X|Y}(x|y)}  \\
   & = \sum_{x \in X,y \in \mathcal Y}P_{XY}(x,y)\log\frac{P_Y(y)}{P_{XY}(x,y)}
\end{split}
\end{equation*}



\end{ndef}

The interpretation of the \emph{Conditional Entropy} is simple: the uncertainty in $X$ when $Y$ is given. Since we know about an event that has occurred ($Y$), intuitively the conditional entropy , or the uncertainty of $X$ occurring given that $Y$ has occurred, will be lesser than the entropy of $X$, since we already have some information about what is happening. We can prove this:

\begin{nprop}\label{entr:prop:2}
Let $X,Y$ be random variables with images $\mathcal X, \mathcal Y$. Then:
$$
0 \leq H(X|Y) \leq H(X)
$$
\end{nprop}
\begin{proof}

The inequality on the left was proved on Proposition \cref{entr:prop:1}. The characterization of when $H(X|Y) = 0$ was also mentioned after it.  Let's look at the inequality on the right. Note that, restricting to the $(x,y)$ where $P_{XY}(x,y) > 0$ and using the definition of the conditional probability we have:
\begin{align*}
H(X|Y) = & \sum_y P_Y(y) \sum_x P_{X|Y}(x|y)\log \frac{1}{P_{X|Y}(x|y)}\\ = & \sum_{x,y} P_Y(y) P_{X|Y}(x,y) \log \frac{P_Y(y)}{P_{XY}(x,y)} = \sum_{x,y} P_{XY}(x,y)\log \frac{P_Y(y)}{P_{XY}(x,y)} 
\end{align*}
and 
$$
H(X) = \sum_x P_X(x) \log \frac{1}{P_X(x)} = \sum_{x,y}P_{XY}(x,y) \log \frac{1}{P_X(x)}
$$
hence,
\begin{equation}\label{eq:dif-expr-mi}
H(X|Y) - H(X) = \sum_{x,y}P_{XY}(x,y) \left( \log \frac{P_Y(y)}{P_{XY}(x,y)} - \log \frac{1}{P_X(x)}\right) = \sum_{x,y}P_{XY}\log \frac{P_Y(y)P_X(x)}{P_{XY}(x,y)}
\end{equation}
so, using Jensen's Inequality , we obtain:
\begin{align*}
\sum_{x,y}P_{XY}\log \frac{P_Y(y)P_X(x)}{P_{XY}(x,y)} \leq & \log \left( \sum_{x,y}\frac{ \cancel{P_{XY}(x,y)} \ \  P_Y(y) P_X(x)}{\cancel{P_{XY}(x,y)}} \right) \\ = & \log\left( \left( \sum_x P_X(x) \right) \left(\sum_y P_Y(y)\right)\right) = \log 1 = 0
\end{align*}
and this leads us to:
$$
H(X|Y) - H(X) \leq 0 \implies H(X|Y) \leq H(X)
$$
as we wanted.
\end{proof}

It must be noted that, on the development of $H(X|Y) - H(X)$, in the first inequality, equality holds if and only if $P_{XY}(x,y) = P_X(x) P_Y(y)$ for all $(x,y)$ with $P_{XY} (x,y) > 0$, as it is said in Jensen's inequality. For the second inequality, equality holds if and only if $P_{XY}(x,y) = 0$, which implies $P_X(x)P_Y(y) = 0$ for any $x\in \mathcal X$, $y \in \mathcal Y$. It follows that $H(X|Y) = H(X)$ if and only if $P_{XY}(x,y) = P_X(x)P_Y(y)$ for all $(x,y) \in \mathcal X \times \mathcal Y$
% Do I have to define it for the continuous case?
% Will I use the continuous or the discrete case?

\section{Mutual Information}

Using the \emph{entropy} of a random variable we can directly state the definition of \emph{Mutual Information} as it follows:

\begin{ndef}[Mutual Information]
Let $X,Z$ be random variables. The \emph{Mutual Information (MI)} is expressed as the difference between the entropy of $X$ and the conditional entropy of $X$ and $Z$, that is:
$$
I(X,Z) := H(X) - H(X|Z)
$$
\end{ndef}

Since the entropy of the random variable $H(X)$ explains the uncertainty of $X$ occurring, the intuitive idea of the \emph{MI} is to determine the decrease of uncertainty of $X$ occurring when we already
know that $Z$ has occurred. We also have to note that, using the definition of the \emph{entropy} and the expression obtained in \ref{eq:dif-expr-mi}, we can rewrite the \emph{MI} as it follows:
\begin{align*}
I(X,Z) & = \sum_{x \in \X}P_X(x) \log \frac{1}{P(x)} - \sum_{x \in \X, z \in \mathcal Z} P_{XZ}(x,z) \log \frac{P_Z(x)}{P_{XZ}(x,z)} \\  & = \sum_{x,z}P_{XZ}\log \frac{P_Z(z)P_X(x)}{P_{XZ}(x,z)} = D_{KL}(P_{XZ} \ || \ P_X P_Z)
\end{align*}
and we have obtained an expression of the mutual information using the \emph{Kullback-Leibler} divergence. This provides with the following immediate consequences:
\begin{enumerate}[label=$(\roman*)$]
\item Mutual information is non-negative , that is : $I(X,Z) \geq 0$.
\item If $X,Z$ are random variables, then its mutual information equals zero if and only if they are independent. This is trivial because if $D_{KL}(P_{XZ} \ || \ P_X P_Z) = 0$, then $P_{XZ} = P_X P_Z$ almost everywhere so $X$ and $Z$ are independent.
\item Since $P_{XZ} = P_{ZX}$ and $P_X P_Z = P_Z P_X$, mutual information is symmetric. That is: $I(X,Z) = I(Z,X)$.
\end{enumerate}
\clearpage

\ctparttext{
  \color{black}
  \begin{center}
  Clarification of this part
  \end{center}
}
\part{Representation Learning}
\chapter{Context}



Before continuing presenting the mathematical notions of the topics that are treated in this work, it is interesting to present what we are pursuing with this work.\\

\emph{Machine Learning} is the part of computer science that studies \emph{algorithms} that improve automatically through experience from examples. These algorithms help computer to discover how to
perform tasks without being explicitly programmed to do them. For the computers to learn, it is mandatory that a finite set of data (or dataset) $\mathcal D$ is available. \\

Depending on how the data (\emph{or signal}) is given to the computer, the machine learning approaches can be divided in three broad categories:
\begin{enumerate}
    \item \emph{Supervised learning}. In this category each point $x_i \in \mathcal D$ in the dataset is \emph{labeled}: each example is related to a tag $y_i \in Y$ that gives information about $x$. The goal in this case is to find 
    a function $g:D \to Y$.
    \item \emph{Unsupervised learning}. In ths case, the data is \emph{unlabeled}, so the approach is completely different. Usually, the goal here is to discover hidden patterns in data or to learn features from it.
    \item \emph{Reinforcement learning}. This is the area concerned with how intelligent agents take decisions in an an specific environment in order to obtain the best reward in their objective.
\end{enumerate}

In this work, we will focus on unsupervised learning. Particularly, in representation learning. In the learning process, machine learning models can not directly give labels to input examples. Before, they must create a \emph{representation} that 
contains the data's key qualities.  Here is where \emph{representation learning} is born. 

% Insert the definition of representation

Representation learning is a set of techniques that allows a system to discover the representations needed for feature detection or classification. 
In contrast to manual feature engineering, feature learning allows a machine to learn the features and to use them to perform a task.\\

Feature learning can be supervised or unsupervised. In supervised feature learning, representations are learned using labeled data.
Examples of this kind of feature learning are supervised neural networks and multilayer perceptron. In unsupervised learning, the features are learned using unlabeled data. 
There are many examples of this, such as independent component analysis (ICP) and autoencoders. In this work, we will be working with unsupervised feature learning.\\

The performance of machine learning methods is heavily dependent on the choice of data features \cite{bengio_representation_2014}. This is why most of the current 
effort in machine learning focuses on designing preprocessing and data transformation that lead to good quality representations. A representation will be of good quality when its features
produce good results at running the models.\\

The main goal in representation learning is to obtain features of the data that are generally good for any supervised task. That is, we would like to obtain
a representation that is either good for image classification (giving an image a label of what we can see in it) or image captioning (producing a text that describes the image).\\

Data's features that are invariant through time are very useful for machine learning models. In \cite{wiskott_slow_2002}, \emph{slow features} are presented. Slow features are defined as features of a signal 
(which can be the input of a model) that vary slowly during time. These kind of features are the most interesting ones when creating representations, since they give an abstract view of the original data.\\

Let us give an example: In computer vision, the value of the pixels in an image can vary fastly. For instance, if we have a zebra on a video and the zebra is moving from one side of the image to the other, due 
to the black stripes of this animal, the pixels will fastly change from black to white and viceversa, so value of pixels is probably not a good feature to choose as an slow feature. However, there will always
be a zebra on the image, so the feature that indicates that there is a zebra on the image will stay positive throughout all the video, so we can say that this is a slow feature.\\

In the following chapters, we will explain 



\clearpage
\chapter{Generative Models}

\label{Chapter:Gen:Models}
The vast majority of the problems in ML are usually of a discriminative nature, which is almost a synonym of supervised learning. However, there also exist problems that involve learning how to generate new examples of the data. More formally:

\begin{ndefC}
\begin{enumerate}
\item \emph{Discriminative models} estimate $p(y|x)$, the probability of a label $y$ given an observation $x$.
\item \emph{Generative models} estimate $p(x)$, the probability of observing the datapoint $x$. If the dataset is labeled, a generative model can also estimate the distribution $p(x|y)$.
\end{enumerate}

\end{ndefC}


From now on, let $\D$ be any kind of observed data. This will always be a finite subset of samples taken from a probability distribution $\pd$. There are models that, given $\D$, try to approximate the 
probability distribution that lies underneath it. These are called \emph{generative models (G.M.)}. 

Generative models can give parametric and non parametric approximations to the distribution $\pd$. 
In our case, we will focus on parametric approximations where the model searches for the parameters that minimize a chosen metric (which can be a distance or other kind of metric such as K-L divergence) between the model distribution and the data distribution. 

We can express our problem more formally as follows. Let $\theta$ be a generative model within a model family $\mathcal M$. The goal of generative models is to optimize:
$$
\min_{\theta \in \mathcal M} d(\pd,p_\theta),
$$
where $d$ stands for the distance between the distributions. We can use, for instance, K-L divergence.

Generative models have many useful applications. We can however remark the tasks that we would like our generative model to be able to do. Those are:
\begin{itemize}
\item Estimate the density function: given a datapoint, $x \in D$, estimate the probability of that point $p_\theta(x)$.
\item Generate new samples from the model distribution $x \sim p_\theta(x)$.
\item Learn useful features of the datapoints.
\end{itemize}

If we have a look again at the example of the zebras, if we make our generative model learn about images of zebras, we will expect our $p_\theta(x)$ to be high for zebra's images. We will also expect the model
to generate new images of this animal and to learn different features of the animal, such as their big size in comparison with cats.

\section{Autoregressive Models}

In time-series theory, autoregressive models use observations from previous time steps to predict values at the current time. 
Fixing an order of the variables $x_1,\dots,x_n$, the distribution for the $i$-th random variable depends on all the preceding values in the particular chosen order. We will make use of the name of these models to 
define the machine learning approach.

A very first definition of \emph{autoregressive models (AR)} would be the following one: \emph{autoregressive models are feed-forward models that predict future values using past values}. Let us go deeper into this 
concept and explain how it behaves.

Again, let $\D$ be a set of $n-$dimensional datapoints $x$. We can assume that $x \in \{0,1\}^n$ for simplicity, without losing generality. If we choose any $x\in \D$, using the chain rule of probability, we obtain
\[
p(x) = \prod_{i=1} ^n p(x_i | x_1,\dots,x_{i-1}) = \prod_{i = 1}^n p(x_i|\bm{x}_{<i}),
\]
where $\bm{x}_{<i} \in \R^{i-1}$ is a vector whose components are the previous $x_j$ for $j = 1,\dots, i-1$, that is: $\bm{x}_{<i}= [x_1,\dots, x_{i-1}]$. \\
It is known that given a set of discrete and mutually dependent random variables, they can be displayed in a table of conditional probabilities. If $K_i$ is the number of states that each random variable can take
then $\prod K_i$ is the number of cells that the table will have. If we represent $p(x_i|\bm{x}_{<i})$ for every $i$ in tabular form, we can represent
any possible distribution over $n$ random variables. 

This, however, will cause an exponential growth on the complexity of the representation, due to the need of specifying $2^{n-1}$ possibilities 
for each case. In terms of neural networks, since each column must sum $1$ because we are working with probabilities, we have $2^{n-1}-1$ parameters for this conditional, and the tabular representation
becomes impractical for our network to learn when $n$ increases.

In autoregressive generative models, the conditionals are specified as we have mentioned before: parameterized functions with a fixed numbers of parameters. More precisely,  we assume 
the conditional distributions to be Bernoulli random variables and learn a function $f_i$ that maps these random variables to the mean of the distribution. Mathematically, we have to find 
$$
p_{\theta_i}(x_i | \bm{x}_{<i}) = \operatorname{Bern}(f_i(x_1,\dots,x_{i-1})),
$$
where $\theta_i$ is the set of parameters that specify the mean function $f_i:\{0,1\}^{i-1} \to [0,1]$.

Then, the number of parameters is reduced to $\sum_{i=1}^n \abs{\theta_i}$ so we can not represent all possible distributions as we could when using the tabular form of the conditional probabilities.
We are now setting the limit of its expressiveness because we are setting the conditional distributions $p_{\theta_i}(x_i|\bm{x}_{<i})$ to be \emph{Bernoulli} random variables with the mean specified by a restricted class 
of parametrized functions. 

Let us see a very simple case first in order to understand it better. Let $\sigma$ be a \emph{sigmoid}\footnotemark non linear function and 
$\theta_i = \left\{\alpha_{0}^{(i)},\alpha_{1}^{(i)},\dots, \alpha_{i-1}^{(i)}\right\}$ the parameters of the mean function. Then, we can define our function $f_i$ as :
$$
f_i(x_1,\dots, x_{i-1}) = \sigma(\alpha_{0}^{(i)} + \alpha_{1}^{(i)}x_1 + \dots + \alpha_{i-1}^{(i)}x_{i-1}).
$$
%------------- Footnotemark
\footnotetext{A sigmoid function is a bounded,differentiable, real function which derivative is non-negative at each point and it has exactly one inflection point.}
%----------------------
In this case, the number of parameters would be $\sum_{i = 1}^n i = \frac{n(n+1)}{2}$, so using \emph{Big }$O$ notation, we would be in the case of $O(n^2)$. We will state now a more general and useful case,
giving a more interesting parametrization for the mean function: \emph{multi layer perceptrons}\footnotemark (MLP).

%------------- Footnotemark
\footnotetext{Multi layer perceptrons are feed-forward neural networks with at least 3 layers: input, hidden and output layers; each one using an activation
function.}
%----------------------

For this example we will consider the most simple MLP: the one with one hidden layer. Let $h_i = \sigma(\bm{A}_i \bm{x}_{<i} + c_i)$ be the hidden layer activation function. Remember that $h_i \in \R^d$. Let
$ \theta_i = \{ \bm{A}_i \in \R^{d \times (i-1)}, \ c_i \in \R^d, \ \alpha^{(i)} \in \R^d, \ b_i \in \R\}$ the set of parameters
for the mean function $f_i$, that we define as:
$$
f_i(\bm{x}_{<i}) = \sigma(\alpha^{(i)}h_i + b_i).
$$
In this case, the number of parameters will be $O(n^2 d)$.

% Link NADE http://proceedings.mlr.press/v15/larochelle11a/larochelle11a.pdf

% LINK RNADE https://arxiv.org/pdf/1306.0186.pdf
This is the simplest example. Currently, there are alternative parametrization models , such as the \emph{Neural Autoregressive Density Estimator} \citep{larochelle_neural_nodate}, that provide a more statistically and computationally efficient solution. In fact, the number of parameters is reduced from $O(n^2 d)$ to $O(nd)$. Also, \emph{RNADE} \citep{uria_rnade_2014} extends NADE to learn generative models over real-valued data, generalizing the case that we have just exposed. However, these models are out of the scope of this project so no further explanation will be given.


\clearpage
\chapter{The InfoNCE Loss}
We are now ready to connnect the concepts of mutual information and generative models that we have presented. In unsupervised learning,
it is a common strategy to predict future information and to try to find out if our predictions are correct.
In \emph{natural language processing},for instance, representations are learned 
using neighbouring words (https://arxiv.org/pdf/1301.3781.pdf), and in images, some studies have been able to predict color from grey-scale (https://arxiv.org/pdf/1505.05192.pdf).

When we talk about high-dimensional data, it is not useful to make use of an unimodal loss function to evaluate our model. If we did it like this, we would be assuming that there is only
one peak in the distribution function and that it is actually similar to a Gaussian.  This is not always true, so we can not assume it for our models. Generative models can be used for this purpose:
they will model the relationships in the data $x$. However,they ignore the context $c$ in which the data $x$ is involved. As an easy example of this, an image contains thousands of bits of information,
while the label that classifies the image contains much less information , say, $10$ bits for $1024$ categories. 

Our goal here will be to seek for a way of extracting shared information between the context and the data. Due to the differences in data dimensionality, if we want to predict the future $x$ using the context $c$, firstly we must
encode our entry data $x$ into a representation which size is comparable to context size. Firstly, an \emph{encoder} is used. An encoder is a model that, given an input $x$, provides a feature map or vector that holds the information
that the input $x$ had. In fact, and here is where we link the mutual information with the current topic, we want our encoder to maximize
$$
I(x,c) = \sum{x,c}p(x,c)\log\frac{p(x|c)}{p(x)}
$$
that is, the mutual information between the input $x$ and the context $c$.  Maximizing the mutual information between $x$ and $c$, we extract the latent variables
that the inputs have in common.
\clearpage

\ctparttext{
  \color{red}
  \begin{center}
    Appendices
  \end{center}
}
\part{APPENDIX A}
\label{APPENDIX:A}

This appendix will be used to set forth some theoretical results that might not always be relevant but are needed to understand some details during this thesis. Not all of them will be proven.

\begin{nprop}[Jensen's Inequality]\label{prop:jensen}
    Let $f:\mathcal D \to \R$ be a concave function and $n\in \mathbb N$. For any $p_1,\dots,p_n \in \R^+_0$ with $\sum p_i = 1$ and any $x_1,\dots,x_n \in \mathcal D$, it holds that:
    $$
    \sum_{i = 1}^n p_i f(x_i) \leq f \left(\sum_{i=1}^n p_i x_i\right).
    $$
    Furthermore, if $f$ is \emph{strictly} concave and $p_i \geq 0$ for all $i  = 1,\dots,n$, then the equality holds if, and only if, $x_1 = \dots = x_n$.
\end{nprop}



\nocite{*}
\bibliographystyle{authordate1}
\bibliography{Bibliography}

\end{document}
