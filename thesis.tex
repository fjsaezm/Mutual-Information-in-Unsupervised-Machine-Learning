
% Plantilla para un Trabajo Fin de Grado de la Universidad de Granada,
% adaptada para el Doble Grado en Ingeniería Informática y Matemáticas.
%
%  Autor original de la plantilla original: Mario Román.
%  Enlace: https://github.com/mroman42/templates
%  Cambios en la plantilla por: Javier Sáez
%  Licencia: GNU GPLv2.
%
% Esta plantilla es una adaptación al castellano de la plantilla
% classicthesis de André Miede, que puede obtenerse en:
%  https://ctan.org/tex-archive/macros/latex/contrib/classicthesis?lang=en
% La plantilla original se licencia en GNU GPLv2.
%
% Esta plantilla usa símbolos de la Universidad de Granada sujetos a la normativa
% de identidad visual corporativa, que puede encontrarse en:
% http://secretariageneral.ugr.es/pages/ivc/normativa
%
% La compilación se realiza con las siguientes instrucciones:
%   pdflatex --shell-escape main.tex
%   bibtex main
%   pdflatex --shell-escape main.tex
%   pdflatex --shell-escape main.tex

% Opciones del tipo de documento
\documentclass[oneside,openright,titlepage,numbers=noenddot,openany,headinclude,footinclude=true, cleardoublepage=empty,abstractoff,BCOR=5mm,paper=a4,fontsize=11pt, dvipsnames]{scrreprt}

% LaTeX packages loaded on start
\usepackage[utf8]{inputenc}
\usepackage[T1]{fontenc}
\usepackage{fixltx2e}
\usepackage{graphicx} % Inclusión de imágenes.
\usepackage{grffile}  % Distintos formatos para imágenes.
\usepackage{longtable} % Tablas multipágina.
\usepackage{wrapfig} % Coloca texto alrededor de una figura.
\usepackage{rotating}
\usepackage[normalem]{ulem}
\usepackage{amsmath}
\usepackage{textcomp}
\usepackage{amssymb}
\usepackage{capt-of}
\usepackage[colorlinks=true]{hyperref}
\usepackage{tikz} % Diagramas conmutativos.
\usepackage{minted} % Código fuente.
\usepackage[T1]{fontenc}
\usepackage{natbib}
\usepackage{caption}
\usepackage{ mathrsfs }



\usepackage[toc,page]{appendix}

%% ---------------------- Self added packages
\usepackage{cancel}
\usepackage{cleveref}
\usepackage{bm}
\usepackage{witharrows}
\usepackage{algorithmic}
\usepackage{ upgreek }
\usepackage{wrapfig}
\usepackage{multirow}
\usepackage{svg}


%License

% License
\usepackage[
    type={CC},
    modifier={by-sa},
    version={4.0},
]{doclicense}

%% ---------------------------------

% Plantilla classicthesis
\usepackage[beramono,eulerchapternumbers,linedheaders,parts,a4paper,dottedtoc,
manychapters,pdfspacing]{classicthesis}

% Geometría y espaciado de párrafos.
\setcounter{secnumdepth}{0}
\usepackage{enumitem}
\setitemize{noitemsep,topsep=0pt,parsep=0pt,partopsep=0pt}
\setlist[enumerate]{topsep=0pt,itemsep=-1ex,partopsep=1ex,parsep=1ex}
\usepackage[top=1in, bottom=1.5in, left=0.9in, right=1.2in]{geometry}
\setlength\itemsep{0em}
\setlength{\parindent}{0pt}
\usepackage{parskip}

% Algoritmos
\usepackage[ruled,vlined]{algorithm2e}
\newcommand\mycommfont[1]{\footnotesize\ttfamily\textcolor{blue}{#1}}
\SetCommentSty{mycommfont}

% Todo notes
\usepackage{todonotes}
\let\marginpar\oldmarginpar

% Profundidad de la tabla de contenidos.
\setcounter{secnumdepth}{3}

% Usa el paquete minted para mostrar trozos de código.
% Pueden seleccionarse el lenguaje apropiado y el estilo del código.
\usepackage{minted}
\usemintedstyle{colorful}
\setminted{fontsize=\small}
\renewcommand{\theFancyVerbLine}{\sffamily\textcolor[rgb]{0.5,0.5,1.0}{\oldstylenums{\arabic{FancyVerbLine}}}}

% Archivos de configuración.
%------------------------
% Math libraries
%------------------------
\usepackage{amsthm}
\usepackage{amsmath}
\usepackage{tikz}
\usepackage{tikz-cd}
\usetikzlibrary{shapes,fit}
\usepackage{bussproofs}
\EnableBpAbbreviations{}
\usepackage{mathtools}
\usepackage{scalerel}
\usepackage{stmaryrd}

%------------------------
% Estilos para los teoremas
%------------------------
\theoremstyle{plain}
\newtheorem{nth}{Theorem}[section]
% For theorems not in sections
\newtheorem{nthC}{Theorem}[chapter]

\newtheorem{nprop}{Proposition}
\newtheorem{lemma}{Lemma}
\newtheorem{corollary}{Corollary}
\theoremstyle{definition}
\newtheorem{ndef}{Definition}[section]
% For definitions not in sections
\newtheorem{ndefC}{Definition}[chapter]

\newtheorem{nproof}{Proof}

\theoremstyle{remark}
\newtheorem{remark}{Remark}
\newtheorem{nexample}{Example}


\theoremstyle{notation}
\newtheorem{notation}{Notation}


\begingroup\makeatletter\@for\theoremstyle:=definition,ndefC,remark,plain\do{\expandafter\g@addto@macro\csname th@\theoremstyle\endcsname{\addtolength\thm@preskip\parskip}}\endgroup

%------------------------
% Macros
% ------------------------

%Frequently used mathematical commands
\newcommand*\diff{\mathop{}\!\mathrm{d}}
\newcommand{\R}{\mathbb{R}}
% \newcommand{\E}{\mathbb{E}}
\newcommand{\bmu}{\bm{\mu}}
\newcommand{\bx}{\bm{x}}
\newcommand{\bX}{\bm{X}}
\newcommand{\bz}{\bm{z}}
\newcommand{\bZ}{\bm{Z}}
\newcommand{\bv}{\bm{v}}
\newcommand{\bh}{\bm{h}}
\newcommand{\bSigma}{\bm{\Sigma}}
\newcommand{\bpi}{\bm{\pi}}
\newcommand{\bLambda}{\bm{\Lambda}}
\newcommand{\btheta}{\bm{\theta}}

\newcommand{\V}{\mathcal{V}}
\newcommand{\D}{\mathcal{D}}
\newcommand{\X}{\mathcal{X}}
\newcommand{\I}{\mathcal{I}}
\newcommand{\Y}{\mathcal{Y}}

\newcommand\ddfrac[2]{\frac{\displaystyle #1}{\displaystyle #2}}

\newcommand\E[2]{\mathbb{E}_{#1}\Big[#2\Big]}
\newcommand\KL[2]{KL\Big(#1 \bigm| #2\Big)}
\newcommand{\bigCI}{\mathrel{\text{\scalebox{1.07}{$\perp\mkern-10mu\perp$}}}}
\newcommand{\bigCD}{\centernot{\bigCI}}

\DeclareMathOperator*{\argmax}{arg\,max}
\DeclareMathOperator*{\argmin}{arg\,min}

% My own commands:

\newcommand{\Prob}{\mathcal{P}}
\newcommand{\Alg}{\mathscr{A}}
\newcommand{\N}{\mathbb N}
\newcommand{\A}{\mathcal A}
%\newcommand{\abs}[1]{\lvert#1\rvert}  
\newcommand{\rv}{\mathbf{X}}
\newcommand{\rvc}{\mathbf{X} = (X_1,\dots,X_n)}
\newcommand{\pd}{p_{\text{data}}}
\newcommand{\xtk}{x_{t+k}}
\newcommand{\ps}{x^+}
\newcommand{\ns}{x^-}

\newcommand{\norm}[1]{\left\lVert#1\right\rVert}
\newcommand{\abs}[1]{\left\lvert#1\right\rvert}


% Set Graphics Path
\usepackage{float}
\graphicspath{{media/}}

% Math operators

\DeclareMathOperator*{\Var}{Var}

\DeclareMathOperator*{\Cov}{Cov}

% For resnet tables

\newcommand{\blockb}[3]{\left[\begin{array}{c}\text{1$\times$1, #2}\\[-.1em] \text{3$\times$3, #2}\\[-.1em] \text{1$\times$1, #1}\end{array}\right]\times#3}




  % En macros.tex se almacenan las opciones y comandos para escribir matemáticas.
\input{classicthesis-config} % En classicthesis-config.tex se almacenan las opciones propias de la plantilla.

% Color institucional UGR
% \definecolor{ugrColor}{HTML}{ed1c3e} % Versión clara.
\definecolor{ugrColor}{HTML}{c6474b}  % Usado en el título.
\definecolor{ugrColor2}{HTML}{c6474b} % Usado en las secciones.

% Datos de portada
\usepackage{titling} % Facilita los datos de la portada
\author{Francisco Javier Sáez Maldonado}
\date{\today}
\title{Mutual Information in Self-Supervised Learning}

% Portada
\include{titlepage}
\usepackage{wallpaper}

\begin{document}

\ThisULCornerWallPaper{1}{media/ugrA4.pdf}
\maketitle


\vspace*{\fill}
\doclicenseThis
The source code of this text and developed programs are available in the Github repository \href{https://github.com/fjsaezm/Mutual-Information-in-Unsupervised-Machine-Learning}{fjsaezm/Mutual-Information-in-Unsupervised-Machine-Learning}

\chapter*{Abstract}
\emph{Representation learning} is the process of, given an input data of any type, training a model that learns to produce a lower dimensional vector that summarizes the information contained in the input. This document will focus on the main used method for representation learning.

Firstly, an introduction to the basic probability concepts is made, along with an introduction on the contrastive noise estimation problem, which will be key in the construction of the \emph{InfoNCE} loss and great inspiration for the contrastive methods that currently achieve the state of art results in this field.

In this context,\emph{mutual information} appears as a good method to obtain good representations by maximizing this function between the input and its representation. We review the most important properties of this measure and, since it is hard to explicitly compute it, we present some lower bounds on this function which can make the task easier.

Lastly, it was empirically shown that maximizing mutual information is not the best way of obtaining good representations for downstream tasks. New frameworks (SimCLR, BYOL) that present different perspectives of the same input to the models have shown to perform better, so we present these frameworks and review what choices have to be explored in order to achieve the best performance using these kind of methods.




\textbf{Keywords:} \emph{representation learning, noise contrastive estimation, entropy, mutual information, lower bounds, siamese networks, triplet loss} and \emph{deep learning}.
\chapter*{Resumen extendido en español}
En este trabajo, trataremos de exponer cómo ha ido evolucionando el \emph{aprendizaje de representaciones} desde los primeros métodos usados hasta los nuevos marcos de trabajo usados en este campo. En este trabajo, realizaremos un estudio \textbf{teórico} de los conceptos necesarios para aproximarnos al problema de aprendizaje de representaciones (partes 1 y 2) y, a continuación y de forma algo más \textbf{práctica}, presentaremos algunos marcos de trabajo utilizados y realizaremos una serie de experimentos utilizando estos marcos (partes 3 y 4).

La parte \textbf{teórica} describe y profundiza los conceptos más importantes y en los que se fundamenta la parte prática.

En la \textbf{primera parte} se comienza haciendo una introducción y motivación profunda al tema (capítulo 1), y se sigue en un extenso capítulo (capítulo 2) en el que se describen todos los conceptos fundamentales para comprender este trabajo. Se dan primero las nociones más básicas sobre probabilidad como el de variable aleatoria, su esperanza y el concepto de {divergencia de Kullback-Leibler}, que será muy relevante pues será una de las formas que tengamos de expresar la \emph{Información Mutua}. Seguidamente, se dan algunas nociones algo más complejas sobre inferencia estadística, como el de función de verosimilitud y los modelos generativos. Se termina este capítulo realizando una  introdución al aprendizaje profundo  resaltando el \emph{aumento de datos}, que proporciona nuevos ejemplos a partir de los datos obtenidos y que juega un papel crucial en el aprendizaje de representaciones.

A continuación (capítulo 3), nos dedicamos a presentar  los conceptos más importantes de la \emph{teoría de la información} en los que se basa nuestro trabajo. Primeramente se expone el concepto de  \emph{entropía} y se dan ciertas propiedades de la misma. Estas propiedades son útiles pues cuando definimos la \emph{información mutua}, al definirse esta en función de la entropía, se pueden extrapolar sin ningún esfuerzo a propiedades de la información mutua. Se introducen además dos de las tres principales \emph{cotas inferiores} para la información mutua: la \emph{cota inferior variacional} y la cota inferior usando la\emph{representación Donsker-Varadhan} de divergencia de Kullback-Leibler. Estas cotas son muy útiles a la hora de hacer aproximaciones al valor real de la información mutua.


La \textbf{segunda parte} está dedicada a estudiar de forma profunda el problema del aprendizaje contrastivo, cómo surgió y cómo ha evolucionado. Es necesario para esto comenzar (capítulo 4) dando una descripción detallada de cómo se ajusta un modelo de regresión logística, pues se utiliza para resolver el problema de la \emph{estimación de ruido contrastiva}, teoría en la que basamos el aprendizaje contrastivo. Más adelante (capítulo 5), se introduce el marco de trabajo del aprendizaje contrastivo en el que, fijado un conjunto de datos, se trata de discriminar entre datos obtenidos de dos distribuciones de probabilidad $P$ y $Q$ diferentes. Para ello, se utiliza la función de pérdida contrastiva, que se demostrará empíricamente que es clave en el aprendizaje de representaciones. Además, se demuestra la última cota inferior que daremos para la información mutua, la \emph{cota contrastiva}. Por último en esta parte, se introducen las funciones de pérdida utilizando tripletas (capítulo 6), que son una generalización de la función de pérdida contrastiva y se demuestra cómo obtener la función de pérdida contrastiva a partir de una función de pérdida usando tripletas.

En la \textbf{parte práctica}, se explican los principales marcos de trabajo que se utilizan y se exponen los experimentos realizados y resultados obtenidos.

La \textbf{tercera parte} presenta dos redes siamesas:\emph{SimCLR} (capítulo 7) y \emph{Bootstrap your own latent (BYOL)} (capítulo 8), los dos marcos de trabajo que han surgido en el año 2020 para el aprendizaje de representaciones. En ambos casos, se da una motivación de por qué surgen, se explica la arquitectura que ambas siguen y las funciones de pérdida que utilizan cada una, y se comenta qué hiperparámetros pueden ser más relevantes a la hora de entrenar los modelos y obtener mejores representaciones para las tareas posteriores como clasificación o regresión.

En la \textbf{cuarta y última parte} se exponen los experimentos realizados utilizando los marcos anteriores. Primeramente (capítulo 9), se exponen los objetivos que se persiguen mediante estos experimentos, que se resumen en adaptar los experimentos existentes a los recursos de los que se disponen. Además, se exponen las tecnologías utilizadas. Seguidamente (capítulo 10), se exponen primero los tres experimentos realizados utilizando SimCLR, un primero general, un segundo aumentando el tamaño y profundidad del \emph{encoder} del marco, y un tercero añadiendo una nueva capa de aumento de datos y preprocesamiento de los mismos. Los experimentos resultan exitosos, obteniendo en el tercero mejoras respecto al primero y resultados acordes a lo previsto. Lo mismo oscurre más tarde cuando realizamos dos experimentos utilizando BYOL, un primero en el que se demuestra empíricamente que la influencia que tenía el \emph{tamaño del batch} en SimCLR se pierde en BYOL, y un segundo en el que se amplía de nuevo el tamaño del encoder, obteniendo también resultados exitosos.

\textbf{Palabras clave:} \emph{información mutua}, \emph{estimación contrastiva de ruido}, \emph{entropía}, \emph{aprendizaje de representaciones}, \emph{redes siamesas}, \emph{pérdida usando tripletas}, \emph{aprendizaje profundo}, \emph{cotas inferiores}.

\newpage


\tableofcontents
\newpage
\listoffigures
\listoftables

\newpage

\ctparttext{
  \color{black}
  \begin{center}
    In this part we will introduce the topic of this thesis, and also explain the underlying concepts of probability theory and probability distributions that will be needed. Also, the \emph{Noise Contrastive Estimation} problem will be presented.
  \end{center}
}
\part{Introduction and basic notions}
\chapter{Introduction, motivation and objectives}

Machine learning is the field of computer science where the computer uses a set of algorithms to learn properties from a given data set $\mathcal X$. The first machine learning problems involved learning a function $f: \mathcal X \to \mathcal Y$ that mapped an input $x \in \mathcal X$ to a label $y \in \mathcal Y$. This discipline is commonly known as \emph{Supervised Learning}.

As years passed, diverse algorithms that manage to find truly complex $f$ functions have been developed and new problems have emerged. Supervised learning has a huge cost: all the examples in the dataset must be labeled. The coss of this resided in the fact that labeling examples is a slow process, and has to be done mostly manually. 

\emph{Unsupervised learning} avoids this problem by trying to infer properties of de data using the \emph{unlabeled} dataset. By not needing to have the labels of our examples, companies can save money and time that they would have invested creating a tag for each example.

When the dimension of the data is high, for instance when treating images computationally, it is usual to first create a \emph{representation} of the input data. Combining this idea with unsupervised learning, we reach to the field of \emph{representation learning}, which studies how to create representations of the data that are useful for performing other tasks such as classification.

Ideally, we would like the original data and the representation created to contain the same information. A way of measuring this is using the \emph{mutual information} $I(X,Z)$ between the input $X$ and the representation $Z$. The mutual information is expressed as:
\[
I(X,Z) = H(X) - H(X|Z),
\]
where $H(X)$ is the entropy of $X$ and $H(X|Z)$ is the conditional entropy.

\begin{figure}[H]
    \centering
    \includegraphics[scale=0.3]{mutual-info}
    \caption{Venn Diagram showing the relationships between the entropies and the conditional entropies of random variables $X$ and $Z$.}
\end{figure}

Since the entropy is a way of measuring ``\emph{how surprising is that an event occurs}'', intuitively the mutual information measures the decrease of uncertainty that we obtain in $X$ when we know that $Z$ has occurred.

Calculating the mutual information between two variables, however, is not an computationally easy problem. Because of this, a way to approach the mutual information maximization is by obtaining lower bounds and maximizing them. Although a few more bounds are proved in this document, we remark the \emph{Contrastive Lower Bound} on the mutual information, which is expressed as follows:
\[
I(X,Z)  \geq - \ell(\theta) + \log N.
\]
In this equation, $\ell(\theta)$ refers to the contrastive loss, which happens to be crucial in representation learning. In short, the \emph{contrastive learning} problem consists in considering elements obtained from the same distribution and elements obtained from a different distribution and learning how to discriminate between the elements of the different distributions. In order to do this, the recently mentioned contrastive loss is used, which usually takes the form
\[
\mathcal L_N = - E_X \left[ \log \frac{f(x,z)}{\sum_{x_j \in X}f(x_j,z)}\right].
\]  
The use of this contrastive loss maximizing the mutual information had a huge impact on the state of art results of representation learning in 2018, achieving very promising results.

However, during the development of this work, this drastically changed. New papers appeared stating that the success of the methods that were maximizing mutual information between the input and its representation was not caused by mutual information, but by the specific form that the contrastive loss has.
\clearpage
\chapter{Probability theory}


Underneath each experiment involving any grade of uncertainty there is a \emph{random variable}. This is no more than a \emph{measurable} function between two \emph{measurable spaces}.
A probability space is composed by three elements: $(\Omega, \Alg, \Prob)$. We will define those concepts one by one.

\section{Basic notions}

\begin{ndef}Let $\Omega$ be a non empty sample space. $\Alg$ is a $\sigma-$algebra over $\Omega$ if it is a family of subsets of $\Omega$ that verify that the emptyset is in $\Alg$, and it is closed under complementation and countable unions. That is:
\begin{itemize}
  \item $\emptyset \in \Alg$
  \item If $A \in \Alg$, then $\Omega \textbackslash A \in \Alg$
  \item If $\{A_i\}_{i \in \mathbb N} \in A$ is a numerable family of $\Alg$ subsets, then $\cup_{i \in \mathbb N} A_i \in \Alg$
\end{itemize}
\end{ndef}


The pair $(\Omega,\Alg)$ is called a \emph{measurable space} To get to our probability space, we need to define a \emph{measure} on the \emph{measurable space}.

\begin{ndef}
Given $(\Omega,\Alg)$, a measurable space, a \emph{measure} $\Prob$ is a countable additive, non-negative set function on this space. That is: $\Prob: \Alg \to \mathbb R_0^+$ satisfying:
\begin{itemize}
  \item $\Prob(A) \geq \Prob(\emptyset) = 0$ for all $A \in \Alg$
  \item $P(\cup_n A_n) = \sum_n P(A_n)$ for any countable collection of disjoint sets $A_n \in \Alg$.
\end{itemize}
\end{ndef}

If $\Prob(\Omega) = 1$, $\Prob$ is a \emph{probability measure} or simply a \emph{probability}. With the concepts that have just been explained, we get to the following definition:

\begin{ndef}
A \emph{measure space} is the tuple $(\Omega, \Alg,\Prob)$ where $\Prob$ is a \emph{measure} on $(\Omega, \Alg)$. If $\Prob$ is a \emph{probability measure} $(\Omega,\Alg,\Prob)$ will be called a \emph{probability space}.
\end{ndef}

Throughout this work, we will be always in the case where $\Prob$ is a probability measure, so we will always be talking about probability spaces. Some notation for these measures must be introduced. Let $A$ and $B$ be two events.
The notation $P(A,B)$ reffers to the probability of the intersection of the events $A$ and $B$, that is: $P(A,B) := P(A\cap B)$.
 It is clear that since $A \cap B = B \cap A$, then $P(A,B) = P(B,A)$. We remark the next definition since it will be important.

\begin{ndef}
Let $A,B$ be two events in $\Omega$. The \emph{conditional probability} of $B$ given $A$ is defined as:
$$
P(B|A) = \frac{P(A,B)}{P(A)}
$$
\end{ndef}



% Introduce here Bayes Theorem
% ------------------------------------------------------------------------------

There is an alternative way to state the definition that we have just made.

\begin{nth}[Bayes' theorem]
Let $A,B$ be two events in $\Omega$, given that $P(B) \neq 0$. Then
$$
P(B|A) = \frac{P(A|B) P(A)}{P(B)}
$$
\end{nth}
\begin{proof}
Straight from the definition of the conditional probability we obtain that:
$$
P(A,B) = P(A|B)P(B)
$$
We also see from the definition that
$$
P(B,A) = P(B|A)P(A)
$$
Hence, since $P(A,B) = P(B,A)$,
$$
P(A|B)P(B) = P(B|A)P(A) \implies P(A|B) = \frac{P(B|A)P(A)}{P(B)}
$$
\end{proof}


However, events might not give any information about another event occurring. When this happens, we call those events to be \emph{independent}. Mathematically, if $A$,$B$ are independent events:
$$
P(A,B) = P(A)P(B)
$$
and as a consequence of this, the conditional probabilty of those events is $P(A|B) = P(A)$.\\


\emph{Random variables} (R.V.) can now be introduced. Their first property is that they are measurable functions. Those kind of functions are defined as it follows:

\begin{ndef}
Let $(\Omega_1, \Alg),(\Omega_2, \mathcal B)$ be measurable spaces. A function $f: \Omega_1 \to \Omega_2$ is said to be \emph{measurable} if, $f^{-1}(B) \in \Alg$ for every $B \in \mathcal B$.
\end{ndef}

As a quick note, we can affirm that if $f,g$ are real-valued measurable functions, and $k \in \mathbb R$, it is true that $kf$, $f+g$ , $fg$ and $f/g$ (if $g$ is not the identically zero function) are also \emph{measurable functions}.

We are now ready to define one of the concepts that will lead us to the main objective of this thesis.

\begin{ndef}[Random variable]
Let $(\Omega,\Alg,\Prob)$ be a probability space, and $(E,\mathcal B)$ be a measurable space. 
A \emph{random variable} is a measurable function $X: \Omega \to E$, from the probability space to the measurable space. This means: for every subset $B \in (E,\mathcal B)$, its preimage
$$
X^{-1}(B) = \{\omega : X(\omega) \in B\} \in \Alg .
$$
\end{ndef}

Using that sums, products and quotients of measurable functions are measurable functions, we obtain that \emph{sums, products and quotients of random variables are random variables}.

Let now $X$ be a R.V. The \emph{probability} of $X$ taking a concrete value on a measurable set contained in $E$, say, $S \in E$, is written as:
$$
P_X(S) = P(X \in S) = P(\{a \in \Omega : X(a) \in S\})
$$

A very simple example of random variable is the following:

\begin{nexample}
  Consider tossing a coin. The possible outcomes of this experiment are \emph{Heads or Tails}. Those are our random events. We can give our random events a possible value. For instance, let \emph{Heads} be $1$ and \emph{Tails} be 0. Then, our random variable looks like this:
  \begin{equation*}
      X  = \left\{ \begin{aligned}
  1 & \text{if we obtain heads} \\
  0 & \text{if we obtain tails}
\end{aligned}\right.
  \end{equation*}

\end{nexample}

In the last example, our random variable is \emph{discrete}, since the set $\{X(\omega): \omega \in \Omega\}$ is finite.
 A \emph{Random Variable} can also be \emph{continuous}, if it can take any value within an interval.\\


\section{Expectation of a random variable}

\begin{ndef}
The \emph{cumulative distribution function } $F_X$ of a real-valued random variable $X$ is its probability of taking value below or equal to $x$. That is:
$$
F_X(x) = P(X \leq x) = P(\{\omega : X(\omega) \leq x\}) = P_X((-\infty,x]) \quad \forall x \in \mathbb R
$$
\end{ndef}

Depending on the image of a random variable $X$, we can difference between certain types of random variables. If the image $\mathcal X$ of $X$ is countable, we call it a \emph{discrete} random variable. Its \emph{probability mass function} gives the
probability of the r.v. being equal to a certain value:
$$
p(x) = P(X = x).
$$
If the image $\mathcal X$ of $X$ is uncountable and real, then $X$ is a \emph{continuous} random variable. In this case there might exist a non-negative Lebesgue-integrable function $f$ such that:
$$
F_X(x) = \int_{\infty}^x f(t) dt,
$$
called the \emph{probability density function} of $X$.\\


Usually, when it comes to applying these concepts to a real problem, we will be looking at multiple variables. We would like to have a collection of random variables each one representing one of this variables.
In order to set the notation for these kinds of situations, we will introduce \emph{random vectors}.

\begin{ndef}
  A random vector is a row vector $\rvc$ whose components are rea-valued random variables on the same probabilty space $(\Omega,\Alg,P)$.
\end{ndef}

The probability distribution of a random variable can be extended in to the \emph{joint probability distribution} of a random vector.

\begin{ndef}
Let $\rvc$ be a random vector. The \emph{cumulative distribution funcion} $F_{\rv} : \R^n \to [0,1]$ of $\rv$ is defined as:
$$
F_{\rv}(x) = P(X_1 \leq x_1 , \dots, X_n \leq x_n)
$$
\end{ndef}

The distribution of each of the component random variables $X_i$ of $\rv$ are called \emph{marginal distributions}.

We would also like to know what are the most probably values that we can obtain out of a random variable.  This is called the \emph{expectation} of a random variable.

\begin{ndef}[Expectation of a \emph{R.V.}]
Let $X$ be a non negative random variable on a probability space $(\Omega,\Alg,\Prob)$. The expectation $E[X]$ of $X$ is defined as:
$$
E[X] = \int_\Omega X(\omega) \ dP(\omega)
$$
\end{ndef}
If $X$ is generic \emph{R.V}, the expectation is defined as:
$$
E[X] = E[X^+] - E[X^-]
$$
where $X^+,X^-$ are defined as it follows:
$$
X^+(\omega) = \max(X(\omega),0) \quad \quad  \quad \quad X^-(\omega) = \min(X(\omega),0)
$$

The \emph{expectation} $E[X]$ of a \emph{random variable} is a linear operation. That is, if $\mathcal Y$ is another random variable, and $\alpha,\beta \in \R$, then
$$
E[\alpha X + \beta \mathcal Y] = \alpha E[X] + \beta E[\mathcal Y]
$$
this is a trivial consequence of the linearity of the \emph{Lebesgue integral}.

As a note, if $X$ is a \emph{discrete} random variable and $\X$ is its image, its expectation can be computed as:
$$
E[X] = \sum_{x \in \X} x  P_X(x)
$$
where $x$ is each possible outcome of the experiment, and $P_X(x)$ the probability under the distribution of $X$ of the outcome $x$. The expression given in the definition before generalizes this particular case.

Using the definition of the \emph{expectation} of a random variable, we can approach to the \emph{moments} of a random variable.

\begin{ndef}
If $k \in \N$, then $E[X^k]$ is called the $k-th$ moment of $X$.
\end{ndef}
If we take $k = 1$, we have the definition of the \emph{expectation}. It is sometimes written as $m_X = E[X]$, and called the \emph{mean}. We use the \emph{mean} in the definition of the variance:

\begin{ndef}
Let $X$ be a random variable. If $E[X^2] < \infty$, then the \emph{variance} of $X$ is defined to be
$$
\Var(X) = E[(X - m_X)^2] = E[X^2] - m_X^2 
$$
\end{ndef}

Thanks to the linearity of the \emph{expectation} of a random variable, it is easy to see that
$$
Var(aX + b) = E[(aX + b) - E[aX + b])^2] = a^2E[(X - m_X)^2] = a^2 \Var(X)
$$


\clearpage
\chapter{Distributions and Kullback-Leibler Divergence}
We have introduced the concepts of \emph{random variable},  \emph{random vector} and its \emph{probability distribution}.  Now, given two distributions, in the following chapters we will like to see how different they are from each other.
In order to compare them, we enunciate the definition of the Kullback-Leibler divergence.

\begin{ndef}[Kullback-Leibler Divergence]
Let P and Q be probability distributions over the same probability space $\Omega$. Then, the Kullback-Leibler divergence is defined as:
$$
D_{KL}(P \ || \ Q) = E_P\left[\log{\frac{P(x)}{Q(x)}}\right]
$$
\end{ndef}
It is defined if and only if $P$ is \emph{absolutely continuous with respect to} $Q$, that is , if $P(A) = 0$ for any $A$ subset of $\Omega$ where $Q(A) = 0$. There are some properties of this definition that must be stated. The first one is the following proposition:

\begin{nprop}
If P,Q are two probability distributions over the same probability space, then $D_{KL}(P|Q) \geq 0$.
\end{nprop}
\begin{proof}
Firstly, note that if $a \in \R^+$, then $\log \ a \leq a-1$. Then:
\begin{align*}
-D_{KL}(P \ || \ Q) & = - E_P\left[\log{\frac{P(x)}{Q(x)}}\right] \\
             & = E_P\left[\log{\frac{Q(x)}{P(x)}}\right] \\
             & \leq E_P\left[\left(\frac{Q(x)}{P(x)} - 1\right)\right]\\
             & = \int P(x) \frac{Q(x)}{P(x)} dx -1 \\
             & = 0
\end{align*}
So we have obtained that $-D_{KL}(P\ ||\ Q) \leq 0$, which implies that $D_{KL}(P\ || \ Q) \geq 0$.
\end{proof}
As a corollary of this proposition, we can affirm that $D_{KL}(P\ ||\ Q)$ equals zero if and only if $P = Q$ almost everywhere. We will also remark the discrete case, as it will be used later. Let $P,Q$ be discrete probability distributions
defined on the same probability space $\Omega$. Then, 
$$
D_{KL}(P\ ||\ Q) = \sum_{x \in \Omega} P(x) \log \left( \frac{P(x)}{Q(x)}\right)
$$
\clearpage
\chapter{Statistical Inference}
Statistical inference is the process of deducing properties of an underlying distribution by analyzing the data that it is available. With this purpose, techniques like deriving estimates and testing hypotheses are used. 

Inferential statistics are usually contrasted with descriptive statistics, which are only concerned with properties of the observed data. The difference between these two is that in inferential statistics, we assume that the data comes from a larger
population that we would like to know.

In \emph{machine learning}, subject that concerns us the most, the term inference is sometimes used to mean \emph{make a prediction by evaluating an already trained model}, and in this context, inferring properties of the model is refered as \emph{training or learning}.

\section{Parametric Modeling}

In the following chapters, we will be trying to estimate density functions in a dataset. To do this we will be using \emph{parametric models}. We say that a \emph{parametric model}, $P_\theta(x)$, 
is a family of density functions that can be described using a finite numbers of parameters $\theta$. We can get to the concept of \emph{log-likelihood} now.

\begin{ndef}
The \emph{likelihood} $\mathcal L(\theta | x)$ of a parameter set $\theta$ is a function that measures how plausible is $\theta$, given an observed point $x$ in the dataset $\D$. It is defined as the value of the 
density function parametrized by $\theta$ at $x$. That is:
$$
\mathcal L(\theta|x) = P_\theta(x).
$$
\end{ndef}

In a finite dataset $\D$ consisting of independent observations, we can write:
\[
\mathcal L(\theta | X) = \prod_{x \in D} P_\theta(x).
\]

This can be computationally hard to work with, so the log-likelihood is often used instead.

\begin{ndef}
Let $\D$ be a dataset of independent observations and $\theta$ a set of parameters. Then, we define the \emph{log-likelihood} $\ell$ as the sum of the logarithms of the evaluations of $p_\theta$ in each $x$ in the dataset. That is:
\[
\ell (\theta | X) = \sum_{x \in \D} \log P_\theta(x).
\]
\end{ndef}

Our goal would be to find the optimal value $\hat{\theta}$ that maximizes the likelihood of observing the dataset $\D$. We get to the following definition:

\begin{ndef}
    We say that $\hat{\theta} = \hat\theta (\D)$ is a \emph{maximum likelihood estimator}(MLE) for $\theta$ if  
    $$
    \hat\theta \in \argmax_{\theta} \mathcal L(\theta | \D)
    $$
    for every observation $\D$. 
\end{ndef}

\section{Minimal sufficient statistics}

In parametric modeling, the goal was to determine the density function under a distribution. Another interesting task can be determining specific parameters or quantities related to a distribution, given a sample $X = (x_1,\cdots,x_n)$.

\begin{ndef}
    Let $(\Omega,\Alg)$ be a measurable space where $\Alg$ contains all singletons. A statistic is a measurable function of the data, that is: $T: X \to \Omega$ where $T$ is measurable.
\end{ndef}
\begin{remark}
    A statistic is also a random variable.
\end{remark}

However, not all statistics will provide useful information for the statistical inference problem, since almost anything can be a statistic. We would like to find statistics that provide relevant information.

\begin{ndef}
    Let $X \sim P_\theta$. Then, the statistic $T(X) = T : (\Omega, \Alg) \to (\mathbb T, \mathcal B)$, is sufficient for a family of parameters $\{P_\theta \ : \ \theta \in \Theta \}$ if the conditional distribution of $X$, given $T = t$, is indepentent of $\theta$.\\
\end{ndef}

\begin{nexample}
The simplest example of a sufficient statistic is the mean $\mu$ of a gaussian distribution with known variance. Oppositely, the \emph{median} of an arbitrary distribution
is not sufficient for the mean since, even if the median of the sample is known, more information about the mean of the population can be obtained from the mean of the sample itself.
\end{nexample}

Although it will not be shown in this document, sufficient statistics are not unique. In fact, if $T$ is sufficient, $\psi(T)$ is sufficient for any bijective mapping $\psi$. It would be interesting to find a sufficient statistic $T$ that is \emph{the smallest} of them.

\begin{ndef}
    A sufficient statistic $T$ is minimal if, for every sufficient statistic $U$, there exists a mapping $f$ such that $T(x) = f(U(x))$ for any $x \in \Omega$.
\end{ndef}
\clearpage
\chapter{Noise Contrastive Estimation}
\label{Chapter:NCE}
Our problem now is, to estimate the density function (p.d.f.) of some observed data.

A sample $X = \{x_1,\dots,x_{T_d}\}$ of a random vector is observed. It follows an unknown pdf $P_d$. We assume that the data p.d.f. belongs to a parametrized family of functions, that is
\[
P_d \in \{P_m(.;\theta)\}_\theta,
\]
where $\theta$ is a vector of parameters. This means that, in fact,
$$
P_d(.) = P_m(.;\theta^*) \quad \text{for some } \theta^*,
$$
so our problem is to find $\theta^*$. 

Any estimate $\hat{\theta}$ must meet the constraints that a normalized p.d.f. should satisfy, that is:
$$
\int P_m(u;\hat{\theta})du = 1, \quad \quad P_m(.;\hat{\theta})\geq 0.
$$
If the constraints are satisfied for any $\theta$ in the set of parameters, we say that the model is normalized, and then we can use the maximum likelihood principle to estimate $\theta$.

Let us assume that the noisy data $Y$ is an i.i.d. sample $\{y_1,\dots,Y_{T_n}\}$ of a random variable with p.d.f. $P_n$. The ratio $P_d/P_n$ of the density functions that generate $X$ and $Y$ respectively, can give us a relative description of the data $X$. If $P_n$ is known, then we can obtain $P_d$ using the ratio that we have just mentioned.

In order to discriminate between elements of $X$ and $Y$, it is needed to compare their properties. We will show that we can provide a relative description of $X$ in the form of an estimate of the ratio $P_d/P_n$.

Let $U = \{u_1,\cdots,u_{T_d + T_n}\}$ be the union of the sets $X$ and $Y$. We assign to each $u_t$ a binary class label:
\[
C_t(u_t) = \begin{cases}
1 & if \ u_t \in X\\
0 & if \ u_t \in Y
\end{cases}
\]
We will now make use of logistic regression, where the posterior probabilities of the classes given the data are estimated. We know that $P_d$ is unknown, we want to model $P(.|C=1)$ with $P_m(.;\theta)$. Note that $\theta$ may include a parameter for the normalization of the model, if it is not normalized. Hence, we have:
\[
P(u|C = 1,\theta) = P_m(u;\theta), \quad \quad P(u|C = 0) = P_n(u),
\]
with
\[
P(C = 1) = \frac{T_d}{T_d + T_n}, \quad \quad P(C = 0) = \frac{T_n}{T_d + T_n}.
\]
Hence, if $\nu = T_n/T_d$, the posterior probabilities for the classes are:
\[
P(C=1|u;\theta) = \frac{P_m(u;\theta)}{P_m(u;\theta) + \nu P_n(u)}, \quad \quad P(C = 0|u; \theta) = \frac{\nu P_n(u)}{P_m(u;\theta) + \nu P_n(u)}.
\]
Denote $G(.;\theta)$ to the log ratio between $P_m(.;\theta)$ and $P_n$:
\begin{equation}\label{log:ratio:G}
G(u;\theta) = \log P_m(u;\theta) - \log P_n(u) = \log \frac{P_m(u;\theta)}{P_n(u)}.
\end{equation}
Also, let $r_\nu$ the logistic function parametrized by $\nu$, that is:
\begin{equation}\label{log:func:nu}
r_\nu(u) = \frac{1}{1 + \nu exp(-u)}.
\end{equation}
Using \ref{log:ratio:G} and \ref{log:func:nu}, we can write
\[
h(u;\theta) := P(C = 1|u ; \theta) =    r_\nu(G(u;\theta)) = \frac{1}{1+ \nu exp(\log \frac{P_m(u,\theta)}{P_n(u)})}. 
\]
Since the class labels $C_t$ are assumed Bernoulli distributed and independent, the conditional log-likelyhood has the form:
\begin{equation}\label{log:likelihood:theta}
\ell(\theta)  = \sum_{t = 1}^{T_d + T_n} C_t \log P(C_t = 1|u_t; \theta) + (1-C_t) \log P(C_t = 0|u_t;\theta).
\end{equation}
Now, in the $t$ such that $u_t$ in $X$,then  $u_t = x_t$ an we have that $P(C_t = 0|x_t;\theta) = 0$, so we obtain that the term that adds to the sum in that certain $t$ is:
\[
1\cdot \log P(C_t = 1|u_t;\theta) = \log h(x_t;\theta).
\]
Using the same argument for $t$ such that $u_t \in Y$, we obtain the following form of the log-likelyhood in \ref{log:likelyhood:theta}:
\begin{equation}\label{log:likelihood:red}
\ell(\theta) = \sum_{t = 1}^{T_d} \log [h(x_t;\theta)] + \sum_{t = 1}^{T_n} \log[1- h(y_t,\theta)].
\end{equation}
Now, optimizing $\ell(\theta)$ with respecto to $\theta$ leads to an estimate $G(.;\hat{\theta})$ of the log-ratio $\log (P_d/P_n)$, so we get an approximate description of $X$ relative to $Y$ by optimizing \ref{log:likelyhood:red}.

If we consider $-\ell(\theta)$, this is known as the \emph{cross entropy function}.

\begin{remark}
    Here, we have achieved the estimation of a p.d.f. , which is an unsupervised (not labeled data) learning problem, logistic regression, which is supervised learning (labeled data).
\end{remark}

Now, if we consider $P_m^0(.;\alpha)$ an unnormalized (doest not integrate $1$) model, we can add a normalization parameter to it in order to normalize it. We can consider
\[
\log P_m(.;\theta) = \log P_m^0(.;\alpha ) + c  , \quad \quad \text{with } \theta=(\alpha,c).
\]
With this model, a new estimator is defined. Considering $X$ as before and $Y$ an artificially generated set with $T_n = \nu T_d$ independent obvservations extracted from $P_n$, known. as the argument $\hat{\theta}_T$ which maximizes
\[
J_T(\theta) = \frac{1}{T_d}\{\sum_{t = 1}^{T_d} \log[h(x_t;\theta)] + \sum_{t=1}^{T_n}\log[1-h(y_t;\theta)]\}.
\]

We have to remark that in this case, we have fixed $\nu$ before $T_n$, so $T_n$ will increase as $T_d$ increases. Now, using the weak law of large numbers, $J_T(\theta) \to J$ in probability, where
\[
J(\theta) = E\{\log[h(x;\theta)]\} + \nu E{\log[1-h(y;\theta)]}.
\]
Let us rename some terms before announcing a theorem. We want to see $J$ as a function of $\log P_m(.;\theta)$ instead of only $\theta$. In order to do this, let $f_m(.) = \log P_m(.;\theta)$, and consider
\[
\tilde{J}(f_m) = E\{\log[r_\nu (f_m(x) - \log P_n(x))]\} + \nu E\{\log [1- r_\nu(f_m(y) - \log P_n(y))]\}.    
\]
The following theorem states that the probability density function $P_d$ of the data can be found by maximizing $\tilde{J}$, that is, learning a nonparametric classifier in \emph{infinite data}.

\begin{nth}
The objective $\tilde{J}(f_m)$ achieves a maximum at $f_m = \log P_d$. Furthermore, there are not other extrema if the noise density $P_n$ is chosen such that it is nonzero whenever $P_d$ is nonzero.
\end{nth}
\clearpage

\ctparttext{
  \color{black}
  \begin{center}
  Information theory is the base for all the following work. In this part, \emph{Entropy} and \emph{Mutual Information} will be explained and, then, lower bounds for mutual information function will be given.
  \end{center}
}

\part{Information Theory}
\chapter{Entropy}
\input{Chapters/2-Information-Theory/1-Entropy}
\clearpage
\chapter{Mutual Information}


Using the entropy of a random variable we can directly state the definition of \emph{mutual information} as follows:

\begin{ndef}
Let $X,Z$ be random variables. The \emph{mutual information (MI)} between $X$ and $Z$ is expressed as the difference between the entropy of $X$ and the conditional entropy of $X$ and $Z$, that is:
$$
I(X,Z) := H(X) - H(X|Z).
$$
\end{ndef}

Since the entropy of the random variable $H(X)$ explains the uncertainty of $X$ occurring, the intuitive idea of the \emph{MI} is to determine the decrease of uncertainty of $X$ occurring when we already
know that $Z$ has occurred. We also have to note that, using the definition of the \emph{entropy} and the expression obtained in Eq. \ref{eq:dif-expr-mi}, we can rewrite the \emph{MI}  it follows:
\begin{align*}
I(X,Z) & = \sum_{x \in \X}P_X(x) \log \frac{1}{P(x)} - \sum_{x \in \X, z \in \mathcal Z} P_{XZ}(x,z) \log \frac{P_Z(x)}{P_{XZ}(x,z)} \\  & = \sum_{x,z}P_{XZ}\log \frac{P_Z(z)P_X(x)}{P_{XZ}(x,z)} = D_{KL}(P_{XZ} \ || \ P_X P_Z)
\end{align*}
and we have obtained an expression of the mutual information using the \emph{Kullback-Leibler} divergence. This provides with the following immediate consequences:
\begin{enumerate}[label=$(\roman*)$]
\item Mutual information is non-negative. That is : $I(X,Z) \geq 0$.
\item If $X,Z$ are random variables, then its mutual information equals zero if, and only if, they are independent. 
This easy to check since if $D_{KL}(P_{XZ} \ || \ P_X P_Z) = 0$, then $P_{XZ} = P_X P_Z$ almost everywhere so $X$ and $Z$ are independent.p
\item Since $P_{XZ} = P_{ZX}$ and $P_X P_Z = P_Z P_X$, mutual information is symmetric. That is: $I(X,Z) = I(Z,X)$.
\end{enumerate}

Later in this document, we will have some sort of random variable $X$ and would like it to maintain the mutual information with itself after being applied a function. The following proposition will be useful:

\begin{nprop}
Let $X,Z$ be random variables. Then, $I(X,Z)$ is invariant under homeomorphism.
\end{nprop}
\begin{proof}
Let $\phi(x)$ be an homeomorphism, i.e., a continuous, monotonic function with $\phi^{-1}(x)$ also continuous and monotonic. Let $X$ be a random variable and $Y$ another one such $y = \phi(x)$ if $x = X(\omega)$ for some $\omega \in \Omega$. Then, if $S$ is a particular subset we have 
\[
P(Y \in S) = \int_S P_Y(y) dy = \int_{\phi^{-1}(S)}P_X(x) dx \stackrel{(1)}{=} \int_S P_X(\phi^{-1}(y)) \abs{ \frac{d \phi^{-1}}{dy}}dy,
\]
where in $(1)$ we have changed from $x$ to $y$. Hence, 
\[
P_Y(y) = P_X(\phi^{-1}(y))\abs{\frac{d \phi^{-1}}{dy}}.
\]
As a consequence of this, $I(X,Z) = I(\phi(X),Z) $ for any homeomorphism $\phi$. By symmetry, the same holds for $Z$.

\end{proof}

\begin{remark} We can set a connection between the mutual information and sufficient statistics. Let $T(X)$ be a statistic. We say that $T(X)$  is sufficient for $\theta$ if its mutual information with $\theta$ equals the mutual information between $X$ and $\theta$, that is:
$$
I(\theta, X) = I (\theta, T(X)).
$$
This means that sufficient statistics preserve mutual information and conversely.
\end{remark}

\section{Lower bounds on Mutual Information}

Although mutual information seems like a relatively intuitive concept, it is most of the times extremely hard to compute it in real life problems in which the distributions $P(x,z),P(x),P(z)$ are not known.

\begin{nexample}
Let $x$ represent an image of size $n \times m$ pixels. Then, the dimension of the single image is $n \cdot m \cdot 3$, for RGB color channels. In these cases, there is no easy way of calculating $P(x)$.
\end{nexample}

Due to this problem related to the \emph{Curse of Dimensionality}, we can try to compute lower bounds of it that are generally easier to calculate. We will now expose two general lower bounds, and we will focus on a third one that will be explained later in this work.

\subsection*{Variational Lower Bound}



Using the expression of the mutual information in terms of entropy, $I(x,z) = H(z) - H(z|x)$, we can give a lower bound on $I(x,z)$ as a function of a probability distribution $Q_\theta(z|x)$. 

\begin{nprop}
Let $X,Z$ be random variables and $Q_\theta(z|x)$ be an arbitrary probability distribution. Then,
$$
I(x,z) \geq H(z) + E_{P_X} \left[ E_{P_{X|Z}}\left[\log Q_\theta(z|x)\right]\right] 
$$
\end{nprop}

\begin{proof}
Recalling that
$$
H(z|x) = - E_{P_{XZ}} \left[ \log P(x,z) - \log P(x)\right],
$$
and that
\begin{align*}
E_{P(x,z)}\left[\log\frac{P(x,z)}{P(x)}\right] & =  \sum_{x,z} P(x,z) \log\frac{P(x,z)}{P(x)} \\ 
& = \sum_{x,z} P(x)P(z|x) \log P(z|x) = \sum_{x,z} P(x) E_{P(z|x)}[\log P(z|x)]\\
 & =  E_{P(x)}\left[E_{P(z|x)}[\log P(z|x)]\right],
\end{align*}
we only have to use the definition of the conditional probability to see that:
\begin{align*}
I(x,z) & =  H(z) - H(z|x) \\
    & =  H(z) + E_{P(x,z)} = H(z) + E_{P(x,z)} \left[ \log \frac{P(x,z)}{P(x)}\right] \\
    & =  H(z) + E_{P(x)} \left[ E_{P(x|z)}\left[\log P(z|x)\right]\right] \\
    & = H(z) + E_{P(x)} \left[ E_{P(x|z)} \left[\log \frac{P(z|x)}{Q_\theta(z|x)}\right] + E_{P(z|x)}\left[\log Q_\theta(z|x)\right]\right] \\
    & =  H(z) + E_{P(x)}\left[ \underbrace{D_{KL}(P(z|x)||Q_\theta(z|x))}_{\geq 0} + E_{P(z|x)}\left[\log Q_\theta(z|x)\right] \right]\\
    & \geq H(z) + E_{P(x)}\left[E_{P(z|x)}\left[ \log Q_\theta(z|x)\right]\right].
\end{align*}
We have taken advantage of the non-negativity of the KL-Divergence.
\end{proof}

Using this bound, and combining this theoretical knowledge with machine learning methods, such as \emph{backpropagation}, we can make $Q_\theta$ be a neural network and maximize this lower bound.

\subsection*{Donsker-Varadhan Representation}

We can also give a lower bound on the mutual information using its KL-Divergence formulation. Firstly, we have to 

\begin{nth}[Donsker-Varadhan]
The KL divergence admits the following dual representation:
\[
D_{KL}(P || Q) = \sup_{T} E_P[T] - \log E_Q[e^T],
\]
where the supremum is taken over all functions $T:\Omega \to \R$ such that both expectations exist.
\end{nth}
\begin{proof}
    TODO
\end{proof}

Using this representation, we reach this lower bound. Let $\mathcal F$ be any class of functions $T: \Omega \to \R$ satisfying the integrability constraints of the theorem. Then, 
$$
I(P,Q) = D_{KL}(P||Q) \geq \sup_{T \in \mathcal F} E_P[T] - \log E_Q[e^T].
$$


\ctparttext{
  \color{black}
  \begin{center}
  In this part, it will be explained what it is understood for \emph{Representation Learning}, explaining the first framework that was used in this field. Also, the \emph{Noise Contrastive Loss} is presented, which will be crutial in the rest of this work.
  \end{center}
}
\part{Representation Learning}
\chapter{Context}



Before continuing presenting the mathematical notions of the topics that are treated in this work, it is interesting to present what we are pursuing with this work.\\

\emph{Machine Learning} is the part of computer science that studies \emph{algorithms} that improve automatically through experience from examples. These algorithms help computer to discover how to
perform tasks without being explicitly programmed to do them. For the computers to learn, it is mandatory that a finite set of data (or dataset) $\mathcal D$ is available. \\

Depending on how the data (\emph{or signal}) is given to the computer, the machine learning approaches can be divided in three broad categories:
\begin{enumerate}
    \item \emph{Supervised learning}. In this category each point $x_i \in \mathcal D$ in the dataset is \emph{labeled}: each example is related to a tag $y_i \in Y$ that gives information about $x$. The goal in this case is to find 
    a function $g:D \to Y$.
    \item \emph{Unsupervised learning}. In ths case, the data is \emph{unlabeled}, so the approach is completely different. Usually, the goal here is to discover hidden patterns in data or to learn features from it.
    \item \emph{Reinforcement learning}. This is the area concerned with how intelligent agents take decisions in an an specific environment in order to obtain the best reward in their objective.
\end{enumerate}

In this work, we will focus on unsupervised learning. Particularly, in representation learning. In the learning process, machine learning models can not directly give labels to input examples. Before, they must create a \emph{representation} that 
contains the data's key qualities.  Here is where \emph{representation learning} is born. 

% Insert the definition of representation

Representation learning is a set of techniques that allows a system to discover the representations needed for feature detection or classification. 
In contrast to manual feature engineering, feature learning allows a machine to learn the features and to use them to perform a task.\\

Feature learning can be supervised or unsupervised. In supervised feature learning, representations are learned using labeled data.
Examples of this kind of feature learning are supervised neural networks and multilayer perceptron. In unsupervised learning, the features are learned using unlabeled data. 
There are many examples of this, such as independent component analysis (ICP) and autoencoders. In this work, we will be working with unsupervised feature learning.\\

The performance of machine learning methods is heavily dependent on the choice of data features \cite{bengio_representation_2014}. This is why most of the current 
effort in machine learning focuses on designing preprocessing and data transformation that lead to good quality representations. A representation will be of good quality when its features
produce good results at running the models.\\

The main goal in representation learning is to obtain features of the data that are generally good for any supervised task. That is, we would like to obtain
a representation that is either good for image classification (giving an image a label of what we can see in it) or image captioning (producing a text that describes the image).\\

Data's features that are invariant through time are very useful for machine learning models. In \cite{wiskott_slow_2002}, \emph{slow features} are presented. Slow features are defined as features of a signal 
(which can be the input of a model) that vary slowly during time. These kind of features are the most interesting ones when creating representations, since they give an abstract view of the original data.\\

Let us give an example: In computer vision, the value of the pixels in an image can vary fastly. For instance, if we have a zebra on a video and the zebra is moving from one side of the image to the other, due 
to the black stripes of this animal, the pixels will fastly change from black to white and viceversa, so value of pixels is probably not a good feature to choose as an slow feature. However, there will always
be a zebra on the image, so the feature that indicates that there is a zebra on the image will stay positive throughout all the video, so we can say that this is a slow feature.\\

In the following chapters, we will explain 



\clearpage
\chapter{Generative Models}

\label{Chapter:Gen:Models}
The vast majority of the problems in ML are usually of a discriminative nature, which is almost a synonym of supervised learning. However, there also exist problems that involve learning how to generate new examples of the data. More formally:

\begin{ndefC}
\begin{enumerate}
\item \emph{Discriminative models} estimate $p(y|x)$, the probability of a label $y$ given an observation $x$.
\item \emph{Generative models} estimate $p(x)$, the probability of observing the datapoint $x$. If the dataset is labeled, a generative model can also estimate the distribution $p(x|y)$.
\end{enumerate}

\end{ndefC}


From now on, let $\D$ be any kind of observed data. This will always be a finite subset of samples taken from a probability distribution $\pd$. There are models that, given $\D$, try to approximate the 
probability distribution that lies underneath it. These are called \emph{generative models (G.M.)}. 

Generative models can give parametric and non parametric approximations to the distribution $\pd$. 
In our case, we will focus on parametric approximations where the model searches for the parameters that minimize a chosen metric (which can be a distance or other kind of metric such as K-L divergence) between the model distribution and the data distribution. 

We can express our problem more formally as follows. Let $\theta$ be a generative model within a model family $\mathcal M$. The goal of generative models is to optimize:
$$
\min_{\theta \in \mathcal M} d(\pd,p_\theta),
$$
where $d$ stands for the distance between the distributions. We can use, for instance, K-L divergence.

Generative models have many useful applications. We can however remark the tasks that we would like our generative model to be able to do. Those are:
\begin{itemize}
\item Estimate the density function: given a datapoint, $x \in D$, estimate the probability of that point $p_\theta(x)$.
\item Generate new samples from the model distribution $x \sim p_\theta(x)$.
\item Learn useful features of the datapoints.
\end{itemize}

If we have a look again at the example of the zebras, if we make our generative model learn about images of zebras, we will expect our $p_\theta(x)$ to be high for zebra's images. We will also expect the model
to generate new images of this animal and to learn different features of the animal, such as their big size in comparison with cats.

\section{Autoregressive Models}

In time-series theory, autoregressive models use observations from previous time steps to predict values at the current time. 
Fixing an order of the variables $x_1,\dots,x_n$, the distribution for the $i$-th random variable depends on all the preceding values in the particular chosen order. We will make use of the name of these models to 
define the machine learning approach.

A very first definition of \emph{autoregressive models (AR)} would be the following one: \emph{autoregressive models are feed-forward models that predict future values using past values}. Let us go deeper into this 
concept and explain how it behaves.

Again, let $\D$ be a set of $n-$dimensional datapoints $x$. We can assume that $x \in \{0,1\}^n$ for simplicity, without losing generality. If we choose any $x\in \D$, using the chain rule of probability, we obtain
\[
p(x) = \prod_{i=1} ^n p(x_i | x_1,\dots,x_{i-1}) = \prod_{i = 1}^n p(x_i|\bm{x}_{<i}),
\]
where $\bm{x}_{<i} \in \R^{i-1}$ is a vector whose components are the previous $x_j$ for $j = 1,\dots, i-1$, that is: $\bm{x}_{<i}= [x_1,\dots, x_{i-1}]$. \\
It is known that given a set of discrete and mutually dependent random variables, they can be displayed in a table of conditional probabilities. If $K_i$ is the number of states that each random variable can take
then $\prod K_i$ is the number of cells that the table will have. If we represent $p(x_i|\bm{x}_{<i})$ for every $i$ in tabular form, we can represent
any possible distribution over $n$ random variables. 

This, however, will cause an exponential growth on the complexity of the representation, due to the need of specifying $2^{n-1}$ possibilities 
for each case. In terms of neural networks, since each column must sum $1$ because we are working with probabilities, we have $2^{n-1}-1$ parameters for this conditional, and the tabular representation
becomes impractical for our network to learn when $n$ increases.

In autoregressive generative models, the conditionals are specified as we have mentioned before: parameterized functions with a fixed numbers of parameters. More precisely,  we assume 
the conditional distributions to be Bernoulli random variables and learn a function $f_i$ that maps these random variables to the mean of the distribution. Mathematically, we have to find 
$$
p_{\theta_i}(x_i | \bm{x}_{<i}) = \operatorname{Bern}(f_i(x_1,\dots,x_{i-1})),
$$
where $\theta_i$ is the set of parameters that specify the mean function $f_i:\{0,1\}^{i-1} \to [0,1]$.

Then, the number of parameters is reduced to $\sum_{i=1}^n \abs{\theta_i}$ so we can not represent all possible distributions as we could when using the tabular form of the conditional probabilities.
We are now setting the limit of its expressiveness because we are setting the conditional distributions $p_{\theta_i}(x_i|\bm{x}_{<i})$ to be \emph{Bernoulli} random variables with the mean specified by a restricted class 
of parametrized functions. 

Let us see a very simple case first in order to understand it better. Let $\sigma$ be a \emph{sigmoid}\footnotemark non linear function and 
$\theta_i = \left\{\alpha_{0}^{(i)},\alpha_{1}^{(i)},\dots, \alpha_{i-1}^{(i)}\right\}$ the parameters of the mean function. Then, we can define our function $f_i$ as :
$$
f_i(x_1,\dots, x_{i-1}) = \sigma(\alpha_{0}^{(i)} + \alpha_{1}^{(i)}x_1 + \dots + \alpha_{i-1}^{(i)}x_{i-1}).
$$
%------------- Footnotemark
\footnotetext{A sigmoid function is a bounded,differentiable, real function which derivative is non-negative at each point and it has exactly one inflection point.}
%----------------------
In this case, the number of parameters would be $\sum_{i = 1}^n i = \frac{n(n+1)}{2}$, so using \emph{Big }$O$ notation, we would be in the case of $O(n^2)$. We will state now a more general and useful case,
giving a more interesting parametrization for the mean function: \emph{multi layer perceptrons}\footnotemark (MLP).

%------------- Footnotemark
\footnotetext{Multi layer perceptrons are feed-forward neural networks with at least 3 layers: input, hidden and output layers; each one using an activation
function.}
%----------------------

For this example we will consider the most simple MLP: the one with one hidden layer. Let $h_i = \sigma(\bm{A}_i \bm{x}_{<i} + c_i)$ be the hidden layer activation function. Remember that $h_i \in \R^d$. Let
$ \theta_i = \{ \bm{A}_i \in \R^{d \times (i-1)}, \ c_i \in \R^d, \ \alpha^{(i)} \in \R^d, \ b_i \in \R\}$ the set of parameters
for the mean function $f_i$, that we define as:
$$
f_i(\bm{x}_{<i}) = \sigma(\alpha^{(i)}h_i + b_i).
$$
In this case, the number of parameters will be $O(n^2 d)$.

% Link NADE http://proceedings.mlr.press/v15/larochelle11a/larochelle11a.pdf

% LINK RNADE https://arxiv.org/pdf/1306.0186.pdf
This is the simplest example. Currently, there are alternative parametrization models , such as the \emph{Neural Autoregressive Density Estimator} \citep{larochelle_neural_nodate}, that provide a more statistically and computationally efficient solution. In fact, the number of parameters is reduced from $O(n^2 d)$ to $O(nd)$. Also, \emph{RNADE} \citep{uria_rnade_2014} extends NADE to learn generative models over real-valued data, generalizing the case that we have just exposed. However, these models are out of the scope of this project so no further explanation will be given.


\clearpage
\chapter{The InfoNCE Loss}
We are now ready to connnect the concepts of mutual information and generative models that we have presented. In unsupervised learning,
it is a common strategy to predict future information and to try to find out if our predictions are correct.
In \emph{natural language processing},for instance, representations are learned 
using neighbouring words (https://arxiv.org/pdf/1301.3781.pdf), and in images, some studies have been able to predict color from grey-scale (https://arxiv.org/pdf/1505.05192.pdf).

When we talk about high-dimensional data, it is not useful to make use of an unimodal loss function to evaluate our model. If we did it like this, we would be assuming that there is only
one peak in the distribution function and that it is actually similar to a Gaussian.  This is not always true, so we can not assume it for our models. Generative models can be used for this purpose:
they will model the relationships in the data $x$. However,they ignore the context $c$ in which the data $x$ is involved. As an easy example of this, an image contains thousands of bits of information,
while the label that classifies the image contains much less information , say, $10$ bits for $1024$ categories. 

Our goal here will be to seek for a way of extracting shared information between the context and the data. Due to the differences in data dimensionality, if we want to predict the future $x$ using the context $c$, firstly we must
encode our entry data $x$ into a representation which size is comparable to context size. Firstly, an \emph{encoder} is used. An encoder is a model that, given an input $x$, provides a feature map or vector that holds the information
that the input $x$ had. In fact, and here is where we link the mutual information with the current topic, we want our encoder to maximize
$$
I(x,c) = \sum{x,c}p(x,c)\log\frac{p(x|c)}{p(x)}
$$
that is, the mutual information between the input $x$ and the context $c$.  Maximizing the mutual information between $x$ and $c$, we extract the latent variables
that the inputs have in common.
\clearpage
\chapter{Triplet Losses}
\label{Chapter:connection:triplets}
We have seen how the framework presented in \cite{oord_representation_2019} uses a generative approach as a part of the representation learning process. Let us set in the case of learning representations of images. In this case, generative models must \emph{generate} each pixel on the image. This can be extremely computationally expensive. 

Until now, we had been trying to minimize the loss in Equation \eqref{NCE:loss}, which we proved that maximizes a lower bound in the mutual information. However, some papers such as \cite{chen_simple_2020}, \cite{tschannen_mutual_2020}, suggest that it is unclear if the success of their methods is caused by the maximization of mutual information between the latent representations, or by the specific form that the constrastive loss has.

In fact, in \cite{tschannen_mutual_2020} they provide empirical proof for the loose connection between the success of the methods that use MI maximization and the utility of the MI maximization in practice. They also empirically proof  that the encoder architecture can be more important than the estimator used to determine the MI.

Even with the empirically proved disconnection between MI maximization and representation quality, recent works that have used the loss function \ref{NCE:loss} have obtained state-of-art results in practice. There is an explanation for this, connecting the recently mentioned loss with a popular triplet loss.

\section{From deep metric learning to triplet losses and its generalization}

Distance metric learning aims to learn an embedding representation of an input data $x$ that preserves the distance between similar data points close and also makes de distance between different datapoints far on the embedding space \citep{Sohn2016ImprovedDM}.

We will consider sets of triplets $(x,\ps,\ns)$ where:
\begin{itemize}
\item The element $x$ is an anchor point,
\item The element $\ps$ is a positive instance,
\item The element $\ns$ is a negative instance.
\end{itemize}

\begin{nexample}
    Let us present a very simple example. If our input image is a cat, that would be the anchor $x$. Clearly, a positive instance would be an image of another cat or even the same cat seen from another perspective. A negative instance would be a photo of any other animal, in this case we use a dog.
    \begin{figure}[H]%!htb]
        \minipage{0.32\textwidth}
          \includegraphics[width=\linewidth]{media/c1}
          \caption*{Anchor}\label{fig:cat1}
        \endminipage\hfill
        \minipage{0.32\textwidth}%
          \includegraphics[width=\linewidth]{media/c2}
          \caption*{Positive example}\label{fig:c2}
        \endminipage
        \minipage{0.32\textwidth}%
          \includegraphics[width=\linewidth]{media/doggo}
          \caption*{Negative example}\label{fig:doggo}
        \endminipage
        \caption{Example of an anchor $x$, a positive instance $\ps$ and a negative instance $\ns$. Images obtained from \emph{Google}.}
        \end{figure}
    \end{nexample}
    


The main idea is to learn a representation of $x$, say $g(x)$, such that the distance of the representation of the input is closer in distance to the representation of the positive sample $\ps$ than the representation of the negative sample $\ns$. Using the norm\footnotemark, we can formally express that as follows: 
$$
\norm{g(x) - g(\ps)}_2 \leq \norm{g(x) - g(\ns)}_2,
$$
for each triplet in the set.


%------------- Footnotemark
\footnotetext{A definition of the norm can be found on Appendix \ref{APPENDIX:A}, Definition \ref{def:norm}. }
%----------------------


Support-vector machines (SVMs) are supervised learning models used for classification or regression problems. They are one of the most robust predicion methods. They search for a hyperplane $h$ in high or infinite dimensional space that separates the data as much as possible, making use of \emph{support vectors}, the datapoints that are closest to the hyperplane. If the data is linearly separable, we can select two hyperplanes $h_1,h_2$ that are parallel to $h$ and making the distance from them to $h$ as large as possible. That region is called the \emph{margin}.

Coming back to our triplets problem, we also want to introduce a margin between the distances of the elements of the triplets, in order to separate positive examples from negative examples as much as possible. This way, we introduce a \emph{margin} term $\alpha$, rewriting our last equation as follows:
\[
\norm{g(x) - g(\ps)}_2 + \alpha < \norm{g(x) - g(\ns)}_2.
\]
Using this inequality, we can define a hinge loss function for each triplet in the set:
\begin{equation}\label{triplet:single:loss}
\ell^\alpha (x,\ps,\ns) = \max \left(0, \norm{g(x) - g(\ps)}_2^2 - \norm{g(x) - g(\ns)}_2^2 + \alpha\right).
\end{equation}
This loss has been defined for a single triplet. Now, we can define a global loss that accumulates the loss in Equation \eqref{triplet:single:loss} using all the triplets in set.

\begin{ndef}
Given a set of triplets, each containing an anchor, a positive example and a negative example, $\mathcal T = \{(x_i,\ps_i,\ns_i)\}_{i \in \Lambda}$, we define a triplet loss as follows:
\begin{equation}\label{triplet:sum:loss}
\mathcal L (x_i,\ps_i,\ns_i) = \sum_{i \in \Lambda} \ell^\alpha(x_i,\ps_i,\ns_i).
\end{equation}

\end{ndef}



We use this loss to train models in order to improve the representations obtained.It would be interesting to present the model non-trivial metric to the learning algorithm. When the representation $g$ improves, this is harder to do, and this results in slow convergence and expensive data sampling methods.

\subsection*{Generalization of triplet losses}

In a single evaluation of the loss function over a triplet during the learning process, we are comparing one positive sample to one negative sample. In practice, after looping over sufficiently many triplets, we expect the distance between positive examples and negative examples to be maximized. However, this will surely be a slow process if our dataset has many examples and also, in each step we will be separating the positive element from the specific negative element to which we are comparing it in that evaluation. Thus, the technique might be unstable \citep{Sohn2016ImprovedDM}.

In order to fix this, a good idea would be to compare in each evaluation a positive sample with multiple negative samples, generalizing the case exposed before. This way, we would like the positive sample to increase its distance to \emph{all} of the negative samples at the same time. Let us present a loss that generalizes the loss in Equation \eqref{triplet:sum:loss}.

\begin{ndef}
Let $\ps$ be a positive example of the anchor $x$, and consider the set $X^- = \{\ns_1,\cdots,\ns_{N-1}\}$ of $(N-1)$ negative samples. Given an encoder $g$, the $(N+1)-$tuplet loss is defined as follows:
\begin{equation}\label{nplus1:tuplet:loss}
\mathcal L_{(N+1)-\text{tuplet}}(x,\ps,X^-) = \log \left( 1+ \sum_{i=1}^{N-1} \exp \left(g(x)^T g(\ns_i) - g(x)^T g(\ps)\right)\right) 
\end{equation}
\end{ndef}

\begin{remark}
If we consider the case $N=2$, we have
\[
\mathcal L_{(2+1)-\text{tuplet}}(x,\ps,\ns) = \log \left( 1+ \exp \left(g(x)^Tg(\ns) - g(x)^T g(\ps)\right)\right).
\]
This expression is very similar to the one in Equation \eqref{triplet:single:loss}. In fact, if the norm in Equation \eqref{triplet:single:loss} is unit, if $g$ minimizes $\mathcal L_{(2+1)-\text{tuplet}}$, then it minimizes $\ell^\alpha$, and hence both losses are equivalent.
\end{remark}

Applying the $(N+1)-$tuplet loss in deep metric learning is computationally expensive. Indeed, if we apply Stochastic Gradient Descent (SGD) with batch size $M$, then we have to evaluate $M \times (N+1)$ times our function $\ell^\alpha$ in each update. Because of this, if we increase $M$ and $N$, the number of evaluations grows quadratically. We would like to avoid this.

Consider the set of $N$ pairs of examples, with the constraint of each pair belonging to a different class, i.e. $X = \{(x_1,\ps_1),\cdots,(x_N,\ps_N)\}$ with $y_i \neq y_j$ for all $i \neq j$. We now build $N$ tuplets where each tuplet has all the positive samples and the $i-th$ anchor, that is:
\[
\{S_i\}_{i=1}^N, \quad \text{where} \quad S_i = \{x_i, \ps_1,\cdots,\ps_N\}.  
\]
We can consider that each tuplet has $x_i$ as anchor, $\ps_i$ as positive example and $\ps_j$ for $j \neq i$ as negative samples, since they were all from different classes.



The InfoNCE loss on Equation \eqref{NCE:loss} has proved to be useful in representation learning. Let us consider a reformulation on it. Firstly, since $f_k$ was an exponential, we can also consider $e^f$ and remove the exponential from $f$, this is just notation. Now, we can rewrite the InfoNCE objective as follows
\begin{align*}
I_{NCE}  & = E\left[ \frac{1}{N} \sum_{i = 1}^N \log \frac{e^{f(x_i,y_i)}}{\frac{1}{N}\sum_{j=1}^N e^{f(x_i,y_j)}}\right]\\
& = a\\
& = \log N - E\left[ \frac{1}{N} \sum_{i=1}^N \log \left( 1+ \sum_{j\neq i}e^{f(x_i,y_j)- f(x_i,y_i)}\right)\right]
\end{align*}


\clearpage

\ctparttext{
  \color{black}
  \begin{center}
  In this part, deep learning and neural networks will first be motivated and later be presented. Then, the two most important frameworks that use contrastive learning for representation learning will be completely explained.
  \end{center}
}
\part{New Frameworks for Representation Learning}
\chapter{Introduction to Deep Learning}
\label{Chapter:Introduction:DL}

In order to be able to understand the chapters that will come later in this document, it is important to make a brief introduction of what \emph{Deep Learning}(DL) reffers to. Deep learning is included in the field of Machine Learning, which is also included in the field of general Artificial Intelligence.

In Chapter \ref{Chapter:Intro:Rep:Learning}, an intuitive definition of what Machine Learning is was given. We said that ML studies the algorithms that improve from experience. Tom M. Mitchell \citep{mitchell_machine_1997} provided a more formal definition of what \emph{learning from experience} means:

\begin{ndefC}
A computer program is said to \emph{learn} from experience $E$ with respect some class of tasks $T$ and performance measure $P$, if its performance at tasks in $T$, as measured by $P$, improves with experience $E$.
\end{ndefC}

We would also like to have a DL definition. In \cite{deng_deep_2014}, multiple similar definitions are given. We present here the simplest of them:

\begin{ndefC}
\emph{Deep Learning} is a class of ML learning techniques that exploit many layers of non-linear information processing for supervised or unsupervised feature extraction and transformation, and for pattern analysis and classification.
\end{ndefC}

Usually, these techniques are based on the biologically inspired \emph{neural networks}(NNs), which consists in several connected units: the \emph{neurons}. Each neuron is basically a Perceptron, which is a is a weighted sum followed by a non-linear function, called an \emph{activator} in the ML context. Formally, the output of each neuron is 
\[
y = \phi\left(w_0 +\sum_{i = 1}^N w_i x_i \right) .   
\]
There are many activation functions, but the following examples must be remarked:
\begin{itemize}
\item Sigmoid. The sigmoid function is defined as follows:
\[
\phi(x) = \frac{1}{1+ e^{-x}}.
\]
This is one of the most common used activation functions. It is differentiable, monotonic and smooth. One of its main disadvantages is that at the right part of the function, the change in the values that the function takes converges to zero, so we get to the \emph{vanishing gradient} problem and the learning is minimal.

\begin{figure}[H]
    \centering
    \includegraphics[width=0.5\linewidth]{sigmoid}
    \caption{Sigmoid. Image from \href{https://xzz201920.medium.com/activation-functions-linear-non-linear-in-deep-learning-relu-sigmoid-softmax-swish-leaky-relu-a6333be712ea}{this Medium article}. } \label{fig:sigmoid}
\end{figure}

\item Hyperbolic Tangent. This function is defined as follows:
\[
\phi(x) = \operatorname{tanh}(x) =  \frac{e^x - e^{-x}}{e^x + e^{-x}}.    
\]
This activation function has a small advantage over the sigmoid: its derivative is more steep, which means it can get more value and the learning can be more efficient.

\begin{figure}[H]
    \centering
    \includegraphics[width=0.6\linewidth]{tanh}
    \caption{Hyperbolic Tangent. Image from \href{https://xzz201920.medium.com/activation-functions-linear-non-linear-in-deep-learning-relu-sigmoid-softmax-swish-leaky-relu-a6333be712ea}{this Medium article}. } \label{fig:tanh}
\end{figure}

\item Rectified Linear Unit (ReLU). This function takes the following form:
\[
\phi(x) = \max\left(0,x\right).    
\]
ReLu is highly computationally efficient and non-linear. Its main problem is that when the inputs approach zero or are negative, the network can not perform back propagation and can not learn.

\begin{figure}[H]
    \centering
    \includegraphics[width=0.6\linewidth]{relu}
    \caption{ReLU. Image from \href{https://xzz201920.medium.com/activation-functions-linear-non-linear-in-deep-learning-relu-sigmoid-softmax-swish-leaky-relu-a6333be712ea}{this Medium article}. } \label{fig:relu}
\end{figure}
\end{itemize}



\section{Neural Networks}

Using neurons and activation functions, we can formally define NNs. A NN with $L$ hidden layers is a deterministic non-linear function $f$, parametrized by a set of matrices $W = \{W_0,\cdots,W_L\}$ and non-linear activation functions $\{\phi_0,\cdots,\phi_L\}$. Given an input $x$, the output $y$ of the network is calculated as follows:
\[
h_0 = \phi_0\left(W_0^Tx\right), \cdots, h_l = \phi_l\left(W_l^T h_{l-1}\right),\cdots, y = \phi_L\left(W_L^T h_{L-1}\right).
\]
Having a NN, we consider it \emph{deep} when the number of hidden layers (and, consequently, the number of matrices) is considered high. 

Neural networks use loss functions, which define how well the output returned by the network matches the real output, reducing the learning problem to an optimization problem. The problem is finding $W^{\operatorname{opt}}$, such that
\[
W^{\operatorname{opt}}   = \argmin_{w} \sum_{n = 1}^N l(y_n, f_w(x_n)),
\]
where $\mathcal D = \{(x_n,y_n)\}$ is a dataset.

This problem is solved using a variant of \emph{stochastic gradient descent (SGD)}. This algorithm involves the computation of the loss function derivatives respect to the network parameters, and updates the parameters using this derivatives. Specifically, the parameters are updated as follows:
\[
W_{t+1} = W_t + \eta \nabla f(W_t),
\]
where $\eta \in \R^+$ is a small constant called the \emph{learning rate}. This algorithm guarantees convergence to local minimums of $f$ and, if $f$ is convex, the algorithm converges to a global minimum.

The last comment about neural networks is that, since the weights $W = \{W_0,\cdots,W_L\}$ are constantly updated, the derivatives have to be computed repeatedly. The computational cost of this is quite high. \emph{Backpropagation} was born to calculate the derivatives of the weights much faster. The intuitive idea is that the gradient of the layer $l$ is computed using the gradient of the layer $l+1$ using the chain rule.

Understanding both SGD and Backpropagation is crucial for understanding how NNs  work. However, in the experimentation part of this work we will focus on researching how a few hyperparameters affect the results of the proposed frameworks, so no further explanation on these important concepts will be provided.
\chapter{SimCLR}
\label{Chapter:SimCLR}
\section{Introduction}

Until this point of the work, we have been presenting the theoretical basis of representation learning using contrastive learning. Previous approaches, such as  the framework presented in \cite{oord_representation_2019}, use a generative approach as a part of the representation learning process. Although this can be beneficial at some points and, in fact, achieved the \emph{state-of-art}\footnotemark empirical results, we have to consider that generative models have some drawbacks. 

%------------- Footnotemark
\footnotetext{\emph{State-of-art} refers to the best results that have been achieved at some point of time.}
%----------------------


Let us set in the case of learning representations of images to present a very simple example. In this case, generative models must \emph{generate} each pixel on the image. This can be extremely computationally expensive. 

Until now, we had been trying to minimize the loss in Equation \eqref{NCE:loss}, which we proved that maximizes a lower bound in the mutual information. However, some papers such as \cite{chen_simple_2020}, \cite{tschannen_mutual_2020}, suggest that it is unclear if the success of their methods is caused by the maximization of mutual information between the latent representations, or by the specific form that the constrastive loss has.

In fact, in \cite{tschannen_mutual_2020} they provide empirical proof for the loose connection between the success of the methods that use MI maximization and the utility of the MI maximization in practice. They also empirically proof  that the encoder architecture can be more important than the estimator used to determine the MI.

Even with the empirically proved disconnection between MI maximization and representation quality, recent works that have used the loss function defined in Equation \eqref{NCE:loss} have obtained state-of-art results in practice. 

\section{The framework}

\emph{SimCLR} \citep{chen_simple_2020} presents a framework that achieved state-of-art results when it was presented in July 2020. It also uses contrastive learning in an specific way that we will present later. 

This framework learns representations by maximizing agreement between examples of the same input obtained by using data augmentation on the input example and a contrastive loss in the latent space. 


\begin{figure}[H]
    \small
        \centering
    \begin{tikzpicture}
        \node at (0,1.8) (h) {$\longleftarrow\,$Representation$\,\longrightarrow$};
        \node[draw, circle] at (0,-1) (x) {$\,~\bm{x}~\,$};
        \node[draw, circle] at (-2.5,0) (x1) {$\tilde{\bm{x}}_i$};
        \node[draw, circle] at (2.5,0) (x2) {$\tilde{\bm{x}}_j$};
        \node at (-2.5,1.8) (h) {$\bm h_i$};
        \node at (2.5,1.8) (c) {$\bm h_j$};
        \node at (-2.5,3) (hh) {$\bm z_i$};
        \node at (2.5,3) (cc) {$\bm z_j$};
        \path[->] 
            (x)  edge [>=latex] node[below,rotate=-25] {$t\sim\mathcal{T}$} (x1)
            (x)  edge [>=latex] node[below,rotate=25] {$t'\sim \mathcal{T}$} (x2)
            (x1)  edge [>=latex] node[left,rotate=0] {$f(\cdot)$} (h)
            (x2)  edge [>=latex] node[right,rotate=0] {$f(\cdot)$} (c)
            (h)  edge [>=latex] node[left,rotate=0] {$g(\cdot)$} (hh)
            (c)  edge [>=latex] node[right,rotate=0] {$g(\cdot)$} (cc);
        \path[<->]
            (hh)  edge [>=latex] node[above,rotate=0] {Maximize agreement} (cc);
        \end{tikzpicture}
        \caption{Figure obtained from \cite{chen_simple_2020}. A simple framework for contrastive learning of visual representations.}
        \label{fig:framework:SimCLR}
    \end{figure}


The framework that is presented follows a linear structure. Figure \ref{fig:framework:SimCLR} depicts it. Let us provide with deeper explanation. We will present the steps in a general way and later we will remark the specific considerations that were used in the implementation of the framework for the experiments. 

The steps followed are:
\begin{enumerate}
\item Firstly, using the input $x$ and two augmentation functions $t,t' \in \mathcal T$, two different views $\tilde x_i,\tilde x_j$ are obtained using data augmentation. They are both sampled from the same family of data augmentations $\mathcal T$. They are \emph{both} considered as positive views.
\item Secondly, a NN base encoder $f(\cdot)$ is used to extract representations for the two different views, obtaining $h_i = f(\tilde{x_i})$, where $h_i \in \R^d$.  

\item Then, a \emph{small} neural network projection $g(\cdot)$ is used. This neural network maps the representations $h_i$ to the space where contrastive loss is applied. Hence, we obtain 
$$
z_i = g(h_i) = W^{(2)}\sigma(W^{(1)}h_i),$$
where $\sigma$ is a nonlinear function and $W^{(i)}$ are the weights matrix.

\item Lastly, the contrastive loss is used for a contrastive prediction task. Using a set $\{\tilde x_k \}$ that includes a pair of positive examples $\tilde x_i,\tilde x_j$, the contrastive loss will (as we have already been doing theoretically) try to identify $\tilde x_j$ in the set for a given $\tilde x_i$. It is important to remark how the contrastive loss is used in this framework:
\begin{enumerate}
\item A minibatch of $N$ samples is randomly taken from the training set. 
\item Using the $N$ samples, we augment each pair to obtain $2N$ data points. The idea is, given a positive pair, use the other $2(N-1)$ as negative examples.
\item We define
\[
sim(u,v) = \frac{u^T v}{\norm{u}\norm{v}},    
\]
the normalized dot product between $u$ and $v$. This function is also known as the \emph{cosine similarity}. Then, the loss function for a positive pair of examples takes the form:
\begin{equation}\label{c:loss:simclr:ind}
\ell_{i,j} = -\log \frac{exp(sim(z_i,z_j)/\tau)}{\sum_{k=1}^{2N} \mathbb{1}_{k \neq i} exp(sim(z_i,z_k)/\tau)},    
\end{equation}
where $\mathbb{1}_{k \neq i} \in \{0,1\}$ produces $0$ if $k = i$ and $1$ elsewhere. $\tau$ is the temperature parameter that we have mentioned before.
\item The final loss is computed across all the positive pairs, both $(i,j)$ and $(j,i)$ in the minibatch.
\begin{equation}\label{c:loss:simclr}
\mathcal L = \frac{1}{2N} \sum_{k=1}^{2N} \left(\ell(2k-1,2k) + \ell(2k,2k-1)\right).
\end{equation}
\end{enumerate}
\end{enumerate}
\begin{remark}
The loss in Equation \eqref{c:loss:simclr} is just another formulation of the one in Equation \eqref{NCE:loss}, applied to this particular way of obtaining positive and negative views.
\end{remark}

We can summarize all this steps in the following algorithm:
\begin{figure}[H]
\begin{algorithm}[H]
    \caption{\label{alg:main} SimCLR's learning algorithm.}
    \begin{algorithmic}
        \STATE \textbf{input:} batch size $N$, temperature $\tau$, structure of $f$, $g$, $\mathcal{T}$.
        \FOR{sampled minibatch $\{\bm x_k\}_{k=1}^N$}
        \STATE \textbf{for all} $k\in \{1, \ldots, N\}$ \textbf{do}
            \STATE $~~~~$draw two augmentation functions $t \!\sim\! \mathcal{T}$, $t' \!\sim\! \mathcal{T}$
            \STATE $~~~~$\textcolor{gray}{\# use augmentations} 
            \STATE $~~~~$$\tilde{\bm x}_{2k-1} = t(\bm x_k)$
            \STATE $~~~~$$\tilde{\bm x}_{2k} = t'(\bm x_k)$
            \STATE $~~~~$\textcolor{gray}{\# use $f$} 
            \STATE $~~~~$$\bm h_{2k-1} = f(\tilde{\bm x}_{2k-1})$  \textcolor{gray}{~~~~~~~~~~~~~~~~~~~~~~~~~~~~~~\# representation}
            \STATE $~~~~$$\bm h_{2k} = f(\tilde{\bm x}_{2k})$      \textcolor{gray}{~~~~~~~~~~~~~~~~~~~~~~~~~~~~~~~~~~~~~~\# representation}
            \STATE $~~~~$\textcolor{gray}{\# use $g$} 
            \STATE $~~~~$$\bm z_{2k-1} = g({\bm h}_{2k-1})$  \textcolor{gray}{~~~~~~~~~~~~~~~~~~~~~~~~~~~~~~~~\# projection}
            \STATE $~~~~$$\bm z_{2k} = g({\bm h}_{2k})$      \textcolor{gray}{~~~~~~~~~~~~~~~~~~~~~~~~~~~~~~~~~~~~~~~~\# projection}
        \STATE \textbf{end for}
        \STATE \textbf{for all} $i\in\{1, \ldots, 2N\}$ and $j\in\{1, \dots, 2N\}$ \textbf{do}
        \STATE $~~~~$ $s_{i,j} = \bm z_i^\top \bm z_j / (\lVert\bm z_i\rVert \lVert\bm z_j\rVert)$ \textcolor{gray}{~~~~~~~~\# compute similarity}\\
        \STATE \textbf{end for}
        \STATE update networks $f$ and $g$ to minimize $\mathcal{L}$ defined in Eq. \eqref{c:loss:simclr}
        \ENDFOR
        \STATE \textbf{return} encoder network $f(\cdot)$, and throw away $g(\cdot)$
    \end{algorithmic}
    \end{algorithm}
    
    \caption{Algorithm that summarizes the learning process that SimCLR follows.}
\end{figure}

Considerations:
\begin{enumerate}
\item In SimCLR, the data augmentation is applied sequentially, and using three techniques of image data augmentation that are very simple and common:
\begin{enumerate}
\item Random cropping followed by resize to original size: If an image is sized $n\times m$ with $n,m \in \mathbb N$, this method consists in selecting a rectangle of size $n'\times m'$ with $n' \leq n, m' \leq m$ of the original image and then resizing it back via upsampling\footnotemark.
\item Random color distortions, consisting in changing the value of certain amount of pixels randomly.
\item Random Gaussian Blur: Gaussian Blur consists in convolving an image with a \emph{Gaussian function}.
\end{enumerate}

\item For the base encoder $f(\cdot)$, a \emph{ResNet} network is used because of its simplicity.
\item For the projection head $g(\cdot)$, a MLP with one hidden layer is used, and \emph{ReLU} is used as the activation function $\sigma$.
\end{enumerate}

%------------- Footnotemark
\footnotetext{Upsampling is increasing the size of an image by inserting new rows and columns in the image matrix and interpolating the values of the introduced pixels using the pixels that we already had in the image.}
%----------------------

\section{Findings of SimCLR}

    Using the algorithm structure that we have presented, the experiments focused on finding what made the biggest impact on the representation learning task. A few of the most relevant findings using SimCLR are: 
    \begin{itemize}
    \item In machine learning, \emph{data augmentation} refers to the process of artificially creating new examples of the data by applying transformations to the data that we already have available. In the SimCLR framework, multiple data augmentations are used and it is empirically shown that this improves the contrastive prediction tasks that yield effective representations.
    
    \item The introduction of a nonlinear transformation that it can be learnt during the learning process. This linear transformation is between the representation and the contrastive loss, so before evaluating the loss function, the representation is applied the nonlinear function.
    
    \item The contrastive loss benefits from normalized embeddings and also from a temperature parameter that has to be adjusted.
    
    \item This loss also benefits from larger batch sizes and longer training, as well as from deeper and wider networks.
\end{itemize}

Later, \emph{SimCLRv2} was presented \citep{chen2020big}, achieving a new state-of-art in semi-supervised learning on the ImageNet dataset. In this new framework, semi-supervised learning is used. This approach involves unsupervised learning followed by supervised fine-tuning\footnotemark.


In the self-supervised part, the images are used without its labels, so that the representations of them that are obtained by the framework are not related to an specific task. Using this technique, \cite{chen2020big} shows how using deeper and wider neural networks for both self-supervised pretraining and fine-tuning improves accuracy greatly. Also, the importance of the projection head $g(\cdot)$ is remarked in this new framework.
%------------- Footnotemark
\footnotetext{\emph{Fine-Tuning} refers to the process of re-training certain parts of an already trained NN so that it focuses its learn on specific examples, such as a different dataset.}
%----------------------
\chapter{Boostrap Your Own Latent}
\label{Chapter:BYOL}

\section{Motivation}

We have shown how contrastive methods rely on comparing different views of the same image with views of other images, considering the ones from the same image as positive and the rest as negative samples. This is why they are called \emph{self-supervised} methods.

Self-supervised methods build upon the cross-view prediction framework, i.e., learning representations by predicting different views (or data augmentations) of the same image. This could lead the frameworks to collapsed representations, such as a constant representation for any view of the image can easily help us to identify objects, but it is useless for downstream tasks.

The methods presented earlier, which use contrastive learning, avoid this problem by trying to discriminate between positive and negative views, as we have already explained.

However, some papers (e.g. \cite{caron2019deep}) have already raised the following question:

\begin{quote}
    \centering
\emph{ Is the use of negative samples necessary to avoid collapsing? }. 
\end{quote}
This question is studied in \cite{grill2020bootstrap}, paper that presents an algorithm that we will be deeply explaining.

The first solution that comes to mind to prevent the collapsing problem would be to use a randomly initialized network to produce the targets of the predictions. As it is probably expected, due to its randomness, it does not produce good representations for downstream tasks. Nonetheless, the representations obtained were empirically much better than initial fixed representation, so it could be interesting to refine this representation in order to make it better for the later tasks. This is the intuitive idea behind \emph{Boostrap your own latent (BYOL)} \citep{grill2020bootstrap}.

\section{BYOL Algorithm}

BYOL's algorithm has certain similarities with the SimCLR framework that we presented in Chapter \ref{Chapter:SimCLR}. The goal of this framework is to learn a representation for an input. In this case, the representation will be noted as $y_\theta$. 

For this purpose, two neural networks are used:
\begin{itemize}
\item An \emph{online} network defined by a set of weights $\theta$.

\item A \emph{target} network, defined by a different set of weights $\upxi$.
\end{itemize}

They both have the same structure, composed of three stages:
\begin{enumerate}
\item An encoder $f_\gamma$,
\item A projector $g_\gamma$,
\item A predictor $q_\gamma$,
\end{enumerate}
where $\gamma \in \{\theta,\upxi\}$. In the online network, despite their different name, the projector $g_\theta$ and the predictor $q_\theta$ have the same architecture.

\begin{remark}
The projector $g_\gamma$ is used because in SimCLR \citep{chen_simple_2020} is proven empirically that this projection improves the general performance of the framework.
\end{remark}

The \emph{difference} between them is that the target network provides the regression targets to train the online network, and its parameters $\upxi$ are an exponential moving average\footnotemark of the online parameters $\theta$. Mathematically, given a rate decay $\tau \in [0,1]$, after each training step $\upxi$ is updated as follows:
\[
\upxi \leftarrow \tau \upxi + (1-\tau)\theta    
\]

%------------- Footnotemark
\footnotetext{A moving average is a calculation to analyze data points by creating a series of averages in different subsets of the data. }
%----------------------

Having presented the networks and its structure, the following steps are followed in BYOL's framework:

\begin{enumerate}
\item The input that both networks receive is different, even if it comes from the same image. As in SimCLR, given an input image $x$ and two distributions of data augmentation for images, $\mathcal T, \mathcal T'$, two views are produced from $x$ to get $v = t(x), v' = t'(x)$ where $t \sim \mathcal T $ and $t' \sim \mathcal T'$.

\item Each produced view is passed to one of the networks. In particular, the first view $v$ is passed to the online network, and follows the next steps:
\[
x \longmapsto v = t(x) \longmapsto y_\theta = f_\theta(v) \longmapsto z_\theta =  g_\theta(y_\theta)    
\]
where $f_\theta,g_\theta$ are the ones that we mentioned before in the structure of each networks. This online network outputs $y_\theta$ and $z_\theta$.

In a similar process, the the target network is passed the second view $v'$, which follows the next steps:
\[
x \longmapsto v' = t'(x) \longmapsto y_\upxi' = f_\upxi(v') \longmapsto z_\upxi' = g_\upxi(y_\upxi')   
\]
where, again, $f_\upxi,g_\upxi$ are the ones mentioned before.


\item Then, using the online network, a prediction $q_\theta(z_\theta)$ is produced. Remark that the prediction is \emph{only} applied to the online network.

\item Having $q_\theta(z_\theta)$ in the online network and $z_\upxi'$ in the target network, they are both $\ell_2$-normalized to
\[
\overline{q_\theta}(z_\theta) = \frac{q_\theta(z_\theta)}{\norm{q_\theta(z_\theta)}} \quad \text{and} \quad \overline{z_\upxi'} = \frac{z_\upxi'}{\norm{z_\upxi'}}. 
\]

\item \label{BYOL:not:last} Now, we can define the mean squared error between the normalized prediction $\overline{q_\theta}(z_\theta)$ and the normalized projection $\overline{z_\upxi'}$:
\[
\mathcal L_{\theta,\upxi} = \norm{\overline{q_\theta}(z_\theta) - \overline{z_\upxi'}}_2^2. 
\]

\item If we stopped in the step \ref{BYOL:not:last}, the framework would be asymmetric between the two networks, since the projection is performed in one of the views, $v$ but not in the other one, $v'$. To fix this, the process described is repeated except that now $v'$ is the input of the online network and $v$ is the input of the target network, producing a new loss $\tilde{\mathcal L}_{\theta,\upxi}$. The final loss is computed as:
\begin{equation}\label{Loss:BYOL}
\mathcal L_{\theta,\upxi}^{\operatorname{BYOL}}     = \mathcal L_{\theta,\upxi} + \tilde{\mathcal L}_{\theta,\upxi} 
\end{equation}
\end{enumerate}

The loss in Equation \eqref{Loss:BYOL} is the one that must be optimized stochastically. Furthermore, since we have expressed $\upxi$ depends on $\theta$, if $\eta$ is the learning rate that we want to apply, the optimization problem can be expressed as follows:
\[
\begin{cases}
    \theta \gets \operatorname{optimizer}\left(\theta,\nabla_\theta \mathcal L_{\theta,\upxi}^{\operatorname{BYOL}},\eta \right), \\
    \upxi \gets \tau \upxi + (1-\tau)\theta 
\end{cases}
\]

The framework can be summarized in the following figure:
\begin{figure}[H]
    \centering 
    \includegraphics[scale=0.8]{Archi-BYOL.pdf}
    \caption{Image from \citep{grill2020bootstrap}. Overview of Bootstrap Your Own Latent Framework. }
\end{figure}

The \emph{sg (stop gradient)} indicates that, at that point, the gradient is not propagated back. 

After the whole model has been fully trained, both the projection $g_\theta$  and the prediction $q_\theta$ are discarded, since what it is interesting for us is the representation of the input, and it is what it will be used in downstream tasks.

Some considerations have to be made about this framework:
\begin{enumerate}
    \item About data augmentation: In BYOL,it is not important if the model learns about the histogram of the image, since we want ot keep any information captured by the target representation into its online network. In the original paper, it is empirically shown how BYOL is more robust to the choice of the image augmentations than SimCLR.

    \item Since the weights of the target network $\upxi$ depend on the weights of the online network $\theta$ by a parameter $\tau$, the first mentioned weights $\upxi$ represent a delayed and stable version of $\theta$. Actually, if the decay rate $\tau$ is $1$, we never update $\upxi$, and if $\tau = 0 $, then the weights of the target network are always being updated. This way, a trade-off is tablished between updating $\upxi$ too often or too slowly. Empirically, it has been shown that the most adequate values that yield to more stable results are $\tau \in [0.9,0.999]$.
\end{enumerate}

\section{Results obtained by this framework}

In this framework, the impact of the batch size varies respect to the importance that it had in SimCLR. In contrast with BYOL's framework, in SimCLR we were using negative samples in our network to learn how to produce the representations. Because of this, BYOL's is expected to be more robust to smaller batch sizes. 

It is empirically shown that the regularization of the weights is crutial in the self-supervised setting. The experiments in the paper showed that removing the weight decay regularization led to model divergence.


BYOL achieved the state-of-art performance results in the linear evaluation. This is, again, one of the most interesting parts since it helps us to evaluate how good the representations that our model create are.

\begin{table}[H]
    \label{table:best:first:simclr}
\centering
\begin{tabular}{llrrr}
Method & Architecture & Params & Top-1 & Top 5 \\ \hline
SimCLR & ResNet-50($2\times)$ & 94M & 74.2 & 92.0 \\
BYOL & ResNet-50($2\times)$ & 94M & \textbf{77.4} & \textbf{93.6} \\
SimCLR & ResNet-50($4\times)$ & 375M & 76.5& 93.2 \\
BYOL & ResNet-50($4\times)$ & 375M & \textbf{79.6}& \textbf{94.8} \\
\end{tabular}
\caption{Comparison between SimCLR and BYOL Top1 and Top5 accuracies using the same architectures on the ImageNet dataset.}
\end{table}

BYOL also outperformed other models such as \emph{MoCo} \citep{he2020momentum} and a second version of the Contrastive Predictive Coding framework \citep{hénaff2020dataefficient}, but they have not been included in the table since they are not deeply studied in this work.

\ctparttext{
  \color{black}
  \begin{center}
  In this last part, the frameworks that we have presented in the last part will be tested. We will train the models in a specific dataset and the results obtained will be compared with the results of the original papers.
  \end{center}
}
\part{Experiments}
\chapter{Introduction}
In this chapter, we will explain the fundamentals and technologies that have been used for the experimentation. We will focus on three main aspects:
\begin{enumerate}
\item The used dataset.
\item The used libraries for the development of the code.
\item Analysis of the code itself.
\item The used metrics to evaluate the obtained results.
\end{enumerate}

The idea of the experimentation part is to test and compare the frameworks that we have presented in Chapters \ref{Chapter:SimCLR} and \ref{Chapter:BYOL}. Their architectures have already been explained, and the original code for both backbones has not been done by me. 

This work will focus on testing how changing the training hyperparameters of the model affects the final results, since the original papers \cite{chen_simple_2020,grill2020bootstrap} already mention that using their structure, the results are affected by those hyperparameters, such as batch size, network depth or network width.

The implementations that have been used can be found in:
\begin{itemize}
\item SimCLR implementation: Official implementation from Google in \url{https://github.com/google-research/simclr/tree/master/tf2}

\item BYOL implementation: not official, found in \url{https://github.com/garder14/byol-tensorflow2}. 
\end{itemize}

Although there exists an official implementation of BYOL, non-official one has been chosen because it uses \emph{Tensorflow}, which makes the implementations easier to understand and modify, which we needed to do. Also, the idea is to make use of \emph{Tensorboard}, a Tensorflow utility that helps with visualization and graph generation of the training and final results.


\section{Hardware and basic libraries}

When running training experiments in DL, the used hardware is one of the most determining factors for the results obtained by the models. There are many reasons for this, such as the time spent on the training a model or the amount of  data that we can fit in the GPUs (which will be our case) or TPUs.

For the experiments of this project, the DECSAI\footnotemark department of the University of Granada generously provided us access to a server that has a few NVIDIA GeForce RTX 3090. This model of GPU is one of the best in the market, having a \emph{compute capability} of $8.6/10$, rated by the NVIDIA company. IT also has a memory of $24GB$, which allows us to experiment with relatively large batch sizes using a single GPU and not having to parallelize the experiment.

%------------- Footnotemark
\footnotetext{The DECSAI's website is \url{https://decsai.ugr.es/}.}
%----------------------


To be able to use this GPUs in our experiments, two basic libraries have to be installed in the server:
\begin{enumerate}
\item \lstinline{tensorflow-gpu}. This is a variant of the \lstinline{tensorflow} library that was developed for using GPUs while using tensorflow.
\item \lstinline{CUDA 11.4}. CUDA is a parallel computing platform and API created by NVIDIA which allows to use a CUDA-enabled GPU for general purpose processing. In other words, CUDA is needed to be able to use the GPU in our \lstinline{python} scripts.
\end{enumerate}

These two libraries, along with other packages that are needed for creating graphics (\lstinline{Matplotlib,Seaborn}) or computing metrics (\lstinline{sklearn}) are installed using a \lstinline{Conda} enviroment in our server's user.


\section{The dataset: CIFAR10}

The computational resources that we have for the experiment are limited. Due to this, we must fix a dataset that, having enough and representative examples, allows us to achieve feasible training time and successful results.

One of the ever most used dataset, which was also used in both SimCLR and BYOL papers, is CIFAR10 \citep{krizhevsky_learning_nodate}. This dataset will be used to test the overall performance of our representation learning methods.

CIFAR10 contains $60.000$ images divided in $10$ classes, where each class contains $10.000$ images. The size of the images is $32\times 32\times 3$, so the size of the images is not very large. This helps us to have faster trainings.

\begin{figure}[H]
    \centering 
    \includegraphics[scale=0.8]{CIFAR10}
    \caption{Ten examples of each class in the CIFAR10 dataset. }
\end{figure}
This dataset has $50.000$ samples for training and $10.000$ for test. The test batch contains the same number of examples of each of the $10.000$ classes in the dataset, that is, it contains $1.000$ examples of each class. 

It is important to remark that the classes are \emph{completely mutually exclusive}. That means that there is no overlap between the classes even if they have similar images, such as \emph{Cars} and \emph{Truck}, which are two of the classes of the dataset.

\section{Tensorflow}



Tensorflow\footnotemark is an open source library for developing machine learning frameworks.  

\begin{wrapfigure}{r}{5cm}
    \caption{Tensorflow logo.}
    \includegraphics[scale=0.2]{tf-logo}
\end{wrapfigure}

%------------- Footnotemark
\footnotetext{Tensorflow documentation can be found at \url{https://www.tensorflow.org/}.}
%----------------------



It can be used for many tasks, but it focuses on training and inference of deep neural networks. It is used for both research and production at \emph{Google}, since it was also developed by the \emph{Google Brain} team for internal use. However, it was later released as open source.

The creation of new models is very simple, offering multiple abstraction levels. This is why it is suitable for our experiments. Also, the code is most of the times easily understandable.

There are other libraries that are widely used in machine learning algorithm development, such as \emph{Pytorch} or \emph{Jax}. However, Tensorflow has been chosen because I was a little bit more familiarized than with the other and because of its simplicity and how common it is. 

There are a few ways to define a NN or a framework using tensorflow. The most classic one is defining a sequential model using \lstinline{Keras}, a tensorflow API that defines layers of a neural network and helps with the implementation of simple NN structures. Let us see how to implement a very simple example of a NN with three \emph{Dense} layers: 

\begin{minted}[mathescape,linenos]{python}
model = keras.Sequential(
    [
        layers.Dense(2, activation="relu", name="layer1"),
        layers.Dense(3, activation="relu", name="layer2"),
        layers.Dense(4, name="layer3"),
    ]
)
\end{minted}

Another way of creating models using tensorflow is by defining a single step of training using \lstinline{tf.GradientTape()} and then executing the single step multiple times in a loop. Using GradientTape , tensorflow performs automatic differentiation, which is needed for the minimization process. Let us see the simplest example, consider the function $f(x) = x^2$, and imagine that we want to obtain $f'(3)$. We can obtain it using \lstinline{tf.GradientTape()} as follows:

\begin{minted}[mathescape,linenos]{python}
x = tf.constant(3.0)
with tf.GradientTape() as g:
  g.watch(x)
  y = x * x
dy_dx = g.gradient(y, x)
\end{minted}

In our case, the gradient is obtained and the applied to the optimizer by using:

\begin{minted}[mathescape,linenos]{python}
with tf.GradientTape() as tape:
    grads = tape.gradient(loss, model.trainable_variables)
    optimizer.apply_gradients(zip(grads, model.trainable_variables))
\end{minted}


\subsection{Tensorboard}

Tensorboard is a Tensorflow's visualization kit. It provides the visualization and tooling needed for machine learning experimentation. Among its more important utilities, we can find:
\begin{itemize}
\item Tracking and visualizing metrics (such as loss, accuracy, entropy) not only during the training but also when the training time has ended.

\item Visualizing the model graph: ops and layers.

\item Visualizing histograms of weights, biases and how tensors change during the training.

\item Projecting high-dimensional data to a lower dimensional space.

\item Displaying images,text and audio data.
\end{itemize}

Also, it is very easy to integrate with tensorflow.  Actually, in most of the cases it is as simple as adding the following \emph{callback} when we fit the model:

\begin{minted}[mathescape,linenos]{python}
    tensorboard_callback = tf.keras.callbacks.TensorBoard
                        (log_dir=log_dir, histogram_freq=1)
    model.fit(x=x_train, y=y_train, 
              epochs=5, 
              validation_data=(x_test, y_test),
              # The added callback produces the magic! 
              callbacks=[tensorboard_callback])  
\end{minted}

In our case, there are some differences, since we are not using the standard \lstinline{fit} function to train our models. Because of this, we have to log the information that we have obtained in each step. To do this, we can use the \lstinline{metrics} python package to group them (as it is done in the SimCLR code), or we can just directly save the information creating a \emph{file writer} and writing the desired variables on this file. We do this in the modification that we have done to BYOL's original code as follows:

\begin{minted}[mathescape,linenos]{python}
train_summary_writer = tf.summary.create_file_writer(args.log_dir)
with train_summary_writer.as_default():
    tf.summary.scalar('top_1_acc',float(acc),step=epoch)
    tf.summary.scalar('top_5_acc',float(top_5_acc),step=epoch)
    tf.summary.scalar('loss', float(losses[-1]), step=epoch)
\end{minted}


\section{Metrics}

As we have seen, firstly, our models create a representation of the input image and then this representation is evaluated using a supervised linear head. This is the most interesting part, since we can see if the representation obtained was really useful for the classification task. We need to present the measures that we will use to measure how good the representations that we are producing are.



\begin{notation}
We will address the \emph{true positives} (the positive samples of a class classified correctly) as $TP$, the \emph{true negatives} (the negative samples classified correctly) as $TN$, the \emph{false positives} (the negative samples classified as positive ones, which is a mistake of our model) as $FP$ and the false negatives (the positive samples classified incorrectly as negative samples) as $FN$.
\end{notation}

Using this notation, the measures that we will be using are the following:

\begin{itemize}
\item \emph{Accuracy}. The classic measure for models. It measures the number of successes our network has obtained producing the correct label for the representation created during the unsupervised part of the network. Formally,
\[
\operatorname{Accuracy} = \frac{TP + TN}{TP + TN + FP + FN}  .  
\] 
In SimCLR, we wil use two kinds of accuracies:
\begin{enumerate}
\item \emph{Top1} accuracy, which is the ordinary accuracy.
\item \emph{Top5} accuracy, which measures if any of the 5 highest probability answers matches the true label.
\end{enumerate}
\end{itemize}





\chapter{Experimentation}
We are now ready to perform the experiments. We will begin exploring SimCLR implementation and results, later we will explore BYOL's implementation and lastly we will compare them in order to see how BYOL tries to improve SimCLR and we will check if it successess or not.

The code used for the experimentations, as well as some files with the results, can be found on the Github repository for this work.

\section{SimCLR exploration}
We will perform an iterative exploration with this framework. We will explore a few range for a subset of the hyperparameters and then we will go deeper into some hyperparameters to try and obtain better results.

\subsection{First approach}
\label{experiments:simclr:first}

The first thing we did to experiment with this framework is to explore a wide range of hyperparameters to see which set of them performed better for us. The script that can be found in \lstinline{code/SimCLR/run.py}.

What we did for this first exploration was to define a \emph{parameter grid} and execute the whole framework using each combination of the parameters. The parameters that were firstly considerered are:
\begin{enumerate}
\item \lstinline{batch_size}. This is one of the most important parameters of SimCLR. In the original paper, it was proven that the higher this parameter, the better the results obtained for the linear classification. This parameter is also important since it has to adapt to our GPU's memory. The options that we have considered for the first experiments are: \lstinline|batch_size = {512,1024}|.

\item \lstinline{temperature}. The temperature parameter $\tau$ plays an important role in the individual loss seen in Equation \eqref{c:loss:simclr:ind}. It is suggested to try values in the range $[0,1]$ and the one that had better performance for the original experiments was around $0.5$, so we chose the following values: \lstinline|temperature = {0.25,0.5,0.75,1}|.

\item \lstinline{color_jitter_strength}. This parameter measures how hard is the color variation in the data augmentation. Previous results show that this parameter is important for the success of the network, so we provide with a big range of values. We include the following values: \lstinline|color_jitter_strength = {0.25,0.5,0.75,1}|.
\end{enumerate}

Using python and the python function \lstinline{itertools.product} we straightforwardly generate all possible different combinations of unions of the parameters so we only have to append them to a general string and execute the \lstinline{run.py} script mentioned before to obtain the results. In total, we obtain $32$ possible combinations, so we will obtain $32$ models. 

The script was executed and took approximately $24$ hours to train and evaluate all different models, obtaining the results in Table \ref{table:experiment:first:simclr} in Appendix \ref{APPENDIX:B}. We have to remark, for each of the two possible batch sizes, the best results obtained. These are:

\begin{table}[H]
    \label{table:best:first:simclr}
\centering
\resizebox{\columnwidth}{!}{
\begin{tabular}{rrrrrrr}
batch\_size   & temperature   & color\_jitter & regularization\_loss & top\_1\_accuracy & top\_5\_accuracy & steps         \\ \hline
 512  &  0.25 &  0.25 &  0.0093      &  0.833         &  0.994          &  9800 \\
 1024 &    0.25       &  0.75 &  0.0093      &  \textbf{0.841}          &  0.995          &  4900 \\
\end{tabular}
}
\caption{Best results for the grid search experiment with SimCLR.}
\end{table}

Let us make some observations. The clearest is that, since the second batch size doubles the first one, the training ends in half the steps. We can see that the temperature obtained in both cases is the same, and it is equal to $0.25$. However, there is a big change in the color jitter parameter: while using $512$ as batch size obtains the best result with the value of $0.25$ for color jitter, using $1024$ as batch size this value changes to $0.75$, which implies stronger color changes on the data augmentation. 

Although both models get to the same value of regularization loss, the model with a batch size of $1024$ obteins a higher \emph{top 1 accuracy}.

\begin{figure}[htp] 
    \centering
    \subfloat[Tulips]{%
        \includegraphics[width=0.45\textwidth]{train_supervised_acc}%
        \label{fig:accs:}%
        }%
    \hfill%
    \subfloat[Rotated, flipped tulips]{%
        \includegraphics[width=0.45\textwidth]{train_supervised_acc}%
        \label{fig:rot:flip:flowers}%
        }%
        \caption{The image on the left is the original image, and the image on the right is a new example generated by first a rotation of a random angle and then a flip.}
\end{figure}

\newpage

\chapter*{Conclusions and further work}

\newpage

\newpage

\chapter*{Acknowledgements}
I would like to thank my tutor, Professor Nicolás Pérez de la Blanca, who has greatly guided me in this work, patiently answering my questions and doubts about the topic.

I would also like to thank my friends for supporting me not only reading this work and providing feedback, but also during the whole bachelor's degree. Specially, thank you to my girlfriend Isabel, who has been my solid ground and has pushed me to be constant and resilient.  

Finally, I would like to thank my family, my very special, weird, but full of love family.

Thank you all. 


\appendix
\renewcommand{\thechapter}{\Alph{chapter}}
\ctparttext{
  \color{red}
  \begin{center}
    Appendices
  \end{center}
}
\part{Appendix}
\chapter{Appendix A}
\label{APPENDIX:A}

This appendix will be used to set forth some theoretical results that might not always be relevant but are needed to understand some details during this thesis. Not all of them will be proven.

\begin{nprop}[Jensen's Inequality]\label{prop:jensen}
    Let $f:\mathcal D \to \R$ be a concave function and $n\in \mathbb N$. For any $p_1,\dots,p_n \in \R^+_0$ with $\sum p_i = 1$ and any $x_1,\dots,x_n \in \mathcal D$, it holds that:
    $$
    \sum_{i = 1}^n p_i f(x_i) \leq f \left(\sum_{i=1}^n p_i x_i\right).
    $$
    Furthermore, if $f$ is \emph{strictly} concave and $p_i \geq 0$ for all $i  = 1,\dots,n$, then the equality holds if, and only if, $x_1 = \dots = x_n$.
\end{nprop}
\chapter{Appendix B}

\label{APPENDIX:B}
In this appendix we will insert some experimental results that have been obtained during this work, but have not been presented before since not all of them are completely relevant.
The most relevant ones have already been presented, so this section is not needed for complete understanding of this work.
\section{SimCLR Experiments}

The results obtained for the first experiment, which was presented in Section \ref{experiments:simclr:first}, are the following:

\begin{table}[H]
    \label{table:experiment:first:simclr}
\centering
\resizebox{\columnwidth}{!}{
\begin{tabular}{rrrrrrr}
batch\_size   & temperature   & color\_jitter & regularization\_loss & top\_1\_accuracy & top\_5\_accuracy & steps         \\ \hline
\textbf{512}  & \textbf{0.25} & \textbf{0.25} & \textbf{0.0093}      & \textbf{0.833}          & \textbf{0.994}          & \textbf{9800} \\
              &               & 0.5           & 0.0089               & 0.832                   & 0.993                   & 9800          \\
              &               & 0.75          & 0.0086               & 0.831                   & 0.994                   & 9800          \\
              &               & 1.0           & 0.008                & 0.83                    & 0.992                   & 9800          \\
              & 0.5           & 0.25          & 0.0136               & 0.819                   & 0.993                   & 9800          \\
              &               & 0.5           & 0.0124               & 0.822                   & 0.993                   & 9800          \\
              &               & 0.75          & 0.0121               & 0.821                   & 0.993                   & 9800          \\
              &               & 1.0           & 0.0118               & 0.817                   & 0.992                   & 9800          \\
              & 0.75          & 0.25          & 0.0161               & 0.809                   & 0.993                   & 9800          \\
              &               & 0.5           & 0.015                & 0.812                   & 0.993                   & 9800          \\
              &               & 0.75          & 0.0141               & 0.805                   & 0.99                    & 9800          \\
              &               & 1.0           & 0.0137               & 0.793                   & 0.99                    & 9800          \\
              & 1.0           & 0.25          & 0.017                & 0.798                   & 0.99                    & 9800          \\
              &               & 0.5           & 0.0163               & 0.797                   & 0.99                    & 9800          \\
              &               & 0.75          & 0.016                & 0.793                   & 0.99                    & 9800          \\
              &               & 1.0           & 0.0155               & 0.782                   & 0.989                   & 9800          \\
\textbf{1024} & \textbf{0.25}          & 0.25          & 0.0103               & 0.836                   & 0.993                   & 4900          \\
              &               & 0.5           & 0.0097               & 0.839                   & 0.995                   & 4900          \\
              &               & \textbf{0.75} & \textbf{0.0093}      & \textbf{0.841}          & \textbf{0.995}          & \textbf{4900} \\
              &               & 1.0           & 0.009                & 0.835                   & 0.993                   & 4900          \\
              & 0.5           & 0.25          & 0.0149               & 0.822                   & 0.993                   & 4900          \\
              &               & 0.5           & 0.0133               & 0.827                   & 0.994                   & 4900          \\
              &               & 0.75          & 0.0132               & 0.826                   & 0.994                   & 4900          \\
              &               & 1.0           & 0.0128               & 0.82                    & 0.993                   & 4900          \\
              & 0.75          & 0.25          & 0.0168               & 0.814                   & 0.991                   & 4900          \\
              &               & 0.5           & 0.0166               & 0.816                   & 0.992                   & 4900          \\
              &               & 0.75          & 0.0157               & 0.813                   & 0.993                   & 4900          \\
              &               & 1.0           & 0.0153               & 0.802                   & 0.989                   & 4900          \\
              & 1.0           & 0.25          & 0.0175               & 0.806                   & 0.991                   & 4900          \\
              &               & 0.5           & 0.0174               & 0.802                   & 0.992                   & 4900          \\
              &               & 0.75          & 0.017                & 0.794                   & 0.99                    & 4900          \\
              &               & 1.0           & 0.0164               & 0.79                    & 0.989                   & 4900         
\end{tabular}
}

\caption{All results for first experiment using SimCLR.}
\end{table}



\nocite{*}
\bibliographystyle{authordate1}
\bibliography{Bibliography}

\end{document}
