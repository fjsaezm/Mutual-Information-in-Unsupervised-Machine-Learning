We have introduced the concepts of \emph{random variable},  \emph{random vector} and its \emph{probability distribution}.
Now, given two distributions, in the following chapters we will like to see how different they are from each other.
In order to compare them, we enunciate the definition of the Kullback-Leibler divergence.

\begin{ndef}
Let $P$ and $Q$ be probability distributions over the same probability space $\Omega$. Then, the Kullback-Leibler divergence is defined as:
$$
D_{KL}(P \ || \ Q) = E_P\left[\log{\frac{P(x)}{Q(x)}}\right].
$$
\end{ndef}
It is defined if, and only if, $P$ is \emph{absolutely continuous with respect to} $Q$, that is, if $P(A) = 0$ for any $A$ subset of $\Omega$ where $Q(A) = 0$.
 There are some properties of this definition that must be stated. 

\begin{nprop}
If $P,$ $Q$ are two probability distributions over the same probability space, then $D_{KL}(P|Q) \geq 0$.
\end{nprop}
\begin{proof}
Firstly, note that if $a \in \R^+$, then $\log \ a \leq a-1$. Then:
\begin{align*}
-D_{KL}(P \ || \ Q) & = - E_P\left[\log{\frac{P(x)}{Q(x)}}\right] \\
             & = E_P\left[\log{\frac{Q(x)}{P(x)}}\right] \\
             & \leq E_P\left[\left(\frac{Q(x)}{P(x)} - 1\right)\right]\\
             & = \int P(x) \frac{Q(x)}{P(x)} \mathop{dx} -1 \\
             & = 0.
\end{align*}
So we have obtained that $-D_{KL}(P\ ||\ Q) \leq 0$, which implies that $D_{KL}(P\ || \ Q) \geq 0$.
\end{proof}
As a corollary of this proposition, we can affirm that $D_{KL}(P\ ||\ Q)$ equals zero if and only if $P = Q$ almost everywhere. 
We will also remark the discrete case, as it will be used later. Let $P,Q$ be discrete probability distributions defined on the same probability space $\Omega$. Then, 
$$
D_{KL}(P\ ||\ Q) = \sum_{x \in \Omega} P(x) \log \left( \frac{P(x)}{Q(x)}\right).
$$

\section{Examples of distributions}

Let us present some examples o common distributions. They will be used further in this document.

\subsection*{Bernoulli}

Think for a moment that you want to model the possible outcomes of an experiment with two possibilites: sucess or failure. Imagine also that you already know that in your experiment there is a probability $p$ of 
achieving success. That is the intuitive idea of a Bernoulli distribution. We can define it more formally as follows: 

The \emph{Bernoulli distribution} is a discrete probability distribution of a random variable that takes two values, $\{0,1\}$, with probabilities $p$ and $q = 1-p$, respectively. We will say that our distribution is a $Bern(p)$.

If $k$ is a possible outcome, we can define
the probability mass function $f$ of a Bernoulli distribution as:
$$
f(k,p) = 
\begin{cases} 
p, \quad & \text{ if } k=1,\\
1-p, \quad & \text{ if } k = 0.
\end{cases}
$$
Using the expression of the mean for discrete random variables, we obtain that $E[X] = p$ and 
$$
\Var[X] = E[X^2] - E[X]^2 = E[X] - E[X]^2 = p-p^2 = p(1-p) = pq.
$$

As a note, this is just a particular case of the \emph{Binomial distribution} with $n=1$.

\subsection*{Gaussian Distribution}

The Gaussian (or normal) distribution is used to represent real-valued random variables whose distributions are not known.
Its importance relies in the fact that, using the \emph{central limit theorem}, we can assume that the average of many samples of
a random variable with finite mean and variance is a random variable whose distribution converges to a normal distribution as the number of samples increases.

\begin{ndef}
We say that the real valued random variable $X$ follows a \emph{normal distribution} of parameters $\mu,\sigma\in \R$ if, and only if,
its probability density function exists and it is determined by
\[
f(x) = \frac{1}{\sigma \sqrt{2\pi}}e^{-\frac{1}{2}\left( \frac{x - \mu}{\sigma}\right)^2},
\]
where $\mu$ is the mean and $\sigma$ is its standard deviation. We denote this normal distribution as $X \sim \mathcal N (\mu,\sigma)$.
\end{ndef}

The particular case where $\mu = 0$ and $\sigma = 1$ is widely used in statistics. In this case, the density function is simpler:
\[
f(x) = \frac{1}{\sqrt{2\pi}}e^{-\frac{1}{2}x^2}.
\]

A remarkable property of these distributions is that, if $f : \R \to \R$ is a real-valued function defined 
as $f(x) = ax+b$, then $f(X) \sim \mathcal N (a\mu + b, \abs{a} \sigma)$.\\

In the same way that we extended random variables to random vectors, we can extend the normal distribution to a multivariate
random distribution.

\begin{ndef}
We say that a random vector $\bm{X} = (X_1,\dots,X_n)$ follows a multivariate normal distributions of parameters
$\mu \in \R^n$, $\Sigma \in \mathcal M_N(\R)$ if, and only if, its probabity density function is:
\[
f(x) = \frac{1}{\sqrt{\det(2\pi \Sigma)}}e^{-\frac{1}{2}(x - \mu )^T \Sigma^{-1} (x-\mu)}.
\]
It is denoted $X \sim \mathcal N(\mu, \Sigma)$.
In this case, $\mu$ is the mean vector of the distribution and $\Sigma$ denotes the covariance matrix.  
\end{ndef}


\section{Parametric Modeling}

In the following chapters, we will be trying to estimate density functions in a dataset. To do this we will be using \emph{parametric models}. We say that a \emph{parametric model}, $p_\theta(x)$, 
is a family of density functions that can be described using a finite numbers of parameters $\theta$. We can get to the concept of \emph{log-likelihood} now.

\begin{ndef}
The \emph{likelihood} $\mathcal L(\theta | x)$ of a parameter set $\theta$ is a function that measures how plausible is $\theta$, given an observed point $x$ in the dataset $\D$. It is defined as the value of the 
density function parametrized by $\theta$ at $x$. That is:
$$
\mathcal L(\theta|x) = p_\theta(x).
$$
\end{ndef}

In a finite dataset $\D$ consisting of independent observations, we can write:
\[
\mathcal L(\theta | X) = \prod_{x \in D} p_\theta(x).
\]

This can be computationally hard to work with, so the log-likelihood is often used instead.

\begin{ndef}
Let $\D$ be a dataset of independent observations and $\theta$ a set of parameters. Then, we define the \emph{log-likelihood} $\ell$ as the sum of the logarithms of the evaluations of $p_\theta$ in each $x$ in the dataset. That is:
\[
\ell (\theta | X) = \sum_{x \in \D} \log p_\theta(x).
\]
\end{ndef}

Our goal would be to find the optimal value $\hat{\theta}$ that maximizes the likelihood of observing the dataset $\D$. We get to the following definition:

\begin{ndef}
    We say that $\hat{\theta} = \hat\theta (\D)$ is a \emph{maximum likelihood estimator}(MLE) for $\theta$ if  
    $$
    \hat\theta \in \argmax_{\theta} \mathcal L(\theta | \D)
    $$
    for every observation $\D$. 
\end{ndef}

\subsection*{Minimal sufficient statistics}

In parametric modeling, the goal was to determine the density function under a distribution. Another interesting task can be determining specific parameters or quantitys related to a distribution 
, given a sample $X = (x_1,\cdots,x_n)$.

\begin{ndef}
    A \emph{statistic} is a measurable function of the data. That is, if $T : \Omega \to \mathbb T$ is measurable, then $T(X)$ is a statistic.
\end{ndef}

However, not all statistics will provide useful information for the statistical inference problem. We would like to find statistics that provide relevant information.

\begin{ndef}
    Let $X \sim P_\theta$. Then, the statistic $T(X) = T : (\Omega, \Alg) \to (\mathbb T, \mathcal B)$, is sufficient for a family of parameters $\{P_\theta \ : \ \theta \in \Theta \}$ if the conditional distribution of $X$, given $T = t$, is indepentent of $\theta$.\\

    Alternatively, we can say $T(X)$ is sufficient for $\theta$ if its mutual information with $\theta$ equals the mutual information between $X$ and $\theta$, that is:
    $$
    I(\theta, X) = I (\theta, T(X))
    $$
\end{ndef}

The easiest example of a sufficient statistic is the mean $\mu$ of a gaussian distribution with known variance. Oppositely, the \emph{median} of an arbitrary distribution
is not sufficient for the mean since, even if the median of the sample is known, more information about the mean of the population can be obtained from the mean of the sample itself.

Although it will not be shown in this document, sufficient statistics are not unique. In fact, if $T$ is sufficient, $\psi(T)$ is sufficient for any bijecte mapping $\psi$. It would be interesting to find a sufficient statistic $T$ that is \emph{the smallest} of them.

\begin{ndef}
    A sufficient statistic $T$ is minimal if, for every sufficient statistic $U$, there exists a mapping $f$ such that $T(x) = f(U(x))$ for any $x \in \Omega$.
\end{ndef}