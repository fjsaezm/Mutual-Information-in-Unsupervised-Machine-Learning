Statistical inference is the process of deducing properties of an underlying distribution by analyzing the data that it is available. With this purpose, techniques like deriving estimates and testing hypotheses are used. 

Inferential statistics are usually contrasted with descriptive statistics, which are only concerned with properties of the observed data. The difference between these two is that in inferential statistics, we assume that the data comes from a larger
population that we would like to know.

In \emph{machine learning}, subject that concerns us the most, the term inference is sometimes used to mean \emph{make a prediction by evaluating an already trained model}, and in this context, inferring properties of the model is refered as \emph{training or learning}.

\section{Parametric Modeling}

In the following chapters, we will be trying to estimate density functions in a dataset. To do this we will be using \emph{parametric models}. We say that a \emph{parametric model}, $P_\theta(x)$, 
is a family of density functions that can be described using a finite numbers of parameters $\theta$. We can get to the concept of \emph{log-likelihood} now.

\begin{ndef}
The \emph{likelihood} $\mathcal L(\theta | x)$ of a parameter set $\theta$ is a function that measures how plausible is $\theta$, given an observed point $x$ in the dataset $\D$. It is defined as the value of the 
density function parametrized by $\theta$ at $x$. That is:
$$
\mathcal L(\theta|x) = P_\theta(x).
$$
\end{ndef}

In a finite dataset $\D$ consisting of independent observations, we can write:
\[
\mathcal L(\theta | X) = \prod_{x \in D} P_\theta(x).
\]

In practice, it is ofter convenient to work with the natural logarithm of the likelihood function. 

\begin{ndef}
Let $\D$ be a dataset of independent observations and $\theta$ a set of parameters. Then, we define the \emph{log-likelihood} $\ell$ as the sum of the logarithms of the evaluations of $p_\theta$ in each $x$ in the dataset. That is:
\[
\ell (\theta | X) = \sum_{x \in \D} \log P_\theta(x).
\]
\end{ndef}

Since the logarithm is a monotonic function, the maximum of $\mathcal L$ and the maximum of its logarithm will occur at the same $\theta$.   Our goal would be to find the optimal value $\hat{\theta}$ that maximizes the likelihood of observing the dataset $\D$. We get to the following definition:

\begin{ndef}
    We say that $\hat{\theta} = \hat\theta (\D)$ is a \emph{maximum likelihood estimator}(MLE) for $\theta$ if  
    $$
    \hat\theta \in \argmax_{\theta} \mathcal L(\theta | \D)
    $$
    for every observation $\D$. 
\end{ndef}

Usually, we seek for likelihood functions that are differentiable, so the derivative test for determining maxima can be applied. Sometimes, the first-order conditions of the likelihood function can be solved explicitly, like in the case of the ordinary least squares estimator which maximizes the likelihood of a linear regression model. However, most of the times, we have to make use of numerical methods to be able to find the maximum of the likelihood function.

There is another concept related to the probability and the set of parameters that the distribution takes:
\begin{ndef}
The \emph{prior probability} is the probability distribution that it is believed to exist before evidence is taken into account.\\

The \emph{posterior probability} is the probability of the parameters $\theta$ given the sampled data $X$, that is, $P(\theta,X)$.
\end{ndef}
The relation with the likelihood function $P(X|\theta)$ is that, given a prior belief that a p.d.f. is $P(\theta)$ and observations $x$ have a likelihood $P(x|\theta)$, the posterior probability is defined using the prior probability as follows:
\[
P(\theta|x) = \frac{P(x|\theta)}{P(x)} P(\theta),   
\]
where we have simply used Bayes' theorem.


\subsection{Generative Models}
From now on, let $\D$ be any kind of observed data. This will always be a finite subset of samples taken from a probability distribution $\pd$. There are models that, given $\D$, try to approximate the 
probability distribution that lies underneath it. These are called \emph{generative models (G.M.)}. 

Generative models can give parametric and non parametric approximations to the distribution $\pd$. 
In our case, we will focus on parametric approximations where the model searches for the parameters that minimize a chosen metric (which can be a distance or other kind of metric such as K-L divergence) between the model distribution and the data distribution. 

We can express our problem more formally as follows. Let $\theta$ be a generative model within a model family $\mathcal M$. The goal of generative models is to optimize:
$$
\min_{\theta \in \mathcal M} d(\pd,p_\theta),
$$
where $d$ stands for the distance between the distributions. We can use, for instance, K-L divergence.

Generative models have many useful applications. We can however remark the tasks that we would like our generative model to be able to do. Those are:
\begin{itemize}
\item Estimate the density function: given a datapoint, $x \in D$, estimate the probability of that point $p_\theta(x)$.
\item Generate new samples from the model distribution $x \sim p_\theta(x)$.
\item Learn useful features of the datapoints.
\end{itemize}

If we have a look again at the example of the zebras, if we make our generative model learn about images of zebras, we will expect our $p_\theta(x)$ to be high for zebra's images. We will also expect the model
to generate new images of this animal and to learn different features of the animal, such as their big size in comparison with cats.

\section{Autoregressive Models}

A very first definition of \emph{autoregressive models (AR)} would be the following one: \emph{autoregressive models are feed-forward models that predict future values using past values}. Let us go deeper into this 
concept and explain how it behaves.

Again, let $\D$ be a set of $n-$dimensional datapoints $x$. We can assume that $x \in \{0,1\}^n$ for simplicity, without losing generality. If we choose any $x\in \D$, using the chain rule of probability, we obtain
$$
p(x) = \prod_{i=1} ^n p(x_i | x_1,\dots,x_{i-1}) = \prod_{i = 1}^n p(x_i|\bm{x}_{<i}), \quad  \text{ where }  \quad \bm{x}_{<i} = [x_1,\dots, x_{i-1}].
$$
We see using that expression how, fixing an order of the variables $x_1,\dots,x_n$, the distribution for the $i$-th random variable depends on all the preceding values in the particular chosen order. 

It is known that given a set of discrete and mutually dependent random variables, they can be displayed in a table of conditional probabilities. If $K_i$ is the number of states that each random variable can take
then $\prod K_i$ is the number of cells that the table will have. If we represent $p(x_i|\bm{x}_{<i})$ for every $i$ in tabular form, we can represent
any possible distribution over $n$ random variables. 

This, however, will cause an exponential growth on the complexity of the representation, because in our case we would need to specify $2^{n-1}$ possibilities 
for each case. In terms of neural networks, since each column must sum $1$ because we are working with probabilities, we have $2^{n-1}-1$ parameters for this conditional, and the tabular representation
becomes impractical for our network to learn.

In autoregressive generative models, the conditionals are specified as we have mentioned before: parameterized functions with a fixed numbers of parameters. More precisely,  we assume 
the conditional distributions to be Bernoulli random variables and learn a function $p_{\theta_i}$ that maps these random variables to the mean of the distribution. Mathematically, we have to find 
$$
p_{\theta_i}(x_i | \bm{x}_{<i}) = Bern(f_i(x_1,\dots,x_{i-1})),
$$
where $\theta_i$ is the set of parameters that specify the mean function $f_i:\{0,1\}^{i-i} \to [0,1]$.

The number of parameters is then reduced to $\sum_{i=1}^n \abs{\theta_i}$, and then we can not represent all possible distributions as we could when using the tabular form of the conditional probabilities.
We are now setting the limit of its expressiveness because we are setting the conditional distributions $p_{\theta_i}(x_i|\bm{x}_{<i})$ to be \emph{Bernoulli} random variables. 

Let us see a very simple case first in order to understand it better and then we will generalize it. Let $\sigma$ be a sigmoid non linear function and 
$\theta_i = \{\alpha_{0}^{(i)},\alpha_{1}^{(i)},\dots, \alpha_{i-1}^{(i)}\}$ the parameters of the mean function. Then, we can define our function $f_i$ as the application of the non linear function to the
sum of the first parameter $\alpha_0^{(i)}$ with the product of each parameter $\alpha_{j}^{(i)}$ with its random variable $x_j$, with $j =1,2,\dots,i-1$ .  That is:
$$
f_i(x_1,\dot, x_{i-1}) = \sigma(\alpha_{0}^{(i)} + \alpha_{1}^{(i)}x_i + \dots + \alpha_{i-1}^{(i)}x_{i-1}).
$$
In this case, the number of parameters would be $\sum_{i = 1}^n i = \frac{n(n+1)}{2}$, so using \emph{Big }$O$ notation, we would be in the case of $O(n^2)$. We will state now a more general and useful case,
giving a more interesting parametrization for the mean function: \emph{multi layer perceptrons}\footnotemark (MLP).

%------------- Footnotemark
\footnotetext{Multi layer perceptrons are feed-forward neural networks with at least 3 layers: input, hidden and output layers; each one using an activation
function.}
%----------------------

For this example we will consider the most simple MLP: the one with one hidden layer. Let $h_i = \sigma(\bm{A}_i \bm{x}_{<i} + c_i)$ be the hidden layer activation function. Remember that $h_i \in \R^d$. Let
$ \theta_i = \{ \bm{A}_i \in \R^{d \times (i-1)}, \ c_i \in \R^d, \ \alpha^{(i)} \in \R^d, \ b_i \in \R\}$ the set of parameters
for the mean function $f_i$, that we define as:
$$
f_i(\bm{x}_{<i}) = \sigma(\alpha^{(i)}h_i + b_i)
$$
In this case, the number of parameters will be $O(n^2 d)$.


\section{Minimal sufficient statistics}

In parametric modeling, the goal was to determine the density function under a distribution. Another interesting task can be determining specific parameters or quantities related to a distribution, given a sample $X = (x_1,\cdots,x_n)$.

\begin{ndef}
    Let $(\Omega,\Alg)$ be a measurable space where $\Alg$ contains all singletons. A statistic is a measurable function of the data, that is: $T: X \to \Omega$ where $T$ is measurable.
\end{ndef}
\begin{remark}
    A statistic is also a random variable.
\end{remark}

However, not all statistics will provide useful information for the statistical inference problem, since almost anything can be a statistic. We would like to find statistics that provide relevant information.

\begin{ndef}
    Let $X \sim P_\theta$. Then, the statistic $T(X) = T : (\Omega, \Alg) \to (\mathbb T, \mathcal B)$, is sufficient for a family of parameters $\{P_\theta \ : \ \theta \in \Theta \}$ if the conditional distribution of $X$, given $T = t$, is indepentent of $\theta$.\\
\end{ndef}

\begin{nexample}
The simplest example of a sufficient statistic is the mean $\mu$ of a gaussian distribution with known variance. Oppositely, the \emph{median} of an arbitrary distribution
is not sufficient for the mean since, even if the median of the sample is known, more information about the mean of the population can be obtained from the mean of the sample itself.
\end{nexample}

Although it will not be shown in this document, sufficient statistics are not unique. In fact, if $T$ is sufficient, $\psi(T)$ is sufficient for any bijective mapping $\psi$. It would be interesting to find a sufficient statistic $T$ that is \emph{the smallest} of them.

\begin{ndef}
    A sufficient statistic $T$ is minimal if, for every sufficient statistic $U$, there exists a mapping $f$ such that $T(x) = f(U(x))$ for any $x \in \Omega$.
\end{ndef}