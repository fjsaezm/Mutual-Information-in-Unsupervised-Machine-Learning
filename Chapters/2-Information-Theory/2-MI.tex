

Using the entropy of a random variable we can directly state the definition of \emph{mutual information} as follows:

\begin{ndef}
Let $X,Z$ be random variables. The \emph{mutual information (MI)} between $X$ and $Z$ is expressed as the difference between the entropy of $X$ and the conditional entropy of $X$ and $Z$, that is:
$$
I(X,Z) := H(X) - H(X|Z).
$$
\end{ndef}

Since the entropy of the random variable $H(X)$ explains the uncertainty of $X$ occurring, the intuitive idea of the \emph{MI} is to determine the decrease of uncertainty of $X$ occurring when we already
know that $Z$ has occurred. We also have to note that, using the definition of the \emph{entropy} and the expression obtained in Eq. \ref{eq:dif-expr-mi}, we can rewrite the \emph{MI}  it follows:
\begin{align*}
I(X,Z) & = \sum_{x \in \X}P_X(x) \log \frac{1}{P(x)} - \sum_{x \in \X, z \in \mathcal Z} P_{XZ}(x,z) \log \frac{P_Z(x)}{P_{XZ}(x,z)} \\  & = \sum_{x,z}P_{XZ}\log \frac{P_Z(z)P_X(x)}{P_{XZ}(x,z)} = D_{KL}(P_{XZ} \ || \ P_X P_Z)
\end{align*}
and we have obtained an expression of the mutual information using the \emph{Kullback-Leibler} divergence. This provides with the following immediate consequences:
\begin{enumerate}[label=$(\roman*)$]
\item Mutual information is non-negative. That is : $I(X,Z) \geq 0$.
\item If $X,Z$ are random variables, then its mutual information equals zero if, and only if, they are independent. 
This easy to check since if $D_{KL}(P_{XZ} \ || \ P_X P_Z) = 0$, then $P_{XZ} = P_X P_Z$ almost everywhere so $X$ and $Z$ are independent.p
\item Since $P_{XZ} = P_{ZX}$ and $P_X P_Z = P_Z P_X$, mutual information is symmetric. That is: $I(X,Z) = I(Z,X)$.
\end{enumerate}

\begin{remark} We can set a connection between the mutual information and sufficient statistics. Let $T(X)$ be a statistic. We say that $T(X)$  is sufficient for $\theta$ if its mutual information with $\theta$ equals the mutual information between $X$ and $\theta$, that is:
$$
I(\theta, X) = I (\theta, T(X)).
$$
This means that sufficient statistics preserve mutual information and conversely.
\end{remark}

\section{Lower bounds on Mutual Information}

Although mutual information seems like a relatively intuitive concept, it is most of the times extremely hard to compute it in real life problems in which the distributions $P(x,z),P(x),P(z)$ are not known.

\begin{nexample}
Let $x$ represent an image of size $n \times m$ pixels. Then, the dimension of the single image is $n \cdot m \cdot 3$, for RGB color channels. In these cases, there is no easy way of calculating $P(x)$.
\end{nexample}

Due to this problem related to the \emph{Curse of Dimensionality}, we can try to compute lower bounds of it that are generally easier to calculate. We will now expose two general lower bounds, and we will focus on a third one that will be explained later in this work.

\subsection*{Variational Lower Bound}



Using the expression of the mutual information in terms of entropy, $I(x,z) = H(z) - H(z|x)$, we can give a lower bound on $I(x,z)$ as a function of a probability distribution $Q_\theta(z|x)$. 

\begin{nprop}
Let $X,Z$ be random variables and $Q_\theta(z|x)$ be an arbitrary probability distribution. Then,
$$
I(x,z) \geq H(z) + E_{P_X} \left[ E_{P_{X|Z}}\left[\log Q_\theta(z|x)\right]\right] 
$$
\end{nprop}

\begin{proof}
Recalling that
$$
H(z|x) = - E_{P_{XZ}} \left[ \log P(x,z) - \log P(x)\right],
$$
and that
\begin{align*}
E_{P(x,z)}\left[\log\frac{P(x,z)}{P(x)}\right] & =  \sum_{x,z} P(x,z) \log\frac{P(x,z)}{P(x)} \\ 
& = \sum_{x,z} P(x)P(z|x) \log P(z|x) = \sum_{x,z} P(x) E_{P(z|x)}[\log P(z|x)]\\
 & =  E_{P(x)}\left[E_{P(z|x)}[\log P(z|x)]\right],
\end{align*}
we only have to use the definition of the conditional probability to see that:
\begin{align*}
I(x,z) & =  H(z) - H(z|x) \\
    & =  H(z) + E_{P(x,z)} = H(z) + E_{P(x,z)} \left[ \log \frac{P(x,z)}{P(x)}\right] \\
    & =  H(z) + E_{P(x)} \left[ E_{P(x|z)}\left[\log P(z|x)\right]\right] \\
    & = H(z) + E_{P(x)} \left[ E_{P(x|z)} \left[\log \frac{P(z|x)}{Q_\theta(z|x)}\right] + E_{P(z|x)}\left[\log Q_\theta(z|x)\right]\right] \\
    & =  H(z) + E_{P(x)}\left[ \underbrace{D_{KL}(P(z|x)||Q_\theta(z|x))}_{\geq 0} + E_{P(z|x)}\left[\log Q_\theta(z|x)\right] \right]\\
    & \geq H(z) + E_{P(x)}\left[E_{P(z|x)}\left[ \log Q_\theta(z|x)\right]\right].
\end{align*}
We have taken advantage of the non-negativity of the KL-Divergence.
\end{proof}

Using this bound, and combining this theoretical knowledge with machine learning methods, such as \emph{backpropagation}, we can make $Q_\theta$ be a neural network and maximize this lower bound.

\subsection*{Donsker-Varadhan Representation}

We can also give a lower bound on the mutual information using its KL-Divergence formulation. Firstly, we have to 

\begin{nth}[Donsker-Varadhan]
The KL divergence admits the following dual representation:
\[
D_{KL}(P || Q) = \sup_{T} E_P[T] - \log E_Q[e^T],
\]
where the supremum is taken over all functions $T:\Omega \to \R$ such that both expectations exist.
\end{nth}
\begin{proof}
    TODO
\end{proof}

Using this representation, we reach this lower bound. Let $\mathcal F$ be any class of functions $T: \Omega \to \R$ satisfying the integrability constraints of the theorem. Then, 
$$
I(P,Q) = D_{KL}(P||Q) \geq \sup_{T \in \mathcal F} E_P[T] - \log E_Q[e^T].
$$