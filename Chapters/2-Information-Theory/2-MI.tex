

Using the entropy of a random variable we can directly state the definition of \emph{mutual information} as follows:

\begin{ndef}
Let $X,Z$ be random variables. The \emph{mutual information (MI)} between $X$ and $Z$ is expressed as the difference between the entropy of $X$ and the conditional entropy of $X$ and $Z$, that is:
$$
I(X,Z) := H(X) - H(X|Z).
$$
\end{ndef}
Since the entropy of the random variable $H(X)$ explains the uncertainty of $X$ occurring, the intuitive idea of the \emph{MI} is to determine the decrease of uncertainty of $X$ occurring when we already
know that $Z$ has occurred. We also have to note that, using the definition of the \emph{entropy} and the same argument that we used to obtain the expression in Eq. \ref{eq:dif-expr-mi}, we can rewrite the \emph{MI}  it follows:
\begin{align}
I(X,Z) & = \sum_{x \in \X}P_X(x) \log \frac{1}{P(x)} - \sum_{x \in \X, z \in \mathcal Z} P_{XZ}(x,z) \log \frac{P_Z(x)}{P_{XZ}(x,z)} \nonumber \\
 & = \sum_{x,z}P_{XZ}\log \frac{P_{XZ}(x,z)}{P_Z(z)P_X(x)} \label{MI:sum:xz}
\end{align}
and if we compare it to the formula of the KL-Divergence, we obtain:
\[
I(X,Z)  = \sum_{x,z}P_{XZ}\log \frac{P_{XZ}(x,z)}{P_Z(z)P_X(x)} = D_{KL}(P_{XZ} \ || \ P_X P_Z),
\]
so we have obtained an expression of the mutual information using the \emph{Kullback-Leibler} divergence. This provides with the following immediate consequences:

\begin{corollary}

\begin{enumerate}[label=$(\roman*)$]
\item Mutual information is non-negative. That is : $I(X,Z) \geq 0$.
\item If $X,Z$ are random variables, then its mutual information equals zero if, and only if, they are independent. 

\item Mutual information is symmetric. That is: $I(X,Z) = I(Z,X)$.
\end{enumerate}
\end{corollary}

\begin{proof}
    \begin{enumerate}[label=$(\roman*)$]
        \item This is trivial using Prop \ref{entr:prop:2} and the definition of the mutual information.
        \item We can use the KL-Divergence formulation to see that since $$
        D_{KL}(P_{XZ} \ || \ P_X P_Z) = 0 \implies P_{XZ} = P_X P_Z,
        $$ 
        almost everywhere then $X$ and $Z$ are independent.
        \item It is a consequence of the fact that $P_{XZ} = P_{ZX}$ and $P_X P_Z = P_Z P_X$.
        \end{enumerate}
\end{proof}

Later in this document, we will have some sort of random variable $X$ and would like it to maintain the mutual information with itself after being applied a function. The following proposition will be useful:

\begin{nprop}
Let $X,Z$ be random variables. Then, $I(X,Z)$ is invariant under homeomorphism.
\end{nprop}
\begin{proof}
Let $\phi(x)$ be an homeomorphism, i.e., a continuous, monotonic function with $\phi^{-1}(x)$ also continuous and monotonic. Let $X$ be a random variable and $Y$ another one such $y = \phi(x)$ if $x = X(\omega)$ for some $\omega \in \Omega$. Then, if $S$ is a particular subset we have 
\[
P(Y \in S) = \int_S P_Y(y) dy = \int_{\phi^{-1}(S)}P_X(x) dx \stackrel{(1)}{=} \int_S P_X(\phi^{-1}(y)) \abs{ \frac{d \phi^{-1}}{dy}}dy,
\]
where in $(1)$ we have changed from $x$ to $y$. Hence, 
\[
P_Y(y) = P_X(\phi^{-1}(y))\abs{\frac{d \phi^{-1}}{dy}}.
\]
As a consequence of this, $I(X,Z) = I(\phi(X),Z) $ for any homeomorphism $\phi$. By symmetry, the same holds for $Z$.

\end{proof}

\begin{remark} We can set a connection between the mutual information and sufficient statistics. Let $T(X)$ be a statistic. We say that $T(X)$  is sufficient for $\theta$ if its mutual information with $\theta$ equals the mutual information between $X$ and $\theta$, that is:
$$
I(\theta, X) = I (\theta, T(X)).
$$
This means that sufficient statistics preserve mutual information and conversely.
\end{remark}

\section{Lower bounds on Mutual Information}

Although mutual information seems like a relatively intuitive concept, it is most of the times extremely hard to compute it in real life problems in which the distributions $P(x,z),P(x),P(z)$ are not known.
\begin{nexample}
Let $x$ represent an image of size $n \times m$ pixels. Then, the dimension of the single image is $n \cdot m \cdot 3$, for RGB color channels. In these cases, there is no easy way of calculating $P(x)$.
\end{nexample}

Due to this problem related to the \emph{Curse of Dimensionality}, we can try to compute lower bounds of it that are generally easier to calculate. We will now expose two general lower bounds, and we will focus on a third one that will be explained later in this work.

\subsection*{Variational Lower Bound}



Using the expression of the mutual information in terms of entropy, $I(x,z) = H(z) - H(z|x)$, we can give a lower bound on $I(x,z)$ as a function of a probability distribution $Q_\theta(z|x)$. 

\begin{nprop}
Let $X,Z$ be random variables and $Q_\theta(z|x)$ be an arbitrary probability distribution. Then,
$$
I(x,z) \geq H(z) + E_{P_X} \left[ E_{P_{X|Z}}\left[\log Q_\theta(z|x)\right]\right] 
$$
\end{nprop}

\begin{proof}
Recalling that
$$
H(z|x) = - E_{P_{XZ}} \left[ \log P(x,z) - \log P(x)\right],
$$
and that
\begin{align*}
E_{P(x,z)}\left[\log\frac{P(x,z)}{P(x)}\right] & =  \sum_{x,z} P(x,z) \log\frac{P(x,z)}{P(x)} \\ 
& = \sum_{x,z} P(x)P(z|x) \log P(z|x) = \sum_{x,z} P(x) E_{P(z|x)}[\log P(z|x)]\\
 & =  E_{P(x)}\left[E_{P(z|x)}[\log P(z|x)]\right],
\end{align*}
we only have to use the definition of the conditional probability to see that:
\begin{align*}
I(x,z) & =  H(z) - H(z|x) \\
    & =  H(z) + E_{P(x,z)} = H(z) + E_{P(x,z)} \left[ \log \frac{P(x,z)}{P(x)}\right] \\
    & =  H(z) + E_{P(x)} \left[ E_{P(x|z)}\left[\log P(z|x)\right]\right] \\
    & = H(z) + E_{P(x)} \left[ E_{P(x|z)} \left[\log \frac{P(z|x)}{Q_\theta(z|x)}\right] + E_{P(z|x)}\left[\log Q_\theta(z|x)\right]\right] \\
    & =  H(z) + E_{P(x)}\left[ \underbrace{D_{KL}(P(z|x)||Q_\theta(z|x))}_{\geq 0} + E_{P(z|x)}\left[\log Q_\theta(z|x)\right] \right]\\
    & \geq H(z) + E_{P(x)}\left[E_{P(z|x)}\left[ \log Q_\theta(z|x)\right]\right].
\end{align*}
We have taken advantage of the non-negativity of the KL-Divergence.
\end{proof}

Using this bound, and combining this theoretical knowledge with machine learning methods, such as \emph{backpropagation}, we can make $Q_\theta$ be a neural network and maximize this lower bound.

\subsection*{Donsker-Varadhan Representation}

We can also give a lower bound on the mutual information using its KL-Divergence formulation. Firstly, we have to 

\begin{nth}[Donsker-Varadhan]
The KL divergence admits the following dual representation:
\[
D_{KL}(P || Q) = \sup_{T} E_P[T] - \log E_Q[e^T],
\]
where the supremum is taken over all functions $T:\Omega \to \R$ such that both expectations exist.
\end{nth}
\begin{proof}
    TODO
\end{proof}

Using this representation, we reach this lower bound. Let $\mathcal F$ be any class of functions $T: \Omega \to \R$ satisfying the integrability constraints of the theorem. Then, 
$$
I(P,Q) = D_{KL}(P||Q) \geq \sup_{T \in \mathcal F} E_P[T] - \log E_Q[e^T].
$$

\subsection*{Contrastive Lower Bound}

In chapter \ref{Chapter:NCE} we presented Noise Contrastive Estimation, that tried to discriminate between elements of two different sets. One was composed of data, $X$, and the other one was composed of noise, $Y$.

Let $(x,z)$ be a data representation drawn from a distribution $P(x,z)$ and $x'$ be some other data drawn from the distribution $P(x)$. Using NCE, we should be able to say that $(x,z)$ was drawn from the distribution $P(x,z)$ (which was $P_d$ in the NCE theory) while $(x',z)$ was drawn from the product of the marginal distributions $P(x)P(z)$ (which was $P_n$ in the explanation of NCE). Let $h_\theta$ be a model that helps us to do this discrimination, with parameters $\theta$. 

As we did before, we want to estimate the ratio $P_d/P_n$ of the different distributions, in this case the ratio would be $P(x,z)/P(x)P(z)$. Let $(x^*,z)$ be a pair drawn from $P(x,z)$ and $X = \{x^*, x_1,\cdots,x_{N-1} \}$, where the rest of the $N-1$ points form pairs $(x_j,z)$ drawn from $P(x)P(z)$ the product of the marginal distribution. We can rewrite the loss \ref{log:likelihood:red} in a simpler expression:
\begin{equation}\label{log:likelihood:rewritten}
l(\theta) = E_X \left[ \log \frac{h_\theta(x^*,z)}{\sum_{x \in X}h_\theta(x,z)}\right]  .
\end{equation}

If we maximize this objective, $h_\theta$ learns to discriminate $(x^*,z)$ from $(x_j,z)$ for $ 1 \leq j < N$ and, thus, we are learning to estimate the ratio $P(x,z)/P(x)P(z)$. Let us see how maximizing $\ell(\theta)$ we are maximizing a lower bound for $I(x,z)$.

\begin{nprop}
Let $X = \{x^*, x_1,\cdots,x_{N-1} \}$, where $x^* \sim P(x,z)$ and the rest of them were sampled from $P(x)P(z)$. Then,
\[
I(x,z) \geq \ell(\theta) + \log N
\]
\end{nprop}
\begin{proof}

Firstly, using Bayes' rule, $P(x^*,z) = P(x^*|z)P(z)$. Hence, since $h_\theta$ estimates $P(x^*,z)/P(x)P(z)$, it also estimates
\[
\frac{P(x^*,z)}{P(x)P(z)} = \frac{P(x^*|z)P(z)}{P(x)P(z)} = \frac{P(x^*|z)}{P(x)}.
\]
Using the definition of the log-likelyhood that we see in \ref{log:likelihood:rewritten}, we see that
\begin{align*}
E_X \left[ \log \frac{h_\theta(x^*,z)}{\sum_{x \in X}h_\theta(x,z)}\right] & =  E_X \left[ \log \frac{h_\theta(x^*,z)}{ h_\theta(x^*,z) + \sum_{j = 1}^{N-1} h_\theta(x_j,z)}\right] \\
& \approx E_X \left[ \log \frac{\frac{P(x^*|z)}{P(x)}}{ \frac{P(x^*|z)}{P(x)} + \sum_{j = 1}^{N-1} \frac{P(x_j|z)}{P(x)}}\right].
\end{align*}
Now, using that $\log(a) = -log(a^{-1})$,
\begin{align*}
E_X \left[ \log \frac{\frac{P(x^*|z)}{P(x)}}{ \frac{P(x^*|z)}{P(x)} + \sum_{j = 1}^{N-1} \frac{P(x_j|z)}{P(x)}}\right] & = E_X\left[ -\log\left( \frac{\frac{P(x^*|z)}{P(x)} + \sum_{j = 1}^{N-1} \frac{P(x_j|z)}{P(x)}}{\frac{P(x^*|z)}{P(x)}}\right) \right] \\
& = E_X\left[ -\log\left( 1+ \frac{ \sum_{j = 1}^{N-1} \frac{P(x_j|z)}{P(x)}}{\frac{P(x^*|z)}{P(x)}} \right) \right] \\
& = E_X\left[ -\log\left( 1+ \frac{ (N-1) E_{X - \{x^*\}}\left[\frac{P(x|z)}{P(x)}\right] }{\frac{P(x^*|z)}{P(x)}}\right)\right] \\
\end{align*}
Now, since $E_{X - \{x^*\}} \left[ \frac{P(x|z)}{P(x)} \right] = \sum_{x_j \in X - \{x^*\}} P(x_j) \frac{P(x_j|z)}{P(x_j)} = 1$, then
\[
E_X\left[ -\log\left( 1+ \frac{ (N-1) E_{X - \{x^*\}}\left[\frac{P(x|z)}{P(x)}\right] }{\frac{P(x^*|z)}{P(x)}}\right)\right]  
= E_X\left[ -\log\left( 1+ \frac{ (N-1)}{\frac{P(x^*|z)}{P(x)}} \right)\right].
\] 
Lastly, using that if $k > 0$, then $- \log a(k+1) \geq -\log(1+ak)$, we obtain:
\begin{align*}
E_X\left[ -\log\left( 1+ \frac{ (N-1)}{\frac{P(x^*|z)} {P(x)}} \right)\right] &  =
E_X\left[ \log \frac{1}{ 1+ \frac{P(x^*)}{P(x^*|z)}(N-1)}\right] \\
\leq E_X\left[ \log\left(\frac{1}{\frac{P(x^*)}{P(x^*|z)}}\frac{1}{N}\right)\right]\\
& = E_X\left[ \log\left(\frac{P(x^*|z)}{P(x^*)} \frac{1}{N}\right)\right]\\
& = E_X \left[ \log\left(\frac{P(x^*|z)}{P(x^*)}\right)\right] - \log N \\
& \stackrel{(1)}{=} E_X \left[ \log\left(\frac{P(x^*,z)}{P(x^*)P(z)}\right)\right] - \log N\\
& \stackrel{(2)}{=} I(x,z) - \log N,
\end{align*}
where, in $(1)$, we have use Bayes' rule again and in $(2)$ we have used the definition of the MI that we found in equation \ref{MI:sum:xz}.
\end{proof}

