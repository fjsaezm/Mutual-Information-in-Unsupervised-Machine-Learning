// TODO - Change name to InfoNCE LOSS

We are now ready to connnect the concepts of mutual information and generative models that we have presented. In unsupervised learning,
it is a common strategy to predict future information and to try to find out if our predictions are correct.
In \emph{natural language processing},for instance, representations are learned 
using neighbouring words \citep{mikolov_efficient_2013}, and in images, some studies have been able to predict color from grey-scale \citep{doersch_unsupervised_2016}.

When we talk about high-dimensional data, it is not useful to make use of an unimodal loss function to evaluate our model. If we did it like this, we would be assuming that there is only
one peak in the distribution function and that it is actually similar to a Gaussian.  This is not always true, so we can not assume it for our models. Generative models can be used for this purpose:
they will model the relationships in the data $x$. However,they ignore the context $c$ in which the data $x$ is involved. As an easy example of this, an image contains thousands of bits of information,
while the label that classifies the image contains much less information , say, $10$ bits for $1024$ categories. Because of this, modeling $p(x|c)$ might not be the best way to proceed if we want
to obtain the real distribution that generates our data. 

Our goal here will be to seek for a way of extracting shared information between the context and the data. Due to the differences in data dimensionality, if we want to predict the future $x$ using the context $c$, firstly we must
encode our entry data $x$ into a representation which size is comparable to context size. Firstly, an \emph{encoder} is used. An encoder is a model that, given an input $x$, provides a feature map or vector that holds the information
that the input $x$ had. In fact, and here is where we link the mutual information with the current topic, we want our encoder to maximize
\begin{equation}\label{EQ:MI}
I(x,c) = \sum_{x,c}p(x,c)\log\frac{p(x|c)}{p(x)}
\end{equation}
that is, the mutual information between the input $x$ and the context $c$.  Maximizing the mutual information between $x$ and $c$, we extract the latent variables
that the inputs have in common.

So, we will use an encoder $g_{enc}$ that transforms the input sequence of observations $x_t$ to a sequence of latent representations
$$
z_t = g_{enc}(x_t).
$$
After we have obtained $z_t$, we use it as input of an autoregressive model to produce a context latent representation:
$$
c_t = g_{ar}(z_{\leq t}).
$$
In this case, $c_t$ will summarize the information of $z_i$ for $i \leq t$. Following the argument that we gave before, predicting the future $\xtk$ using only a generative model, say $p_k(\xtk|c)$
might not be correct, since we would be ignoring the context. \\

\begin{figure}[H]
    \centering 
    \includegraphics[scale=0.4]{contrastive_repr4.pdf}
    \caption{Image from \citep{oord_representation_2019}. Overview of Contrastive Predictive Coding framework using audio signal as input. }
\end{figure}



Let us see how we train the encoder $g_{enc}$ and the autoregressive model $g_{ar}$.\\

Let $X = \{x_1,\dots,x_N\}$ be a set of $N$ random samples. $X$ will contain positive sample taken from the distribution $p(x_{x+k}|c_t)$ and $N-1$ negative
samples from the distribution propused $p(\xtk)$. With this set, we would like to optimize the following loss function:
\begin{equation}\label{NCE:loss}
\mathcal L_N = - \underset{X}{E} \left[\log \frac{f_k(\xtk,c_t)}{\sum_{x_j \in X} f_k(x_j,c_k)}\right].
\end{equation}

Let us have a look at the \emph{categorical cross-entropy} loss function:
\[
    \text{CE} = -\sum_i^C t_i \log(f(s_i))    
\]
where $C$ is the number of possible classes in a classification problem, $t_i$ are the groundtruth of each class and $f(s_i)$ is the application of an activation function (sigmoid,softmax) to the the score $s_i$ of each class.

We can say that $\mathcal L_N$ is no more than the categorical cross-entropy of classifying the positive sample correctly, with the argument of the logarithm being the prediction
of the model. If we note with $[d = i]$ as an indicator of the sample $x_i$ being the positive sample in $X$, the optimal probability for this loss can be written as $p(d = i|X,c_t)$. 

Now, the probability that $x_i$ was drawn from the conditional distribution $p(\xtk|c_t)$ that has the context in account, rather than the proposal distribution $p(\xtk)$ that does not have $c_t$ in account, leads us to the following expression:
%\begin{align*}
%p(d = i | X , c_t) = & \frac{p(x_i|c_t) \prod_{l \neq i}p(x_l)}{\sum_{j=1}^N p(x_j|c_t) \prod_{l \neq j} p(x_l)}\\
%                   = & \frac{\frac{p(x_i|c_t)}{p(x_i)}}{\sum_{j=1}^N \frac{p(x_j|c_t)}{p(x_j)}}.
%\end{align*}
$$
p(d = i | X , c_t) = \frac{ \frac{p(x_i|c_t)}{p(x_i)}}{\sum_{j=1}^N \frac{p(x_j|c_t)}{p(x_j)}}.
$$

This is the optimal case for \eqref{NCE:loss}. Lets provide with further explanation on this formula. Firstly, it is necessary to see that:
$$
\begin{WithArrows}
p( d = i | X) & =   \frac{p(d=i,X)}{p(X)} \\
              & =  \frac{X | d = i}{p(X)} \\
              & =  \frac{p(X | d = i)}{\sum_{j = 1}^N p(X|d = j) p(d = j)} \Arrow{$\left(\ast\right)$}\\
              & =  \frac{D(i) p(x_i) \prod_{j \neq i}q(x_j)}{\sum_{j=1}^N D(j) p(x_j)\prod_{k \neq j}q(x_k)} \Arrow{assume D is uniform} \\
              & =  \frac{p(x_i)\prod_{j \neq i}q(x_j)}{\sum_{j=1}^N p(x_j)\prod_{k \neq j}q(x_k)}, 
\end{WithArrows}
$$

Where, in $\left(\ast\right)$, we have used that since $p(X|d=i)$ reffers to the probability of the sample $X$ given that $d = i$ which means that $x_i$ has been extracted from $p(\xtk|c_t)$ and 
$x_j$ for $j \neq i$ have been extracted from $p(\xtk)$ (noted $q$ in the formula to remark the difference), that means $p(X|d=i) = p(x_i)\prod_{j\neq i} q(x_j)$. Also, we can assume that $D$ is uniform because each $x_i$ with $i = 1,\dots,N$ 
has the same probability to have been chosen to be the positive sample in $X$.


In fact, if we denote $f(\xtk,c_t)$  as the density ratio that preserves the mutual information between $\xtk$ and $c_t$ in the mutual information definition \eqref{EQ:MI}, we have just proved that if $k$ is a step in the experiments, then
\begin{equation}\label{EQ:Proportional}
f_k(\xtk,c_t)  \propto \frac{p(t_{x+k}| c_t)}{p(\xtk)},
\end{equation}
where $\propto$ means that the member on the left is proportional to the member on the right. We can see that the optimal value $f_k(\xtk,c_t)$ does not depend on $N-1$, the number of negative samples in $X$.
Using this density ratio, we are relieved from modeling the high dimensional distribution $x_{t_k}$. In \citep{oord_representation_2019}, paper that we have followed and tried to explain in this document, for instance, the following log-bilinear model expression is used:
$$
f_k(\xtk,c_t) = exp(z_{t+k}^T W_k c_t).
$$


In the proposed model, we can either use the representation given by the encoder $(z_t)$ or the representation given by the autoregressive model $(c_t)$ for downstream tasks. Clearly, the representation that aggregates information from past inputs will be more useful
if more information about the context is needed. Furthermore, any type of models for the encoder and the autoregressive models can be used in this kind of framework.

Now, it will be shown how optimizing the loss presented in \ref{NCE:loss}, we are maximizing mutual information between $c_t$ and $z_{t+k}$. Using that the optimal value for $f(\xtk,c_t)$ was proven to be $\frac{p(\xtk|c_t)}{p(\xtk)}$, if we have that:
\[
L_N^{\operatorname{opt}} = - E_X \log \left[ \frac{\frac{p(\xtk|c_t)}{p(\xtk)}}{ \sum_{x_j \in X} \frac{p(x_{j}|c_t)}{p(x_{j})}} \right].
\]
We can split the denominator in the positive sample $\xtk$ and the negative samples, and use that $-log(a) = log(a^{-1})$ to obtain:
\[
L_N^{\operatorname{opt}} = E_X \left[\frac{ \frac{p(\xtk|c_t)}{p(\xtk)} + \sum_{x_j \in X_{\operatorname{neg}}} \frac{p(x_j|c_t)}{p(x_j)}  } {\frac{p(\xtk|c_t)}{p(\xtk)} } \right] = E_X \log \left[ 1+ \frac{\sum_{x_j \in X_{\operatorname{neg}}}\frac{p(x_j|c_t)}{p(x_j)}}{\frac{p(\xtk)}{p(\xtk|c_t)}} \right],
\]
and, since the negatives are uniformly distributed, this we obtain:
\[
E_X \log \left[ 1+ \frac { \sum_{x_j \in X_{\operatorname{neg}}} \frac{p(x_j|c_t)}{p(x_j)}} {\frac{p(\xtk)}{p(\xtk|c_t)}} \right] \approx E_X \log \left[ 1+ \frac{ (N-1) E_{x_j}  \left[ \frac{p(x_j|c_t)}{p(x_j)}\right]}{\frac{p(\xtk)}{p(\xtk|c_t)}} \right].
\]
If we observe that $E_{x_j} \left[ \frac{p(x_j|c_t)}{p(x_j)} \right] = \sum_{x_j} p(x_j) \frac{p(x_j|c_t)}{p(x_j)} = 1$, then we can derive the last expression as follows:
\begin{align*}
E_X \log \left[ 1+ \frac{ (N-1) E_{x_j}  \left[ \frac{p(x_j|c_t)}{p(x_j)}\right]}{\frac{p(\xtk)}{p(\xtk|c_t)}} \right] =  & E_X \log \left[ 1 + \frac{N-1}{ \frac{p(\xtk|c_t)}{p(\xtk)}}\right]\\
 = & E_X \log \left[ 1+ (N-1)\frac{p(\xtk)}{p(\xtk|c_t)}\right]
\end{align*}
