%------------------------
% Math libraries
%------------------------
\usepackage{amsthm}
\usepackage{amsmath}
\usepackage{tikz}
\usepackage{tikz-cd}
\usetikzlibrary{shapes,fit}
\usepackage{bussproofs}
\EnableBpAbbreviations{}
\usepackage{mathtools}
\usepackage{scalerel}
\usepackage{stmaryrd}

%------------------------
% Estilos para los teoremas
%------------------------
\theoremstyle{plain}
\newtheorem{nth}{Theorem}[section]
% For theorems not in sections
\newtheorem{nthC}{Theorem}[chapter]

\newtheorem{nprop}{Proposition}
\newtheorem{lemma}{Lemma}
\newtheorem{corollary}{Corollary}
\theoremstyle{definition}
\newtheorem{ndef}{Definition}[section]
% For definitions not in sections
\newtheorem{ndefC}{Definition}[chapter]

\newtheorem{nproof}{Proof}

\theoremstyle{remark}
\newtheorem{remark}{Remark}
\newtheorem{nexample}{Example}


\theoremstyle{notation}
\newtheorem{notation}{Notation}


\begingroup\makeatletter\@for\theoremstyle:=definition,ndefC,remark,plain\do{\expandafter\g@addto@macro\csname th@\theoremstyle\endcsname{\addtolength\thm@preskip\parskip}}\endgroup

%------------------------
% Macros
% ------------------------

%Frequently used mathematical commands
\newcommand*\diff{\mathop{}\!\mathrm{d}}
\newcommand{\R}{\mathbb{R}}
% \newcommand{\E}{\mathbb{E}}
\newcommand{\bmu}{\bm{\mu}}
\newcommand{\bx}{\bm{x}}
\newcommand{\bX}{\bm{X}}
\newcommand{\bz}{\bm{z}}
\newcommand{\bZ}{\bm{Z}}
\newcommand{\bv}{\bm{v}}
\newcommand{\bh}{\bm{h}}
\newcommand{\bSigma}{\bm{\Sigma}}
\newcommand{\bpi}{\bm{\pi}}
\newcommand{\bLambda}{\bm{\Lambda}}
\newcommand{\btheta}{\bm{\theta}}

\newcommand{\V}{\mathcal{V}}
\newcommand{\D}{\mathcal{D}}
\newcommand{\X}{\mathcal{X}}
\newcommand{\I}{\mathcal{I}}
\newcommand{\Y}{\mathcal{Y}}

\newcommand\ddfrac[2]{\frac{\displaystyle #1}{\displaystyle #2}}

\newcommand\E[2]{\mathbb{E}_{#1}\Big[#2\Big]}
\newcommand\KL[2]{KL\Big(#1 \bigm| #2\Big)}
\newcommand{\bigCI}{\mathrel{\text{\scalebox{1.07}{$\perp\mkern-10mu\perp$}}}}
\newcommand{\bigCD}{\centernot{\bigCI}}

\DeclareMathOperator*{\argmax}{arg\,max}
\DeclareMathOperator*{\argmin}{arg\,min}

% My own commands:

\newcommand{\Prob}{\mathcal{P}}
\newcommand{\Alg}{\mathscr{A}}
\newcommand{\N}{\mathbb N}
\newcommand{\A}{\mathcal A}
%\newcommand{\abs}[1]{\lvert#1\rvert}  
\newcommand{\rv}{\mathbf{X}}
\newcommand{\rvc}{\mathbf{X} = (X_1,\dots,X_n)}
\newcommand{\pd}{p_{\text{data}}}
\newcommand{\xtk}{x_{t+k}}
\newcommand{\ps}{x^+}
\newcommand{\ns}{x^-}

\newcommand{\norm}[1]{\left\lVert#1\right\rVert}
\newcommand{\abs}[1]{\left\lvert#1\right\rvert}


% Set Graphics Path
\usepackage{float}
\graphicspath{{media/}}

% Math operators

\DeclareMathOperator*{\Var}{Var}

\DeclareMathOperator*{\Cov}{Cov}

% For resnet tables

\newcommand{\blockb}[3]{\left[\begin{array}{c}\text{1$\times$1, #2}\\[-.1em] \text{3$\times$3, #2}\\[-.1em] \text{1$\times$1, #1}\end{array}\right]\times#3}




